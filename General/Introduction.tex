\section{Introduction}

Sichua est le centre du monde civilisé, c'est à dire peu de chose en ce bas monde.
Mais si il existe une civilisation des races humanoïdes, c'est à Sichua qu'en est 
son centre. C'est autour de l'estuaire du fleuve Tervyn que les petits immeubles 
s'entremêlent sans fin dans une brume grise. Le bruit
incessant des vendeurs de rues et des marteaux dans les forges s'y mêlent aux cris
des marins sur les bateaux et des oiseaux qui les poursuivent. À Sichua 
tout semble chaotique à l'étranger nouvellement arrivé mais
s'organise en fait comme une fourmillière où chacun tient son rôle. Sur l'ile de 
la source, où la ville prit son origine, les élites dirigent. Au nord, le port 
militaire, assure la domination de Sichua sur les mers. En levant la tête, on 
observe la tour phare géante, siège de l'archimage de Sichua. Vers le sud, les 
ruelles débouchent bientôt sur de grandes places où s'installent des marchés qui
semblent interminables. Mais continuez dans cette direction et vous déboucherez
sur la misère la plus sombre ou l'opulence la plus vulgaire selon que vous allez 
au pied de la colline ou à son sommet. Car Sichua est la ville des contrastes
et vous y trouverez toute les fortunes et toutes les races civilisées. Enfin,
c'est la ville de toutes les opportunités, mais méfiez vous, vous n'êtes ni les 
premiers, ni les derniers à rêver de gloire et de fortune en ces murs.


