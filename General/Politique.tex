\section{Organisation Politique}

\subsection{Dirigeants}

   \paragraph{Pentumvirat ou conseil des cinq: }
       Ils se réunissent régulièrement pour décider des affaires de la 
       ville. Ils peuvent décréter de nouvelles loi ou en abolir d'ancienne,
       levé des impots ou la mobilisation générale en cas de guerre. \\

   \paragraph{Duc de Sichua, représentant des nobles: }
       En principe un noble élu par ses pairs, en fait toujours le duc \\
   \paragraph{Prévost des marchands: }
       Élu par les marchands payant des taxes \\
   \paragraph{Grand prètre: } 
       Souvent celui de la justice, mais élu aussi \\
   \paragraph{Archimage: }
       Gardien de la tour phare et de l'école de magie \\
   \paragraph{Représentant de la plèbe (nom romain?): }
       Élu indirectement, il est souvent affilié à la guilde des voleurs \\

   \paragraph{Les chambellans:} élus dans chaque quartier, ils servent de référents
       entre la population et l'aristocratie. Ils sont généralement des
       personnages locaux importants: prètres, riches marchands, noble
       ou mafieu local. Ils élisent le représentant des plébeiens. \\

\subsection{Justice}

Les textes de lois décrétés par le conseil sont conservés par la 
bibliothèques du temple du dieu de la justice \\

  \paragraph{Cour commerciale}
    Au palais du prévost, gère tous les conflits civils lié aux contrats
    au vol, aux marriages etc. \\
  \paragraph{Cour ducale}
    Au palais ducal, gère les crimes de sang et politiques. \\
  \paragraph{Cour religieuse}
    Au temple de la justice, gère les crimes religieux hérésie, démonologie
    et diableries. \\
  \paragraph{Conseil de l'ordre des mages}
    Sanctionne principalement les mauvais élèves de l'école de magie, en
    principe il peut exiler tout mage qui se comporterait de manière 
    dangeureuse pour le publique: création d'objets vicieux ou expériences 
    magiques dangereuses. \\
  \paragraph{Cour du chambellan }
    Elle règle tous les petits littiges du quotidien. C'est 
    plus un lieu de médiation que de justice. Ses décisions peuvent
    toujours être contsté dans une autre cour supérieure. Ces cours peuvent
    se saisir de n'importe quelle affaire jugé ou non par un chambellan.
    Dans les faits, les affaires d'importances ne passe jamais par ce 
    niveau de justice. \\
  \paragraph{Cour suprème}
    Il en coute une fortune de la convoquer et on reste généralement en 
    prison en attendant, la cour suprème réunit le pentumvirat et juge
    en dernière instance. Dans des cas excepionnels, ils jugent les 
    affaires les plus importantes et les plus secrètes. \\

\subsection{Taxes}
  \paragraph{Taille} levé par le duc par quartier/foyer
    Gardes et entretiens des murs \\
  \paragraph{Dime agricole} pour les prêtres
    Soins et entretien des temples \\
  \paragraph{Payages} sur les marchandises passant les portes de la ville, 
                      dans les ports et sur les ponts. Elle est collectée 
                      par le prevost et elle sert à
    l'entretiens des routes, des ports et la sécurité sur les marchés \\

