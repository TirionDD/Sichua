Présence des 9 dieux de l'archimonde:
Dieu de la Justice ERIL LB en centre ville sur l'îlot.
Dieu de la mort OULTOR LN au cimetière.
Dieu des enfers LE interdit cf maison noble associée
Dieu de l'agriculture NB dans les faubourgs
Dieu de la nature N à la source et un petit temple sur colline tranquilité.
Dieu de la pourriture et de la décomposition NE interdit mais égouts
Dieu des marchands CB sur la rive droite.
Dieu de la mer et des tempètes CN dans les ports. 
Dieu démon CE interdit mais existe un cercle secret

Diverses bibliothèques en particulier dans la biblio du dieu de la mort
sur l'histoire et à la biblio du temple de la justice

\subsection{Temple de la Justice}

Dedié au dieu Eril, le temple de la justice trone en plein centre de la ville sur l'île de la Source.
On remarque immédiatement ce bâtiment massif à la façade couverte de marbre qui donne immédiatement
sur l'avenue du duc Erland. Plusieurs statues ornent la façade un dizaine de mêtres au dessus du sol,
celles-ci représentent les grands prêtres les plus illustres de l'histoire du temple. Celui-ci fut 
d'apr¦es la légende érigé aux débuts de la cités il y a environ 1400 ans, mais tout bon érudit vous
dira que c'est fort improbable. La cité était alors bien trop pauvre et un édifice aussi luxueu date très 
certainement du premier grand age d'or de la ville, il y a environ 1000 ans.

PNJ: Grand Prètre, Maître Paladin (Klermor d'Herinard), Le Grand Justice

\subsection{Temple de la Mort}

Au bord du grand cimtière du sud. Construction assez recente. Plusieurs petits mausolés annexes.

\subsection{Temple d'Azazel}

Temple secret dissimulé dans une famille de sorciers

\subsection{Temple de l'Agriculture}

Au marché aux produits agricole, donne sur la place.

\subsection{Temple de la Source}

Dédié au dieu de la nature sur l'ile de la source dont il est l'origine du nom.

\subsection{Temple de la Pourriture}

Dieu interdit et essenciellement inconnu. Il en existe un temple

\subsection{Temple du Commerce}

Plusieurs temples en ville. Un grand temple annexe au palais du prévot des marchands.

\subsection{Temple de la Mer}

Grand temple sur un ilot isolé devant la ville. Nombreux mosolés dans les zones portuaires

\subsection{Temple du Chaos}

Enfoui sous l'habitation de nobles, c'est une seconde connection vers l'outre terre inconnue des
autorités.

