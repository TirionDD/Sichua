\section{Une Nouvelle Menace}

\subsection*{Troubles à Tayr}

Une perturbation à envahi les tréfonds et les équilibres basculent, la surface 
finit par être touché lorsqu'un vers pourpre gigantesque attaque la ville de 
Tayr. Les aventuriers sont appelé à la rescousse et doivent chasser la creature 
dans les environs. On leur apprend que la créature semble sensible aux vibrations
et si ils enquêtes en ville, ils peuvent en apprendre plus sur les capacités de 
la créature.

La creature peut être finalement découverte s'attaquant à un arras ou les pas
des chevaux l'ont attiré. Il y a une petite demeurre qui peut servir d'abris,
néanmoins la créature ne se soucie guère d'une telle structure qui n'est qu'un 
fétu de paille pour elle. 

Les autorités r'evèlent après l'attaque, que c'est loin d'être la première.
C'est n'eanmoins la première de cette envergure. Il y a de fortes inquiétudes
sur ce qui peut bien se passer dans les Tréfonds. Des aventuriers sont partis
ces derniers mois, peu sont revenu, si les aventuriers demandent qui est revenu
une tension s'installe. Un puissant groupe d'aventuriers de haute naissance 
est descendu il y a 6 semaines,
un paladin (herinard), un magicien (noble?), un ranger (elfe du nord, noble)
et un druide hobbit. Seul ces deux derniers sont revenus, malheureusement ils n'ont
plus vraiment toute leur tête. Ils ont parlé d'une ombre noir qui rentrait dans 
leur tête, rien de très clair, et ont finalement disparu dans les parties sauvages
des forêts du nord.

\subsection*{Les forets du nord}

Quête optionnelle pour poursuivre les deux survivants du précedent groupe. Les
trolls sont terrorisé et demande de l'aide aux aventuriers. Finalement les 
deux hommes sont rencontré, le combat est necessairement fatal, à moins de
leur lancé une restauration majeure. Dans ce cas ils expliquent, qu'il existe
une malediction sur les tréfonds qui entre dans les esprits. Surement un coup
des flagelleurs. Il y a une sorte d'ombre qui s'incerre dans les esprits, c'est 
difficile à decrire.

\subsection*{Descente dans les tréfonds}

Il existe un grand nombre de méthodes pour descendre dans les tréfonds. Néanmoins
aucune d'elle n'est particulièrement simple d'accès. La connaissance sur le commerce
dans la région ou des contacts avec les seigneurs d'Herinard peuvent mener le groupe 
à se renseigner sur la Dergade pour trouver un accès vers le comptoir commercial. Une
connaissance accrue des montagnes sauvages peut permettre de savoir que les géants de 
pierre ont des accès aux tréfonds. Les nains de Sichua ont un tel passage sous leur 
garde, mais seulement un nain ou quelqu'un connaissant en détails les légendes de la 
ville se souviendra de ce détail. Enfin, lorsque les aventuriers ont visité la guilde des
voleurs ou ensuite lorsque des hommes serpents y sont apparus, il est possible d'avoir 
identifié le passage qui s'y trouve.


\subsubsection*{Le comptoir commercial de Twelmdharik Erthanir}

Le guide initial du groupe mène les aventuriers a un avant poste commercial tenu par 
un elfe noir, Dezarik, un gnome des roches, Krim, et un nain gris Vartak. Dezarik est 
un commerçant dans toute son âme et toujours prêt à faire un bon deal. Malheureusement, 
la folie ne l'a pas épargné et il a pris la mauvaise habitude de manger ses visiteurs. 
Il servira ainsi leur guide aux aventurier dès le premier soir, à voir combien de temps 
il leur prendra pour s'apercevoir du problème. Les joueurs devront donc trouver une 
voie de sortie par leur propres moyens. Si les aventuriers ne se doute de rien et pense 
juste que leur guide s'est enfui, la thèse de Dezarik, il les dirigera vers les 
fomorians dans l'espoir de se débarrasser des aventuriers. En effet Dezarik n'est pas 
complètement fou et à conscience qu'il n'a aucune chance contre les aventuriers. 

\subsubsection{Descente avec les géants}

Les géants de pierre ne guident les aventuriers dans les tréfonds que si ceux-ci
acceptent d'aller voler une épée magique aux géants du feu. Si les joueurs acceptent
cette mission quelque peu suicidaire, il ont la chance d'attirer un ver pourpre lors 
de leur rencontre des géants du feu. Dans la confusion ils ont probablement une chance 
de voler l'arme et s'échapper sans trop de pertes.

\subsubsection{Descente avec les nains}

Les nains ne veulent rien avoir à faire avec les tréfonds et demande un don
substanciel (arme magique ou artefact nain) pour ouvrir le passage aux 
aventuriers.


\subsubsection{Passage par la guilde des voleurs}

Si les Yuan-ti n'ont pas investit la guilde, leur ville est juste en dessous et 
ils ne comptent pas laisser vivant des humains connaissant un passage secret 
vers leur cité!


\subsection*{Enquête dans les tréfonds}

La solution au problème se trouve dans l'antre des flagelleurs mentaux. Les aventuriers 
pourront y mettre fin à la malédiction néanmoins ils n'y trouveront pas l'éminence grise 
l'ayant causé. Ils y trouveront par contre leur vieil ennemi ayant survécu à leur rencontre 
à Sichua (le nécromant ou le démoniste) et apprendront que l'éminence grise est déjà en 
route pour Sichua pour dévorer la ville et finalement invoqué Azazel.

\subsubsection*{Introduction}

Cette partie est assez libre et permet aux aventuriers de rencontrer les differentes 
factions des tréfonds. Les elfes noirs, les nains gris, les gnomes des roches, les 
fomorians, les fungus, les oeils tyrans, les flagelleurs mentaux, les yuan ti, les 
géants de pierre, les géants de feu et les hommes poissons.

 Les membres du poste marchand ne savent pas trop ce qui se passe mais ils savent que 
tout le monde semble attaquer tout le monde et se retranche sur ses places fortes.

Lorsqu'ils decouvrent que l'origine des soucis est la ville des flageleur, on leur
fait bien comprendre qu'il serait suicidaire d'attaquer la ville sans alliés.

\subsubsection*{Twelmdharik Erthanir (elfes noirs)}

Les elfes noirs sont retranchés dans menzob. Et profitent des troubles pour attaquer 
sans merci tout ce qu'ils peuvent. Ils comptent bien tirer des bénéfices de cette 
histoire. En réalité les mères matronnes sont inquiètes car elles ne savent pas ce 
qu'il se passe et ont du mal à communiquer avec les diables et démons en cette période 
troublé. L'utilisation d'aventuriers extérieurs les intéresse car elles ne veulent 
pas que leur ignorance soit connu dans la cité. Si les aventuriers défont un groupe 
d'elfes noirs ils peuvent être recruté par la prêtresse a leur tête, si celle ci 
survit suffisamment. Elle promet une faveur de la matrone en échange d'information 
solide sur la source de la folie. Ils connaissent en particulier des routes menant 
à la surface. 

\subsubsection*{Forteresse Kalmur (nains gris)}

Les nains gris sont totalement refermé sur eux ils sont particulièrement paranoïaque 
et se concentre sur les aberrations qui apparaissent dans leurs mines. Ils ont vu un 
sombre mage passer il y a quelques mois et soupçonnent les elfes noirs. Si les 
aventuriers arrivent à créer un contact, ils les embauchent pour tuer des monstres 
dans leur mines. Si les aventuriers font plusieurs missions de ce type ils finissent 
par être envoyé contre un ver.

\subsubsection*{Dissumerik (gnomes ???)}

Les gnomes des roches ont vu leur ville ravagé par un ver possiblement identique à 
celui retrouvé a la surface. Une recherche avancée montre que ce ver est certainement 
encore vivant, il y en aurait donc plusieurs! Les gnomes ne savent pas d'où il sort 
ni ce que c'est et se cache dans leurs ruines en cherchant une nouvelle demeure. Ils 
sont très craintifs et secrets. Ils n'ont rien à offrir mais peuvent devenir des 
alliés de long terme. Les aventuriers peuvent tenter de combattre le ver ou aider 
les gnomes a se dégager un nouveau camp en nettoyant un nid d'ombre des roches.

\subsubsection*{Faezlum (fomorians)}

Les fomorians sont totalement berserk et ont perdu toute connection a la réalité. Ils 
attaque en hurlant à Azazel, gloire au grand ver et autres inepties. Tant que les 
aventuriers approchent le coeur de leur territoire ils en rencontre de plus en plus. 
Au coeur de leur territoire leur roi porte une couronne qui semble agir comme une 
antenne amplifiant la maladie sur son peuple. Une analyse du trésor royal montre de 
nombreuses couronnes de provenances diverses qui semblent toutes maudites. Le peuple 
fomorians semblent avoir le cerveau grillé par le nombre de fois où ils ont été 
contraints mentalement...

\subsubsection*{Chpof Stuff (fungus)}

Les fungus sont devenu rampant dans plusieurs régions et lâchent des spores a tout 
va. Si les aventuriers arrivent à rencontrer un chef, celui ci est euphorique et 
parle de conquête de la surface. À la différence des fomorians les fungus sont en 
règles générales plutôt censés. Il est possible de libérer leur chef du sort et 
ainsi de gagner des alliés. Ceux ci peuvent fournir une base arrière et des informations.

\subsubsection*{Panoptalon (oeil tyran)}

Les oeils tyrans n'apprécient pas vraiment la folie ambiante. C'est eux les fou 
d'habitude et ils ont dernièrement une lucidité fort désagréable ! Ils massacrent 
tout ce qui pourrait ressembler à une cause des troubles. Ils peuvent en cas de 
discussion très convaincante demander aux aventurier de libérer les fomorians du 
sort. Ils savent que c'est la faute des flagelleurs tout ce qui se passe et peuvent 
le reveler en donnant la mission. Sans dire que ce faisant ils recupereront le 
contrôle des fomorians en cas de succès. Les yeux tyrans n'ont bien-sûr aucune 
intention de payer les aventuriers et comptent les garder comme esclave en fin 
de mission. Ils leur promettent néanmoins l'objet de leur rêves quel qu'il soit.

\subsubsection*{Kysoparguya (flagelleurs mentaux)}

Les flagelleurs mentaux trouvé sont affaiblis car ils n'ont plus de cerveau 
central. Ils ont été attaqués et vaincu sur leur propre terrain mais ne savent 
pas comment. Ils ne peuvent plus partager de mémoires et sont déprimés. Si on 
leur parle des vers, ils expliquent leur origine et que c'est un grand sacrilège. 
Un être immonde doit être a l'origine de tout celà, un commentaire surprenant 
pour de tels créatures. Ils sont néanmoins affamé et tenteront de dévorer les 
aventuriers. Plus loin dans leur territoire on trouve leur cité en ruine et le 
necromancien y siégeant. Il annonce que son maître est déjà en route pour sichua 
pour la prochaine étape du plan. Lui attendait avec impatience les aventuriers. 
Le coeur d'un paladin d'Eril est le dernier ingrédient qui lui manque pour se 
transformer en liche, un état bien plus avancé que celui de son aïeul qu'il semble 
mépriser. Il dispose d'une petite armée de momies de flagelleurs. Dans la salle 
le cerveau des flagelleurs est torturé en permanence par une machine lui délivrant 
des décharges électriques a intervalles réguliers. Ce dispositif a été mis en place 
pour tenir les flagelleurs mentaux a distance. Ceux ci peuvent ressentir par 
empathie la douleur ressentie par le maître cerveau sur des centaines de kilomètres 
et ne peuvent y survivre bien longtemps. Néanmoins l'intensité du phénomène est tel 
qu'il affecte toutes Les créatures pensantes après un certain temps. Les aventuriers 
peuvent détruire la machine ou achever le cerveau pour mettre fin à cette affaire.

Ils découvrent alors des écritures dans un langage inconnu. Elle ne peuvent être 
lu que par magie, la langue etant inventée. C'est le journal du seigneur Oolomth,
un flagelleur qui explique au début du document qu'il cache des informations au
maître cerveau. Néanmoins il doit les écrire, car il les dissimule si profondément
dans son esprit, qu'il les oublie lui même. Il explique en detail son plan ultra
complexe (jet investigation DC 15 pour comprendre autre chose que le simple fait 
qu'il est à l'origine de tout les problèmes) qui consiste un plan pour détruire
les flagelleurs et absorber le pouvoir de tout Kysoparguya. Il compte ainsi prendre 
le controle du cerveau, dévorer tout ces compatriotes et opérer un "transformation".
Il compte aussi liberer les vers, puis
retourner le cervaux contre les flagelleurs pour empêcher ceux qui auraient échappé 
au massacre de revenir nettoyer le bordel. En parallèle, il implante des pions
à Sichua ou il se rendra une fois assez puissant. C'est le seul endroit ou suffisament 
de personnes sont rassemblées pour finaliser l'invocation d'Azazel. Une réussite 
DC 20 au jet d'investigation permet de comprendre que le plan n'est pas réellement
de libérer Azazel, le maître pense que Azazel sera détruit par les autres dieux si 
elle vient sur le plan materiel, en fait le maître veut  piquer la place d'Azazel en 
enfer au passage.

\subsubsection*{Szelrethin (yan-ti)}

Les yuan ti ont déserté leur campement et sont remontés vers la surface. Cette 
information peut être obtenu des quelques hommes serpents restants dans leur ancienne 
cité. Ils ont miraculeusement trouvé une voie alors qu' ils résident ici depuis des 
siècles ! Les aventuriers auront néanmoins du mal à suivre leur trace et devront 
affronter des gardes. Ils débarquent alors dans la guilde des voleurs. 

\subsubsection*{Forteresse parretem (géants de pierre)}

Les géants de pierre des tréfonds sont des sages de leur race. Ils sont reclu dans 
leur bibliothèque des tréfonds. Néanmoins la folie ne les a pas épargné et ils ont 
pris des accents millénaristes en se persuadant de l'arrivée prochaine de la fin des 
temps. Ils sont largement irrationnel et peuvent attaquer, aider ou ignorer les 
joueurs selon leur comportement. Leur bibliothèque contient des informations sur 
toutes les races des tréfonds et l'origine des monstruosités tel les vers pourpres. 

\subsubsection*{Forteresse Tarkin (géant de feu)}

Les géants du feu occupent de puissantes forges sous le mont Tarkin. Ils sont capables 
de travaux impressionnants mais sont à présent persuadés qu'on veut leur voler leurs 
oeuvres, en particulier les géants de pierre, les joueurs qui mentionnent les géants 
de pierre risque donc fort d'être attaqué! Ils sont à la recherche d'une épée qui leur 
a été volé alors qu'elle devait être transporté vers la surface, ils soupçonnent les 
géants de pierre. Peut être même en alliance avec des géants du froid ! Tout ça est du 
délire, si le groupe arrive à pister l'expédition ils retrouvent les restes des géants 
clairement attaqué par un ver au vu des nombreuses galeries récentes convergeant au 
lieux du combat. L'arme pourra être retrouvée dans l'estomac de la créature... aux 
aventuriers de voir si ils veulent poursuivre un ver qui a décimé un groupe de 5 géants. 
L'épée est inutilisable mais les géants pourront l'échanger contre une "braise magique" 
en fait une pierre ionique. Si les aventuriers ont l'idée d'aller revendre l'épée aux 
géants de pierre ceux ci offre un tome donnant +2 a une caractéristique en échange. Les 
géants de feu on des remorhaz comme animaux de compagnie et on en trouve pas mal de 
sauvage dans leur territoire. 

\subsubsection*{Blebalbugda (hommes poissons)}

Les hommes poissons sont habituellement des créatures pour le moins étranges en temps 
normal. Ils pondent leurs oeufs par une regurgitations particulièrement visuelle laissant 
derrière eux un amas d'oeufs gluans de 2-3cm de rayon. Les aventuriers pourraient aussi 
apercevoir certains hommes poissons uriner sur de tels tas, c'est raisonnablement qu'ils 
en deduiront le moyen de reproduction de ces créatures !  Toute interférence dans ce 
processus déclenchera l'hostilité des hommes poissons. En particulier un aventurier 
urinant sur des oeufs est un affront terrible équivalent à un viol ! Les hommes poissons 
sont devenu totalement mégalomane et planifient la domination des tréfonds. Ils embauchent 
les aventuriers pour s'attaquer aléatoirement a tous leurs voisins. Ils n'ont néanmoins 
pas réellement de quoi les payer après coup... Au contraire, les aventuriers pouraient voir 
une partie de leurs possessions disparaitre. Il est important que les joueurs negotient
un paiement en troupe pour attaquer les flagelleurs, sinon ils seront fort déçus...

\subsection{Retour à Sichua}

A écrire!
