\section{Le Cerveau se Révèle}

L'ombre de Sichua
 - Assassin orphelin sans aucun véritable nom.
 - Sans aucune famille, une seule chose compte lui-même
 - Il veut devenir immortel à tout prix
 - Il est aussi maître chanteur. Il utilise ça surtout contre ses anciens
   employeurs. Il lui arrive aussi de fouiller les documents de ses
   victimes à la recherche de dossiers incriminant pour leur proches.
 - C'est un gros joueur, il laisse pas mal d'argent Aux Dés d'Argent
 - Membre du conseil des cinq


Johan van Hjelmaster
 - Vampire (ancien noble)
 - De grande intelligence c'est maître du crime organisé
 - Il utilise principalement l'entrisme, le chantage et les assissinats
 - Sa faiblesse est sa femme qu'il n'a pas pu conserver en vie. Il conserve son corps religieusement.
 - Demeurre avec la dépouille de sa femme dans les sous-sols de la guilde des voleurs, il la controle 
   en sous-main
 - Son Phylactery se trouve dans la maison de sa famille
 - Il est le dirigeant de fait de sa famille
 - Il veut prendre le pouvoir, considère le duc un faible car il partage le controle de la ville

\subsection{Trouver le Maître Assassin}

\subsubsection*{Le Conseil des cinq}

Ce chapitre commence au petit matin après l'invocation par une réunion du 
conseil des cinq où les aventuriers sont invités.

A l'arrivée des joueurs le conseil semble quelque peu chaotique. Les marchands 
n'ont plus de leader clair et deux marchands (noms) sont présent au conseil. 
Ils annoncent d'emblés que leurs gardes sont dispachés pour défendre les entrepots
et ne sont pas disponible car ils ont subis trop de pertes. Le duc lui défend 
son château et le quartier nobles ou il y eu des pillages. Il accuse la plèbe des 
quartiers sud mais refuse de bouger ses troupes tant que les chambellans n'auront
pas remis leurs ouailles sous control. L'archimage ne participe que par une image 
magique dans une boule de cistale. Lui et ses mages sont retranchés dans la tour,
l'école est fermé jusqu'à nouvel ordre. La plus part des magiciens sont dans la tour
ou on rejoint leur famille (généralement noble ou bourgeoise) pour les protéger. 
Il informe au passage que les nains sont aussi retranchés. Le premier chambellan 
demande lui que l'on protège la population, mais il n'a rien à offrir et personne 
ne semble interessé pour le protéger. Il semble seulement ajouter à la confusion
en lançant régulièrement des remarques qui braquent tout le monde. Le grand prêtre 
est le seul personnage sensé. Il veut détruire les groupes de cultistes à l'aide 
de la liste obtenu par les joueurs. 

Les cinq finissent par converger et veulent que les joueurs trouvent et exécutent 
le chef des assassins qui est apparement le dernier membre du tryptique. Un éspérant
que le "maître du dessus" soit Azazel et non un quatrième larron!

Avant leur sortie du palais ducale, le fils du duc prend les joueurs à part et leur 
explique que la guilde des voleurs à une importance pour l'équilibre social. Il ne 
voudrait pas que les joueurs massacre tout l'état major de la guilde. Il leur suggère m
même de trouver des alliés en interne à la guilde. Si les joueurs acceptent, il leur 
indique les grandes lignes de la guilde : regroupe plusieurs type d'escrocs liés au 
sein d'un haut conseil. Le complexe refleterait cette organisation, malhereusement
il n'a jamais pu être localisé. Pourtant le duc y a consacré de nombreux espions.
Il existe néanmoins des soupçons autour des équipages de Harald Karak.

A la sortie du palais, le mage s'occupant du fonctionnement de la boule de crystal
indique aux joueurs que son maître est à la recherche d'un ancien artefact en 
possession des voleurs et qu'il souhaite les rencontrer pour en parler. Il recherche
une stèle couverte de runes naines. Il leur fournit un bag of holding pour rapporter 
l'objet. Il ne donne aucun détails sur l'usage de cet artefact, mais indique que 
celui ci est certainement dans un lieu central du complexe des voleurs.

\subsubsection{Trouver le Quartier Général de la Guilde des Voleurs}

Si les joueurs ont fait la quête pour tenter d'infiltrer la guilde des voleurs 
dans le chapitre précédent, ils peuvent finalement retrouver la halfline Karin.
Si les joueurs la prese suffisament, elle leur explique qu'elle les a escroqué
et tente de s'en sortir en vidant ses poches. Si les joueurs la pousse un peu
plus, elle révèle finalement le nom de son chef: Meribald Dalloris.

Les joueurs ont pu rencontrer par le passé une voleuse halfeline du nom de
Kirin. Celle ci ne veut/peut révéler l'emplacement de la guilde. Elle ne souhaite
pas trop y aller de toute manière, les rumeurs ne sont pas geniales. Une solution
est de lui dire d'y aller et de la suivre discrètement. Si elle sait qu'elle est 
accompagnée, elle ne sais plus ou c'est. Une autre solution c'est de lui demander 
ou est son chef, elle indique que c'es le patron des dés d'argents... Elle est
pas voleuse, c'est une escroc et ils ont été l'une de ses dernière victime. Seul
lui peut introduire de nouveaux membres dans la guilde. Elle indique que 
le serment prété est magique et ne peut etre rompu (DC 15 arcane indique que
cette histoire ne tient pas la route, DC 15 intuition indique qu'elle y croit dur 
comme fer).

\subsection*{La Demeurre van Hjelmaster}

