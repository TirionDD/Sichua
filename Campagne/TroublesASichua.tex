\section{Introduction}

Missions Part-branche-numbranche. Ajouter un guide pour naviguer entre les missions.

Joueurs:
\begin{itemize}
  \item {\bf Ghech} sorcier drakeïde (Mikael)
  \item {\bf Myrdim} barde demi-elfe (Kévin)
  \item {\bf Seyonne} moine demi-elfe (Julien)
  \item \sout{{\bf Mialee} ranger haute elfe (Cyrielle)}
  \item {\bf Thia} druide haute elfe (Cyrielle)
\end{itemize}

Ghech est lié avec un grand-ancien par un pacte de chaine. Son familier est un 
monodron. Le grand ancien est à la recherches de connaissances pour une raison
inconnue.

Myrdim est un peu kleptomane, ce qui lui attire des ennuis régulièrement.

Seyonne a appris son art auprès de démonistes qui l'avaient enlevés

Mialee accueille un groupe d'orphelin dans leur petite maison délabrée. Elle
est malheureusement décédée lors d'une rixe dans les faubourgs nord.

Thia, la s\oe{}r de Milee est une druide du temple de la nature en centre ville.

\section{Troubles à Sichua}

Le groupe de joueur a la liberté de démarrer ses recherches de différentes manières en fonction
du passé des personnages et de leur motivations. Ensuite, la manière dont leurs aventures se déroulent
peuvent changer du tout au tout leur parcours, si ils s'adressent au temple ceux-ci peuvent les rediriger 
vers le cultistes. Une rencontre aléatoire peut les faire se rediriger vers la recherche du nécromancien 
ou encore la recherche d'argent peut les amener à travailler avec la guilde des alchimistes. Dans tout
les cas, cette première partie de la campagne se termine lorsqe les joueurs défont l'un de ces trois
adversaires.

\subsection*{1-1-1 Les rats géants}

Depuis quelques temps des rats géants sont apparu en ville. 
Plusieurs chambellans de quartiers touché offrent une PO par carcasse.
Deux exemples de chambellans: Familan un marchand de pots en cuivres et 
casseroles, c'est un vieux monsieur avec de l'embonpoint et une grosse 
moustache blanche. Il peut orienter les PJs vers la quête.
Orond est lui chambellan au c\oe{}ur des quartiers attaqué par les rats,
c'est un maréchal férand plutôt jeune avec une barbe noire raisonnablement
taillé. Il est musclé et généralement transpirant dans sa boutique sur chauffé.
Devant chez lui s trouve un tas de cadavre de rats.

Nos jeunes aventuriers peuvent tenter une incursion dans les égouts pour trouver un grand nombre de rats.
N'importe quel habitant sait que les égouts sont un endroit dangereux pour une personne seule, mais pour 
un groupe, normalement il n'y a pas de problème.

Etape 1: Rassembler des informations et de l'équipement (torches et barre à mine sont utile)

Etape 2: Se rendre de nuit dans le quatier le plus infesté. Il y a quelques autres groupes qui cherchent
des rats. Ils sont formé 1d4+2 bandits MM343. Les PJs et un groupe découvrent 5 rats géants MM327, le 
meilleur jet de survie mène la traque. S'en suit une dispute sur qui doit récupérer les corps pour la 
prime. Tout est possible, jets contesté DC fixe en fonction de la force de l'argument. Si ils tuent les
autres aventuriers -> Mission échapper à la garde.

Étape 3: La victoire pousse le groupe à descendre dans les égouts. Descriptions égouts, odeur infame,
champignons géants dans tous les coins (DC15 nature pour les reconnaitre, jet de sauvegarde sagesse DC15 
pour ne pas s'evanouir si on y touche). On entend en permanence les couinements de rats (non géant) 
provenant de toute part, on les aperoient aussi à la limite de la lumière de la torche. Les aventuriers 
poursuivent quelques rats géants 
isolés (combats non joués). Ils en attrapent un toute les 10 minutes environ, après 1d4 rats capturés, 
jet de survie DC12 pour sentir qu'ils approchent un "nid" affrontent 5 rats géants MM327 apres deux 
tours 2 rats géants supplémentaires
arrivent par derrière. Certains rats semblent a moitié pourris comme si ils étaient dejà mort...

Étape 4: Sur le retour, l'aire semble lourd et le silence est complet dans l'égout. Certains des corps 
de rats aux ceintures se mettent à fremir et bouger... Sans barre à mine
jet DC 12 survie pour retrouver leur chemin. Sinon jet DC15 force pour ouvrir une porte. En cas d'échecs
ou de non fuite, deux squellettes apparissent MM272. A la fin du combat 1d6 corps de rats perdu semblent 
avoir fuient. Deuxieme tentative de sortie réussie sinon arrivent dans un couloir ou se trouvent une dizaine
de squelettes. Fuite oblegatoire.

A la sortie de l'égout l'équipe gagne un 300 xp et peut aller faire un rapport au chambellan. Il est
séptique et renvoie les PJ vers le temple de la justice sur l'affaire des morts vivants.

\subsection*{1-1-2 Un nécromant en ville}

Au temple on leur propose: Rencontre Temple démonistes (Mission 1-3-1) ou de ramener une preuve et une 
localisation précise. Les paladins préviennent néanmoins les PJs qu'un nécromant capable de lever une
armée de squelètes est certainement bien trop dangereux et ils doivent se contenter d'observer si il
y a réellement un nécromancien dans les égoûts.

Klermor d'Herinard -> Paladin au temple de la justice (Eril)

Enquete dans les sous sols 
 -> Quelques rats (bcp de rats morts)
 -> Jet de survie DC15, en cas d'echecs tombe dans une embuscade de champignons -> 5 violet fungus MM 138
 -> Trouve une "habitation" le necromant a fuit
 -> Les PJs tombent dans une embuscade -> 2 spectres mineurs MM 279 dont un poltergeist -> HP 18 domages 2d6 +3 pour toucher.
 -> Autel de pierre gravé de symboles étranges

Retour au temple
 -> C'est un signe d'un diable. Le nécromant a fait un pacte avec les enfers -> Azazel - Prince des enfers DC20 religion
 -> Envoie les PJs surveiller le cimetière du sud pendant la nuit en attendant
 -> Troisième nuit -> rencontre morts vivants -> rats nombreux si rien n'a été fait avec le tas de chez Orond -> 8 Giant zombie rats MM 327 -> fortitude du mort vivant MM316
 -> Rencontre le gardien du cimetière Paire en fait le nécromancien déguisé -> leur indique l'origine des troubles, un mosollé -> 2 ghouls MM148
 -> Découvrent un campement précaire

Rapport au temple
 -> Trop occupé avec un culte démoniaque qui prend de l'ampleur
 -> Conseil aux PJs de garder le cimetière contre le nécromancien
 -> Embuscade par des golems de bois qui remplaçaient des bancs
 -> Attaque par deux espions MM349 de nuit chez eux


\subsection*{1-1-3 En guerre contre la mort}

Surveillance de divers cimetières
 -> Finissent par attraper un groupe de pilleurs de tombe COMBAT
 -> Ils revendent les corps à la tour des mages, mais il est hors de question de témoigner
 -> Arrivent a découvrir les allés et venues suspicieuses du nécromant
 -> Lorsqu'il est attaqué il fuit avec porte dimensionnel et laisse les Pjs dans une embuscade COMBAT
 -> Renseignement sur son identité -> il est trop important pour être arreté sans preuves en béton
 -> Infiltration de la tour ou de la demeurre et COMBAT nécessaire


\subsection*{1-2-1 Sur la piste du fungus}

Un alchimiste du nom de Ferrangus emploie les aventuriers pour aller chercher des champignons dans les
égouts. La mission est identique à 1-1-1, cueillir les champignons nécessite un jet sleight of hand DC15
si les PJs savent qu'il est dangereux. Sinon échec automatique. 1d6 dégats psy à cause des spores.

Peut rediriger vers la recherche des mort-vivants 

\subsection*{1-2-2 A la chasse}

TROUVER CREATURES a chasser qui soit logiquement à la base de la création de golems. Doit comporter
un dilemme moral moyen de plus en plus fort

\subsection*{1-2-3 Une sombre affaire}

Infiltrer les combats clandestin de rue pour enlever des gladiateurs. Ceux ci servent à fabriquer
des golems de chair. Dilemme moral majeur. Les PJ comprennent qu'ils travaillent pour un criminel
dangereux!

\subsection*{1-3-1 Un culte turbulent}

Depuis quelques jours, les membres du temple du dieu de la justice (un groupe de clercs et paladins) 
suspectent la présence de démonistes. Malheureusement c'est hors de la ville dans un faubourg 
et donc sous la juridiction du duc qui n'est pas du tout interessé par ces histoires de religions.
Les paladins ne peuvent intervenir directement, il souhaitent que les PJs aillent jeter un oeil.

Étape 1: Le faubourg est pauvre et principalement habité par des ouvriers agricoles. Les aventuriers
s'y rendent, si ils questionnent à propos d'une secte, ils sont attaqués par 3 acolytes MM342. Si ils
trainent suffisamment dans le quartier avec un look assez pourri, ils se font alpaguer pour joindre le 
culte à minuit. Sinon ils doivent faire parler un cultiste.
-> Derrière l'auberge du fer à cheval se reunissent les membres du culte

Étape 2: Surprendre les cultistes. Les PJs peuvent assister aux culte en tant que croyant ou en éspions.
En tant que croyant, on leur demande de boire une coupe de sang "humain" (en fait du sang de cochon).
Les cultistes ne semblent pas très dangereux. Par contre, si ils sont pris à éspionner ils se font 
poursuivre par un groupe de paysans armés de gourdins MM345.

Étape 3: Les PJs font leur rapport au culte. Ils se font jeter si ils ont tués des civils, les paladins ne 
les étrippent pas simplement pour ne pas être impliqué. Sinon, les paladins proposent de payer plus pour 
que les PJs continuent à enquêter sur le leader du culte.

Étape 4: Les PJs peuvent essayer de coincer Bardan le gourou du culte (Fanatique MM345). Il ne parlera
sous aucun pretexte. Les PJs s'aperoivent que c'est un vrai clerc d'un dieu démoniaque. Il porte sur lui 
une amulette (CA+1). Le porteur commence à faire des cauchemars.

\subsection*{1-3-2 Un véritable culte démoniaque}

Le porteur de l'amulette se fait aborder dans la rue par un cultiste. Si il sont consulté les paladins
proposent aux joueurs de s'infiltrer dans le culte au plus haut niveau. Le groupe se fait enroler. 
Les premières missions sont raisonnable et consiste a trouver un lieu de culte (nettoyage de quelques
fungis et morts vivants dans les egouts). Assez vite les cérémonies deviennent macabres, des hommes se 
scarifient au début, puis se sont des étrangers qui sont torturés. Un jour un jeune paladin a le coeur 
arraché, l'un des pj doit aider a tenir la victime. On demande au PJ d'enlever d'autres paladins ou
clercs et on leur propose de se scarifier. La seule question est quand arreteront-ils les frais. Vers le 
milieu de l'initiation, ils identifient le véritable leader du culte.

\subsection*{1-3-3 Détruire le culte}

Le culte commence a invoquer des démons la déstruction (ou l'alliance totale?) devient nécessaire.

Les Paladins peuvent aider, la principale difficulté est de coincé les chefs du culte et en particulier
le sorcier. Il faut aussi affronter des invocations d'autant plus puissante que les PJs ont tardé à changer
de piste. 


%TODO Ajouter des indices qu'il y a un maitre derrière tout ça


