
À faire: un guide pour naviguer entre les missions.
Discution sur l'équilibrage en terme d'xp.
Ajouter des indices qu'il y a un maitre derrière tout ça.
Ajouter des objets magiques correctement décrits.

\section{Introduction}

Dans ce premiers chapitre il existe trois principaux adversaires pour les joueurs: le nécromant, 
l'alchimiste et le démoniste. Le chapitre ce termine lorsque l'un de ces trois PNJ est découvert et 
défait. Le groupe de joueurs a donc la liberté de poursuivre ses recherches de différentes manières 
en fonction de leur passé et de leurs motivations. Il ont donc dès le départ la liberté de se diriger
vers les PNJ de leur choix qui les guideront alors vers l'une des branches de l'aventure en fonction 
de leurs contacts et préocupations. 

La manière dont les aventures se déroulent peut changer du tout au tout d'une partie à l'autre et il
est tout à fait possible de passer d'une séquence à une autre en cours de route. Par
exemple, si les joueurs travaillent pour l'alchimiste et s'adressent au temple pour des soins ceux-ci 
peuvent les rediriger sur la recherche des cultistes au lieu de finir la séquence de l'alchimiste. Dans 
tous les cas, cette première partie de la campagne se termine lorsque les joueurs défont l'un de ces trois
adversaires.

En terme de niveau, il est attendu que les joueurs finissent ce chapitre au niveau 5.


\section{Le Nécromant}
\subsection{Les rats géants}
\label{ss:RatsGeants}

Depuis quelques temps des rats géants et particulièrement corriaces sont apparus en ville. Le chambellans 
Orond des quartiers sud qui sont les plus touchés offre une pièce d'or par carcasse. Ceci représente une
somme non négligeable pour des personnages de niveau 1. Orond est un maréchal férand plutôt jeune avec une 
barbe noire raisonnablement taillée. Il est musclé et généralement transpirant dans sa boutique surchauffée.
Devant chez lui se trouve un tas de cadavre de rats et il explique à toute personne interessé qu'il paye
une pièce d'or par corps. Il brule les cadavres dans sa forge en fin de journée, ce qui envahi le quartier
avec un horrible odeur de chaires brulées.

Les jeunes aventuriers peuvent tenter une incursion dans les égouts pour trouver un grand nombre de rats.
N'importe quel habitant sait que les égouts sont un endroit dangereux pour une personne seule, mais pour 
un groupe, normalement il n'y a pas de problème. Néanmoins, la présence des rats peut refroidir les ardeurs 
des aventuriers, qui se contenteront alors de commencer par chasser quelques rats en surfaces.

Lorsqu'ils se rendent de nuit dans le quartier le plus infesté, les aventuriers remarquent qu'ils ne
sont pas les seul à la poursuite de cette prime miraculeurse. Il y a ainsi quelques autres groupes qui 
cherchent des rats, ceux ci sont composés de 1d4+2 jeunes hommes armés de piques, fourches et gourdins
(pour ces chasseurs, utiliser les caractéristiques du guerrier tribal page~\pageref{GuerrierTribal}). 
Les joueurs et un groupe découvrent simultanement cinq rats géants (page~\pageref{RatGeant}), le 
meilleur jet de survie entre les chasseurs et les joueurs mène la traque des rats et réussit à les coincer 
dans une ruelle. S'en suit une dispute sur qui doit récupérer les corps pour la prime. Tout est possible, 
jets DC 17 en intimidation pour faire abandonner les chasseurs de rat. Sinon jet DC 15 persuasion permet 
aux aventuriers de prendre trois corps si ils ont mené la chasse deux dans le cas contraire. En fonction 
de la force des arguments et le déroullement du combat, le meneur est invité à adapter le DC. Si ils tuent 
les autres aventuriers, le chambellan ne distribuera pas de prime du fait des meurtre de la nuit et se 
méfiera des joueurs pour le reste de la campagne.

La victoire facile pousse généralement le groupe à descendre dans les égouts pour trouver plus de rats. 
Dans le cas contraire répéter l'étape précédente mais avec deux groupes de chasseurs rendant la chasse peut 
rentable. Décrire aux joueurs les égoûts lorsqu'ils y pénètrent: ce qu'ils remarque en premier et avant 
tout c'est l'odeur infame qui règne dans les tunnels, ceux-ci semblent creusé dans la roche et altèrenent
entre caves natuelles assez vastes et étroits couloirs. Des champignons géants poussent dans les caves
naturelle, un jet DC15 en nature permet de les reconnaitre, ils sont empoisonnés et un jet de sauvegarde 
de sagesse DC15 est nécessaire pour ne pas s'evanouir lorsqu'on y touche. Par ailleurs, on entend en 
permanence les couinements de rats (non géant) provenant de toute part, on les aperçoient aussi à la 
limite de la lumière des torches. Les aventuriers pataugeant dans la fange jusqu'au milieu des mollets
poursuivent 1d4 rats géants isolés (combats non joués) lors de leur première heure dans les égoûts. Avec 
un jet de survie DC12 les joueurs trouvent la piste d'un "nid" et affrontent cinq rats géants 
(page~\pageref{RatGeant}) dans un couloir après deux tours de combat deux rats géants supplémentaires
arrivent par derrière. Certains rats semblent a moitié pourris comme si ils étaient déjà mort...

Chaque aventurier peut porter un nombre de rats géants égal à sa force divisé par 5 arrondi à l'inférieur.
Néanmoins à la fin du combat, l'ambiance à changé, l'aire semble lourd et le silence est complet dans 
l'égout. Alors que les aventuriers se déplacent vers une sortie, certains des corps 
de rats aux ceintures se mettent à fremir et bouger. Si les aventuriers ne tentent pas de s'échapper 
ils rencontrent un, puis deux puis trois squelettes et ainsi de suite de manière incrémentale 
(page~\pageref{Squelette}). Le nombre de squelette augmente d'une unité chaque fois que les joueurs 
décident de ne pas fuir ou échoue dans leur tentative de fuite. Pour chaque tentative de fuite, les 
joueurs doivent fair un jet de survie, sur un 20 ils trouvent une sortie non bloquée, sur un 15 ils trouvent
une sortie bloquée. Si ils sont équippés d'une barre à mine il suffit d'un jet DC15 de force pour 
ouvrir la porte, sans barre à mine la difficulté est de 20. La difficulté pour un jet de crochetage est de 
18 car la serrure est rouillée. En cas d'échecs à trois jets sur une porte apparissent des squelettes
(page~\pageref{Squelette}). A la fin du combat 1d6 corps de rats semblent avoir rongé leurs liens et fuit. 

A la sortie de l'égout, le petit matin se lève et les os de squelettes se transforment en poussière si 
l'équipe en a ramassé. Le groupe peut aller faire un rapport au chambellan et récupérer une prime. 
Néanmoins Orond est très séptique à l'histoire des squelettes, il renvoie quand même les joueurs vers 
le temple de la mort pour discuter de l'affaire. Les joueurs passent niveau 2 si ce n'est pas déjà fait.

\subsection{Un nécromant en ville}

Au temple de la mort les héros sont aussi reçu avec scéptisicisme, mais les prètres sont tout de même
intressé. On leur propose en premier lieu d'aller se renseigner au temple de la justice, où il rencontres
Klermor d'Herinard qui leur mentionne les soucis de démonistes (page~\pageref{Demoniste}) et tente de 
recruter les joueurs, mais il ne dispose d'aucune signialisation de nécromancien. Si les joueurs décident
de poursuivre dans la recherche du nécromancien ils devront ramener une preuve, voir mieux, une 
localisation précise aux prètres de la mort. Ils préviennent néanmoins les aventuriers qu'un nécromant 
capable de lever une armée de squelètes est certainement bien trop dangereux et ils doivent donc bien
se contenter d'observer et rapporter l'information si il y a réellement un nécromancien dans les égoûts.

Les joueurs redescendent dan les égoûts, cette fois les rats géants sont plus rares et aucun ne semble 
mort-vivant. Les joueurs en tuent 1d4+1 lors de leur exploration sans besoin de jouer les combats. Lors de
leurs trajets ils rencontrent par contre pas mal de rats à moitié décomposés, ceux-ci semblent néanmoins
en trop mauvais étât pour toucher la prime. Un jet de survie DC 15 permet de retrouver les lieux d'où 
semblait provenir les squelette rapidement, en cas d'echecs les aventuriers se perdent un peu et tombent 
dans une embuscade de champignons animés (-> 5 violet fungus MM 138 \pageref{FungusViolet}).

Les joueurs trouvent finalement un coin de caverne naturel amenagé qui pourrait certainement être la tanière
du nécromant. On y trouve un tas d'herbes qui servent d'ingrédients dans des préparations magiques (jet 
d'arcane DC 12 pour le savoir) néanmoins les composants de valeur semblent être manquant (jet d'investigation
DC 15). Dans cette tanière on trouve aussi une couche dans un coin et un autel. Mais alors que les joueurs
s'en approche, sortent des murs deux spectres (\pageref{Spectre} dont un poltergeist -> HP 18 
domages 2d6 +3 pour toucher). À la fin du combat, les joueurs peuvent noter les runes inscritent sur l'autel.
Un jet de religion DC20 est nécessaire pour les reconaître, ce sont des signes utilisé pour pactiser avec
les demons et diables et une rune est celle de Azazel, prince des enfer. En cas d'échec au jet de religion, 
les joueurs peuvent obtenir ces informations des prètres du temple de la mort ou de la justice.

Après consultation du clergé du temple de la mort, ceux-ci vont inspecter le camp du nécromancien et 
confirment que c'est bien la signature d'un nécromancien. Le pacte que celui-ci semble avoir conclu avec les 
enfers n'en est que plus inquiétant. Des différents cimetières, le temple est surtout inquiet de la 
surveillance du grand cimetière du sud et demande aux joueurs de les aider à le surveiller contre salaire 
(10 PO par nuit et un supplément de 50 si ils trouvent quelque chose). À leur arrivé au cimetière sud,
les joueurs rencontrent un gardien nommé Paire, celui-ci est un homme du nécromancien et rapporte à son 
maître toutes les allées et venues des joueurs.

%TODO Paire: cheveux graisseux etc. (comme Rogue de HarryP) 

L'une des premières nuit, les joueurs sont attaqués par un groupe de 8 rats géant morts-vivants 
(page~\pageref{}). Au début du combat, seul un 
joueur particulièrement observateur (perception passive DD 15) remarquera la nature des rats. Sinon, c'est 
lorsqu'un des rats se relevera grâce à la fortitude du mort-vivant que les joueurs comprendront qu'ils
n'ont pas affaire à des créatures normales. Si les aventuriers tentent de ramener les corps de rats à Orond,
celui-ci refuse de payer, le tas de rats devant son échoppe a été volé la veille! Heureusement la menace
des rats géants semblent avoir largement diminué dans le quartier, l'offre de une PO par corps n'est donc
plus valable.

Trois jours plus tard, Paire indique qu'il a vu quelqu'un rodé pret d'un mosolé et demande aux joueurs 
d'aller investiguer. Rien n'est observable de l'extérieur à part que la porte semble ouverte, ce qui est
déjà anormal en tant que tel. Lorsque un joueur inspecte l'une des trois portes de la pièce principale, 
celles-ci s'ouvrent brusquement et trois goules (page~\pageref{}) attaquent les joueurs. Une fois débarrassé
des monstres, les joueurs trouvent un petit campement précaire dans une des pièces avec 28 PO mais aucuns
objets de valeurs. Un jet d'investigation permet d'en savoir plus, un DC 15 permet aux joueurs de comprendre
que le camp n'est plus habité, un DC 20 permet de comprendre que ce camp n'a jamais été habité et que ce 
n'est qu'un piège. Si les joueurs confrontent Paire à ce sujet celui-ci tente de fuir ou de se suicider,
un interogatoire serré permet de lui faire avouer son réel rôle. Il ne connait pas le nom du nécromancien
mais sait que c'est un professeur à l'école de magie avec des contacts dans les bas-fonds.  Si les joueurs 
ne le sont pas déjà, ils passent niveau 4 lorsqu'ils identifient le nécromancien et passent à la section
suivante.

%TODO Carte mosolé?  

Si les joueurs se rendent au temple de la justice, ceux-ci leurs indiquent travailler actuellement contre 
un culte démoniaque,
si l'aventures a mené les joueurs dans cette direction par le passé, ils peuvent être redirigés à nouveau 
sur cette voie. Sinon, le temple indique au joueurs de garder un \oe{}il sur le cimetierre, car le 
nécromancien devra bien finir par y revenir. Les joueurs ont ainsi le choix de poursuivre une menace 
plutôt qu'une autre. Dans les jours qui suivent la reprise de l'enquête, les joueurs sont attaqués
chez eux ou dans la rue par deux éspions (page~\pageref{}). Si l'un d'eux est capturé, un interogatoire
serré lui fera avouer qu'il travail pour XXX un caïd local à la réputation sulfureuse. Si les joueurs
négotient bien avec leur prisonnier ou en enquêtant dans les bas-fonds, ils apprennent que le quartier 
géneral de XXX se trouve dans un bordel, le XXX, et que XXX est un demi-ogre à la tête d'une bande de
malfrats assez importante. Il existe plusieurs méthode pour obtenir des informations de ce groupe, les
hommes de XXX servent effet de pourvoyeurs de corps au nécromancien depuis plusieurs semaines et la 
capture d'un
seul d'entre eux peut suffire à obtenir ces informations. Il est bien sûr aussi possible de prendre 
d'assaut la batisse et d'intéroger XX après l'avoir vaincu. Ceci irrite néanmoins largement les 
autorités ducales, qui n'apprécient guère le groupe de justicier travaillant pour les temples. Si les
joueurs posent des questions au temple ou dans la rue, ils apprennent que les gardes ducaux étaient 
certainement corrompus.

%TODO Description de l'auberge et des hommes de XXX a mettre dans les lieux 


\subsection{L'école de magie}

Lorsque les joueurs se rendent à l'école de magie, si ils ne sont pas encore au courrant que le nécromancien
qu'ils cherchent est un professeur ils peuvent être mal orientés. En effet, si on questionne les gardes de
la tour de magie, le chef de la sécurité renverra les aventuriers vers XXX, le dernier élève renvoyé 
pour pratique illégale de la nécromancie. XXX peut être retrouvé en ville dans une boutique d'alchimiste,
après avoir interogé son frêre qui est à la tête d'une importante
entreprise de commerce de textils en gros. Si on intéroge son frêre, celui ci indique que XXX a fait des 
experiences malsaines sur le corps de leur s\oe{}ur et il refuse de discuter plus en détails. Néanmoins 
aucune activitée suspicieuse ne semble entourer l'activitée de XXX et 
lorsque il est interogé sur son renvoie, il racconte son histoire. En tant que benjamin de sa fratrie, ses
parents ont tenté de lui trouver une autre profession que le commerce de textils qui devait revenir à son 
ainé XXX. C'est ainsi qu'il est entré à l'académie de magie, non par passion mais par devoir. Alors qu'il
débutait sa cinquième année d'étude, XXX sa s\oe{}ur est décédée de manière brutale d'une maladie 
fulgurante. C'est alors que Soren van Hjelmaster, son meilleur amis, lui à fait part d'un plan pour
tenter de la faire revivre. Il disais qu'il existait des sorts de résuréction arcanique gardé secret. Les
deux étudiants s'introduirent discrètement dans la bibliothèque privée du professeur de nécromancie et 
subtilisèrent un ancien livre de rituels nécromants. Mais deux jeunes étudiants n'étaient pas capable
de pratiquer cette magie puissante et le rituel échoua transformant sa s\oe{}ur en une horrible goule. 
Les parents de XXX le chassère ainsi que l'école de magie, par amitié XXX ne dénonça jamais son ami Soren,
qui est à présent un puissant professeur de l'école du phare. Néanmoins, depuis toutes ces années XXX
a commencé à douter de l'histoire de Soren et se demande de plus en plus si il n'y avait pas plus dans 
cette histoire...

% Statut de mon groupe actuel 
Etape 2: Un adversaire identifié
 -> Arrivent a découvrir les allés et venues suspicieuses du nécromant
 -> Lorsqu'il est attaqué il fuit avec porte dimensionnel et laisse les Pjs dans une embuscade COMBAT -> Flameskull MM134
 -> Renseignement sur son identité -> il est trop important pour être arreté sans preuves en béton
 -> L'équipe passe niveau 4

Etape 3: Aller chercher l'ennemi dans sa tanière
 -> Infiltration de la tour ou de la demeurre et COMBAT nécessaire -> Scouts MM349
 -> Surveillance de la tour
 -> Golems de bois dans son salon
 -> Le dernier combat du nécromant -> comme Herinos p18

%TODO Fiche PNJ
Soren van Hjelmaster
 - Nécromancien (famille noble)
 - Professeur de nécromancie à l'école de la tour phare
 - Il lui arrive de se déguiser pour passer innaperçu
 - Il se rend dans différends lieux de la ville pour faire ses expériences


\section{L'Alchimiste}

\subsection{Sur la piste du fungus}

Un alchimiste du nom de Ferrangus emploie les aventuriers pour aller chercher des champignons dans les
égouts. La mission est identique à la séquence "Les rats géants" page \pageref{ss:RatsGeants}, cueillir 
les champignons nécessite un jet sleight of hand DC15
si les PJs savent qu'il est dangereux. Sinon échec automatique. 1d6 dégats psy à cause des spores.

Peut rediriger vers la recherche des mort-vivants 

\subsection{À la chasse}

TROUVER CREATURES a chasser qui soit logiquement à la base de la création de golems. Doit comporter
un dilemme moral moyen de plus en plus fort

\subsection{Une sombre affaire}

Infiltrer les combats clandestin de rue pour enlever des gladiateurs. Ceux ci servent à fabriquer
des golems de chair. Dilemme moral majeur. Les PJ comprennent qu'ils travaillent pour un criminel
dangereux!
 -> L'équipe passe niveau 3

Combat contre l'alchimiste impossible en présence de ses machines de guerre (golems).

\section{Le Démoniste}
\label{Demoniste}

\subsection{Un culte turbulent}

Depuis quelques jours, les membres du temple du dieu de la justice (un groupe de clercs et paladins) 
suspectent la présence de démonistes. Malheureusement c'est hors de la ville dans un faubourg 
et donc sous la juridiction du duc qui n'est pas du tout interessé par ces histoires de religions.
Les paladins ne peuvent intervenir directement, il souhaitent que les PJs aillent jeter un oeil.

Étape 1: Le faubourg est pauvre et principalement habité par des ouvriers agricoles. Les aventuriers
s'y rendent, si ils questionnent à propos d'une secte, ils sont attaqués par 3 acolytes MM342. Si ils
trainent suffisamment dans le quartier avec un look assez pourri, ils se font alpaguer pour joindre le 
culte à minuit. Sinon ils doivent faire parler un cultiste.
-> Derrière l'auberge du fer à cheval se reunissent les membres du culte

Étape 2: Surprendre les cultistes. Les PJs peuvent assister aux culte en tant que croyant ou en éspions.
En tant que croyant, on leur demande de boire une coupe de sang "humain" (en fait du sang de cochon).
Les cultistes ne semblent pas très dangereux. Par contre, si ils sont pris à éspionner ils se font 
poursuivre par un groupe de paysans armés de gourdins MM345.

Étape 3: Les PJs font leur rapport au culte. Ils se font jeter si ils ont tués des civils, les paladins ne 
les étrippent pas simplement pour ne pas être impliqué. Sinon, les paladins proposent de payer plus pour 
que les PJs continuent à enquêter sur le leader du culte.

Étape 4: Les PJs peuvent essayer de coincer Bardan le gourou du culte (Fanatique MM345). Il ne parlera
sous aucun pretexte. Les PJs s'aperoivent que c'est un vrai clerc d'un dieu démoniaque. Il porte sur lui 
une amulette (CA+1). Le porteur commence à faire des cauchemars.

\subsection{Un véritable culte démoniaque}

Le porteur de l'amulette se fait aborder dans la rue par un cultiste. Si il sont consulté les paladins
proposent aux joueurs de s'infiltrer dans le culte au plus haut niveau. Le groupe se fait enroler. 
Les premières missions sont raisonnable et consiste a trouver un lieu de culte (nettoyage de quelques
fungis et morts vivants dans les egouts). Assez vite les cérémonies deviennent macabres, des hommes se 
scarifient au début, puis se sont des étrangers qui sont torturés. Un jour un jeune paladin a le coeur 
arraché, l'un des pj doit aider a tenir la victime. On demande au PJ d'enlever d'autres paladins ou
clercs et on leur propose de se scarifier. La seule question est quand arreteront-ils les frais. Vers le 
milieu de l'initiation, ils identifient le véritable leader du culte.

\subsection{Détruire le culte}

Le culte commence a invoquer des démons la déstruction (ou l'alliance totale?) devient nécessaire.

 -> L'équipe passe niveau 3
Les Paladins peuvent aider, la principale difficulté est de coincé les chefs du culte et en particulier
le sorcier. Il faut aussi affronter des invocations d'autant plus puissante que les PJs ont tardé à changer
de camps. 



\section{Suivi groupe}

Joueurs:
\begin{itemize}
  \item {\bf Ghech} sorcier drakeïde (Mikael)
  \item {\bf Myrdim} barde demi-elfe (Kévin)
  \item {\bf Seyonne} moine demi-elfe (Julien)
  \item \sout{{\bf Mialee} ranger haute elfe (Cyrielle)}
  \item {\bf Thia} druide haute elfe (Cyrielle)
\end{itemize}

Ghech est lié avec un grand-ancien par un pacte de chaine. Son familier est un 
monodron. Le grand ancien est à la recherches de connaissances pour une raison
inconnue.

Myrdim est un peu kleptomane, ce qui lui attire des ennuis régulièrement.

Seyonne a appris son art auprès de démonistes qui l'avaient enlevés

Mialee accueille un groupe d'orphelin dans leur petite maison délabrée. Elle
est malheureusement décédée lors d'une rixe dans les faubourgs nord.

Thia, la s\oe{}r de Milee est une druide du temple de la nature en centre ville.

Le groupe a fait la mission "Les rats géants", "Un nécromant en ville" et "Un culte turbulent"

Suivi niveau: faire un tableau

Familan un marchand de pots en cuivres et 
casseroles, c'est un vieux monsieur avec de l'embonpoint et une grosse 
moustache blanche. Il peut orienter les PJs vers la quête.
Orond est lui chambellan au c\oe{}ur des quartiers attaqué par les rats,

Klermor d'Herinard -> Paladin au temple de la justice (Eril)


