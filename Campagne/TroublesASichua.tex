
À faire: un guide pour naviguer entre les missions.
Discution sur l'équilibrage en terme d'xp.
Ajouter des indices qu'il y a un maitre derrière tout ça.
Ajouter des objets magiques correctement décrits.

\section{Introduction}

Dans ce premiers chapitre il existe trois principaux adversaires pour les joueurs: le nécromant, 
l'alchimiste et le démoniste. Le chapitre ce termine lorsque l'un de ces trois PNJ est découvert et 
défait. Le groupe de joueurs a donc la liberté de poursuivre ses recherches de différentes manières 
en fonction de leur passé et de leurs motivations. Il ont donc dès le départ la liberté de se diriger
vers les PNJ de leur choix qui les guideront alors vers l'une des branches de l'aventure en fonction 
de leurs contacts et préocupations. 

La manière dont les aventures se déroulent peut changer du tout au tout d'une partie à l'autre et il
est tout à fait possible de passer d'une séquence à une autre en cours de route. Par
exemple, si les joueurs travaillent pour l'alchimiste et s'adressent au temple pour des soins ceux-ci 
peuvent les rediriger sur la recherche des cultistes au lieu de finir la séquence de l'alchimiste. Dans 
tous les cas, cette première partie de la campagne se termine lorsque les joueurs défont l'un de ces trois
adversaires.

En terme de niveau, il est attendu que les joueurs finissent ce chapitre au niveau 5.


\section{Le Nécromant}
\subsection{Les rats géants (Niv. 1)}
\label{ss:RatsGeants}

Depuis quelques temps des rats géants et particulièrement corriaces sont apparus en ville. Le chambellans 
Orond des quartiers sud qui sont les plus touchés offre une pièce d'or par carcasse. Ceci représente une
somme non négligeable pour des personnages de niveau 1. Orond est un maréchal férand plutôt jeune avec une 
barbe noire raisonnablement taillée. Il est musclé et généralement transpirant dans sa boutique surchauffée.
Devant chez lui se trouve un tas de cadavre de rats et il explique à toute personne interessé qu'il paye
une pièce d'or par corps. Il brule les cadavres dans sa forge en fin de journée, ce qui envahi le quartier
avec un horrible odeur de chaires brulées.

Les jeunes aventuriers peuvent tenter une incursion dans les égouts pour trouver un grand nombre de rats.
N'importe quel habitant sait que les égouts sont un endroit dangereux pour une personne seule, mais pour 
un groupe, normalement il n'y a pas de problème. Néanmoins la présence des rats peut refroidir les ardeurs 
des aventuriers, qui se contenteront alors de commencer par chasser quelques rats en surfaces.

Lorsqu'ils se rendent de nuit dans le quartier le plus infesté, les aventuriers remarquent qu'ils ne
sont pas les seul à la poursuite de cette prime miraculeurse. Il y a ainsi quelques autres groupes qui 
cherchent des rats, ceux ci sont composés de 1d4+2 jeunes hommes armés de piques, fourches et gourdins
(pour ces chasseurs, utiliser les caractéristiques du guerrier tribal page~\pageref{GuerrierTribal}). 
Les joueurs et un groupe découvrent simultanement cinq rats géants (page~\pageref{RatGeant}), le 
meilleur jet de survie entre les chasseurs et les joueurs mène la traque des rats et réussit à les coincer 
dans une ruelle. S'en suit une dispute sur qui doit récupérer les corps pour la prime. Tout est possible, 
jets DC 17 en intimidation pour faire abandonner les chasseurs de rat. Sinon jet DC 15 persuasion permet 
aux aventuriers de prendre trois corps si ils ont mené la chasse deux dans le cas contraire. En fonction 
de la force des arguments et le déroullement du combat, le meneur est invité à adapter le DC. Si ils tuent 
les autres aventuriers, le chambellan ne distribuera pas de prime du fait des meurtre de la nuit et se 
méfiera des joueurs pour le reste de la campagne.

La victoire facile pousse généralement le groupe à descendre dans les égouts pour trouver plus de rats. 
Dans le cas contraire répéter l'étape précédente mais avec deux groupes de chasseurs rendant la chasse peut 
rentable. Décrire aux joueurs les égoûts lorsqu'ils y pénètrent: ce qu'ils remarque en premier et avant 
tout c'est l'odeur infame qui règne dans les tunnels, ceux-ci semblent creusé dans la roche et altèrenent
entre caves natuelles assez vastes et étroits couloirs. Des champignons géants poussent dans les caves
naturelle, un jet DC15 en nature permet de les reconnaitre, ils sont empoisonnés et un jet de sauvegarde 
de sagesse DC15 est nécessaire pour ne pas s'evanouir lorsqu'on y touche. Par ailleurs, on entend en 
permanence les couinements de rats (non géant) provenant de toute part, on les aperçoient aussi à la 
limite de la lumière des torches. Les aventuriers pataugeant dans la fange jusqu'au milieu des mollets
poursuivent 1d4 rats géants isolés (combats non joués) lors de leur première heure dans les égoûts. Avec 
un jet de survie DC12 les joueurs trouvent la piste d'un "nid" et affrontent cinq rats géants 
(page~\pageref{RatGeant}) dans un couloir après deux tours de combat deux rats géants supplémentaires
arrivent par derrière. Certains rats semblent a moitié pourris comme si ils étaient déjà mort...

Chaque aventurier peut porter un nombre de rats géants égal à sa force divisé par 5 arrondi à l'inférieur.
Néanmoins à la fin du combat, l'ambiance à changé, l'aire semble lourd et le silence est complet dans 
l'égout. Alors que les aventuriers se déplacent vers une sortie, certains des corps 
de rats aux ceintures se mettent à fremir et bouger... Sans barre à mine
jet DC 12 survie pour retrouver leur chemin. Sinon jet DC15 force pour ouvrir une porte. En cas d'échecs
ou de non fuite, deux squellettes apparissent MM272. A la fin du combat 1d6 corps de rats perdu semblent 
avoir fuient. Deuxieme tentative de sortie réussie sinon arrivent dans un couloir ou se trouvent une dizaine
de squelettes. Fuite oblegatoire.

A la sortie de l'égout l'équipe passe niveau 2 et peut aller faire un rapport au chambellan. Il est
séptique et renvoie les PJ vers le temple de la justice sur l'affaire des morts vivants.

\subsection{Un nécromant en ville (Niv. 2)}

Au temple on leur propose: Rencontre Temple démonistes (Mission 1-3-1) ou de ramener une preuve et une 
localisation précise. Les paladins préviennent néanmoins les PJs qu'un nécromant capable de lever une
armée de squelètes est certainement bien trop dangereux et ils doivent se contenter d'observer si il
y a réellement un nécromancien dans les égoûts.

Klermor d'Herinard -> Paladin au temple de la justice (Eril)

Etape 1: Enquete dans les sous sols 
 -> Quelques rats (bcp de rats morts)
 -> Jet de survie DC15, en cas d'echecs tombe dans une embuscade de champignons -> 5 violet fungus MM 138
 -> Trouve une "habitation" le necromant a fuit
 -> Les PJs tombent dans une embuscade -> 2 spectres mineurs MM 279 dont un poltergeist -> HP 18 domages 2d6 +3 pour toucher.
 -> Autel de pierre gravé de symboles étranges

Etape 2: Retour au temple
 -> C'est un signe d'un diable. Le nécromant a fait un pacte avec les enfers -> Azazel - Prince des enfers DC20 religion
 -> Envoie les PJs surveiller le cimetière du sud pendant la nuit en attendant
 -> Troisième nuit -> rencontre morts vivants -> rats nombreux si rien n'a été fait avec le tas de chez Orond -> 8 Giant zombie rats MM 327 -> fortitude du mort vivant MM316
 -> Rencontre le gardien du cimetière Paire en fait le nécromancien déguisé -> leur indique l'origine des troubles, un mosollé -> 3 ghouls MM148
 -> Découvrent un campement précaire ne semble plus habité DC15, voir jamais habité DC 20 -> Paire n'est pas trouvable ensuite

Etape 3: Rapport au temple
 -> Trop occupé avec un culte démoniaque qui prend de l'ampleur
 -> Conseil aux PJs de garder le cimetière contre le nécromancien
 -> Attaque par deux espions MM349 durant la journée chez eux -> d'autres gros bras travaillant dans les cimetières les ont embauchés


\subsection{En guerre contre la mort (Niv. 2)}

% Statut de mon groupe actuel 
Etape 1: Surveillance de divers cimetières
 -> Finissent par attraper un groupe de pilleurs de tombe COMBAT -> 1 espion MM349, 2 thugs MM350 et un demiogre MM238 
 -> Ils revendent des corps à la tour des mages, mais il est hors de question de témoigner -> ils ont peur

Etape 2: Un adversaire identifié
 -> Arrivent a découvrir les allés et venues suspicieuses du nécromant
 -> Lorsqu'il est attaqué il fuit avec porte dimensionnel et laisse les Pjs dans une embuscade COMBAT -> Flameskull MM134
 -> Renseignement sur son identité -> il est trop important pour être arreté sans preuves en béton
 -> L'équipe passe niveau 3

Etape 3: Aller chercher l'ennemi dans sa tanière
 -> Infiltration de la tour ou de la demeurre et COMBAT nécessaire -> Scouts MM349
 -> Surveillance de la tour
 -> Golems de bois dans son salon
 -> Le dernier combat du nécromant -> comme Herinos p18

\section{L'Alchimiste}

\subsection{Sur la piste du fungus (Niv. 1)}

Un alchimiste du nom de Ferrangus emploie les aventuriers pour aller chercher des champignons dans les
égouts. La mission est identique à la séquence "Les rats géants" page \pageref{ss:RatsGeants}, cueillir 
les champignons nécessite un jet sleight of hand DC15
si les PJs savent qu'il est dangereux. Sinon échec automatique. 1d6 dégats psy à cause des spores.

Peut rediriger vers la recherche des mort-vivants 

\subsection{À la chasse (Niv. 2)}

TROUVER CREATURES a chasser qui soit logiquement à la base de la création de golems. Doit comporter
un dilemme moral moyen de plus en plus fort

\subsection{Une sombre affaire (Niv. 2)}

Infiltrer les combats clandestin de rue pour enlever des gladiateurs. Ceux ci servent à fabriquer
des golems de chair. Dilemme moral majeur. Les PJ comprennent qu'ils travaillent pour un criminel
dangereux!
 -> L'équipe passe niveau 3

Combat contre l'alchimiste impossible en présence de ses machines de guerre (golems).

\section{Le Démoniste}

\subsection{Un culte turbulent (Niv. 1)}

Depuis quelques jours, les membres du temple du dieu de la justice (un groupe de clercs et paladins) 
suspectent la présence de démonistes. Malheureusement c'est hors de la ville dans un faubourg 
et donc sous la juridiction du duc qui n'est pas du tout interessé par ces histoires de religions.
Les paladins ne peuvent intervenir directement, il souhaitent que les PJs aillent jeter un oeil.

Étape 1: Le faubourg est pauvre et principalement habité par des ouvriers agricoles. Les aventuriers
s'y rendent, si ils questionnent à propos d'une secte, ils sont attaqués par 3 acolytes MM342. Si ils
trainent suffisamment dans le quartier avec un look assez pourri, ils se font alpaguer pour joindre le 
culte à minuit. Sinon ils doivent faire parler un cultiste.
-> Derrière l'auberge du fer à cheval se reunissent les membres du culte

Étape 2: Surprendre les cultistes. Les PJs peuvent assister aux culte en tant que croyant ou en éspions.
En tant que croyant, on leur demande de boire une coupe de sang "humain" (en fait du sang de cochon).
Les cultistes ne semblent pas très dangereux. Par contre, si ils sont pris à éspionner ils se font 
poursuivre par un groupe de paysans armés de gourdins MM345.

Étape 3: Les PJs font leur rapport au culte. Ils se font jeter si ils ont tués des civils, les paladins ne 
les étrippent pas simplement pour ne pas être impliqué. Sinon, les paladins proposent de payer plus pour 
que les PJs continuent à enquêter sur le leader du culte.

Étape 4: Les PJs peuvent essayer de coincer Bardan le gourou du culte (Fanatique MM345). Il ne parlera
sous aucun pretexte. Les PJs s'aperoivent que c'est un vrai clerc d'un dieu démoniaque. Il porte sur lui 
une amulette (CA+1). Le porteur commence à faire des cauchemars.

\subsection{Un véritable culte démoniaque (Niv. 2)}

Le porteur de l'amulette se fait aborder dans la rue par un cultiste. Si il sont consulté les paladins
proposent aux joueurs de s'infiltrer dans le culte au plus haut niveau. Le groupe se fait enroler. 
Les premières missions sont raisonnable et consiste a trouver un lieu de culte (nettoyage de quelques
fungis et morts vivants dans les egouts). Assez vite les cérémonies deviennent macabres, des hommes se 
scarifient au début, puis se sont des étrangers qui sont torturés. Un jour un jeune paladin a le coeur 
arraché, l'un des pj doit aider a tenir la victime. On demande au PJ d'enlever d'autres paladins ou
clercs et on leur propose de se scarifier. La seule question est quand arreteront-ils les frais. Vers le 
milieu de l'initiation, ils identifient le véritable leader du culte.

\subsection{Détruire le culte (Niv. 2)}

Le culte commence a invoquer des démons la déstruction (ou l'alliance totale?) devient nécessaire.

 -> L'équipe passe niveau 3
Les Paladins peuvent aider, la principale difficulté est de coincé les chefs du culte et en particulier
le sorcier. Il faut aussi affronter des invocations d'autant plus puissante que les PJs ont tardé à changer
de camps. 



\section{Suivi groupe}

Joueurs:
\begin{itemize}
  \item {\bf Ghech} sorcier drakeïde (Mikael)
  \item {\bf Myrdim} barde demi-elfe (Kévin)
  \item {\bf Seyonne} moine demi-elfe (Julien)
  \item \sout{{\bf Mialee} ranger haute elfe (Cyrielle)}
  \item {\bf Thia} druide haute elfe (Cyrielle)
\end{itemize}

Ghech est lié avec un grand-ancien par un pacte de chaine. Son familier est un 
monodron. Le grand ancien est à la recherches de connaissances pour une raison
inconnue.

Myrdim est un peu kleptomane, ce qui lui attire des ennuis régulièrement.

Seyonne a appris son art auprès de démonistes qui l'avaient enlevés

Mialee accueille un groupe d'orphelin dans leur petite maison délabrée. Elle
est malheureusement décédée lors d'une rixe dans les faubourgs nord.

Thia, la s\oe{}r de Milee est une druide du temple de la nature en centre ville.

Le groupe a fait la mission "Les rats géants", "Un nécromant en ville" et "Un culte turbulent"

Suivi niveau: faire un tableau

Familan un marchand de pots en cuivres et 
casseroles, c'est un vieux monsieur avec de l'embonpoint et une grosse 
moustache blanche. Il peut orienter les PJs vers la quête.
Orond est lui chambellan au c\oe{}ur des quartiers attaqué par les rats,


