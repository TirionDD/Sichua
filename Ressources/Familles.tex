\subsection*{Familles Nobles}

\subsubsection*{Ducs de Seryn}

Les ducs de Seryn sont la principale famille possedante de la région et disposent
d'une armée puissante en charge de la sécurité exterieur de la ville ainsi que
des régions environnantes. Ils sont aussi à la tête de la plus puissante flotte de 
la région et tiennent les pirates à bonne distance des côtes du duché.

Les Seryn furent ``Les Rois des Cinq\footnote{Humains, Elfes, Nains, Halfelins 
et Gnomes} Peuples'' avant d'être duc. Ils ont perdu Sichua il y a 1600 ans  
La guerre dura 250 ans pour tenter de reprendre la ville, il n'y eu aucun 
vainqueur et les menaces exterieurs finirent par forcé une alliance pour la
survie de la civilisation, ce fut l'origine du conseil des cinq.

- Le Duc, Hermans III de Seryn est un homme relativement agé et sage. Dans sa
  jeunesse, en plus de l'éducation de moine au monastère de la tranquillité,
  il a reçu l'éducation militaire traditionelle à son rang 
  (Guerrier maitre d'arme 9, moine paume 4)
  ce qui en fait un personnage parfois étrangement philosophique.

- Le fils ainé du duc, Hemerik de Seryn est un jeune politicien intelligent 
  qui gère déjà une bonne partie des affaires courantes de la famille. Il
  a reçu une éducation à la tour phare de magicien en plus de son éducation
  de jeune noble (Guerrier EK 6, Magicien Abjuration 4)

- Magdane de Seryn, jeune cousine du duc et général de son armée 
  (Guerrier champion 11). Elle passe
  l'essenciel de son temps préoccupé par le club des chasseurs. Elle n'est
  pas très intelligente et ne participe pas directement (ou involontairement)
  aux batailles de pouvoirs de la ville.

- L'oncle du duc Arthur de Seryn est un vieil homme très sage et bienfaisant.
  Il est le grand prètre du temple de la justice (Eril) de Sichua. Il a fait
  carrière comme prètre (clerc de la lumière lvl 14) et considère que la 
  justice est avant tout de nourrire les pauvres, ce qui en fait un être à
  part dans la haute noblesse.

- Noms pour d'autres membres de la famille: Énéril, Sérédil, Médéral, Rémère...

\subsubsection*{Comtes d'Herinar}

Les comptes d'Herinar sont la seconde plus puissante famille de la région. Néanmoins,
l'essenciel de leurs forces et possessions se trouvent au sud ouest à l'interieur
des terres. Ce sont les principaux rivaux des Seryn, mais leur éloignement
géographique ainsi que leur manque de force marine les empèchent d'égaler les Seryn.
Ils disposent néanmoins d'une armée de taille simillaire aux Seryn et certainement 
mieux entrainée, du fait des invasions fréquentes de créatures monstrueuses ou de
tribus sauvages en provenance des terres vièrges qui se trouvent au delà du compté. 

La famille tient son nom d'un ancien dragon dominant la région et que la famille 
servait comme intendants et intermédiaires avec les races civilisées. Les Herinar 
tiennent leurs pouvoirs de cette
ancienne alliance, les autres maisons nobles pensent que ces histoires sont des 
légendes et que les Herinar sont simplement des snobs. Pourtant c'est la verité,
mais avec le temps la puissance du sang d'Herinar s'affaiblie.
Pour cette raison, les Herinar privilégient les mariages entre cousins ou au moins
nobles ayant des Herinar dans leur arbre généalogique. Les Herinar ne disposant pas 
de pouvoirs magiques innée sont systématiquements exclu de la ligne de succession et
marié hors de la famille.

- Le Compte Tirion d'Herinar est un puissant ensorceleur et paladin de Kima
  (lignée draconique lvl 7 et paladin 5) qui mêne ses armés au combat de manière
  très efficace.

- Son fils ainé Klermor d'Herinar, dépourvu de pouvoirs d'ensorceleur, fut 
  considéré comme maudit par Kima. Il a alors rejoint le culte d'Eril et est à 
  présent le grand paladin de la justice à Sichua.

- Sa fille Killia d'Herinar est une puissante ensorceleuse, la plus puissante 
  depuis de nombreuses générations (lignée draconique lvl 11). Elle est 
  l'héritière officielle du compté, on lui cherche encore un époux de valeur.
  Si l'un des joueurs est un puissant ensorceleur originaire d'un autre famille,
  les Herinar tenterons certainement d'en savoir plus dans l'espoir d'amener un 
  nouveau sang magique dans la famille.

- Son neveu Tenère d'Herinar est un beau parleur et roublard (ensorceleur 3, 
  barde 3, roublard arcannique 3) servant d'ambassadeur à la famille à Seryn.
  On peut le croiser dans les bas fonds travestis, si on lui demande, c'est 
  uniquement pour obtenir des informations...

\subsubsection*{Barons d'Harmar}

La famille d'Harmar est à la tête de la prison. C'est une ancienne famille de 
grande influence au sein de l'école de magie.

- Baron Herinos d'Harmar (page \pageref{ErinosdHarmar}).

\subsubsection*{Barons Van Hjelmaster}

La famille van Hjelmaster est venu de loin, mais c'était il y a des 
siècles et malgré leur nom étranger, ils sont à Sichua chez eux. Leurs

Ils sont secrètement jaloux de Harmar et en particulier des secrets
qui entourent la prison. Mais le plus grand secret de la famille est
que le ``baron'' n'est pas véritablement le maître de la famille. C'est
son ancêtre Yalyr Van Hjelmaster, un puissant vampire, qui dirige réellement
la famille.

- Le baron Agranac Van Hjelmaster est un vieil homme aux cheveux rares
et blancs. Il porte généralement une tenue noire assez glauque. 

- La baronne Erina Van Hjelmaster sa femme

- Leur fils Soren Van Hjelmaster (voir page \pageref{SorenVH})

- Leur fille Pennilla Van Hjelmaster

\subsubsection*{Seigneurs d'Hertac}

À la tête de la prison

 - Germar d'Hertac, voir page \pageref{GermardHertac}. 

\subsection*{Familles Bourgeoise}

\subsubsection*{Karak}

Les Karak sont de puissants armateurs, les plus informés vous diront qu'ils
ne mouillent pas que dans des affaires très claires. En réalité ils sont les
armateurs de plusieurs navire pirate et le chaf de famille est le Maître pirate
de la guilde des voleurs.

- Harald Karak la cinquantaine, il dirige la famille. Ses fils sont capitaines 
sur divers navires. 

\subsubsection*{Dorant}

Les Dorant sont une famille de marchand d'étoffe installé entre les quai et
les quartiers plus riches. Ils font le commerce à la fois de grande 
quantité et de grande qualité. 

L'un des membres de la famille est alchimiste et en retrait suite au 
scandale décrit dans la campagne (page \pageref{TAS-Magie}).

\subsection*{Autres Familles Diverses}

