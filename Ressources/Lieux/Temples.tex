\section{Temples de Sichua}

\subsection{Brève théogonie}
D'abord fut {\bf CHAOS}, célébré par les sans dieux, (non-)patron de la mer 
astrale. \\
Puis furent {\bf LUMIÈRE} et {\bf NUIT}, représentés par un dieu et une déesse, 
tous deux neutres, dont les religions sont oubliées. Ils formèrent les plans 
élémentaires et la magie, puis les 9 dieux de l'archimonde: \\
- {\bf Dieu de la justice, ERIL (Céleste-LB)} patron du plan de l'Elysée, grand 
       temple en centre ville sur l'îlot. \\
- {\bf Dieu de la mort, OULTOR (Monstruosité-LN)} patron du plan de l'ombre, 
       grand temple au cimetière du sud-ouest et chez les nains. \\
- {\bf Déesse des enfers, AZAZEL (Diable-LM)} patronne du plan des enfers, il 
       est interdit mais les Hjelmaster ont un temple au sous-sol de leur 
       demeurre. \\
- {\bf Déesse de l'agriculture, EWEA (Humanoïde-NB)} patronne du plan des 
       champs infinis, grand temple sur la place du marché aux légumes, 
       mausolées nombreux dans les faubourgs ouest et chez les nains. \\
- {\bf Dieu de la nature, SYLVANUS (Plante-N)} patron du plan materiel, temple 
       à la source sur l'îlot. \\
- {\bf Déesse de la pourriture et de la décomposition, KEMILLE (Humanoïde-NM)} 
       patronne du plan de la guerre, il est interdit mais quelques mausolées 
       sont dans les égouts et dans les tribus d'orques, gobelins etc. \\
- {\bf Déesse des colporteurs et marchands, RILLA (Fée-CB)} patronne du plan 
       féérique, temples nombreux sur la rive droite, principal au sein de la 
       prévoté. \\
- {\bf Déesse de la mer et des tempètes, KIMA (Dragon-CN)} patronne du plan 
       sauvage, grand temple sur un îlot devant la ville et des mausolées 
       dans les ports. \\
- {\bf Dieu démon, THROKHRARTH (Démon-CM)} patron du plan des abysses, il est 
       interdit mais il existe un cercle secret à sa botte. \\

\subsection{Temple de la Justice}

Dedié au dieu Eril, le temple de la justice trone en plein centre de la ville 
sur l'île de la Source. On remarque immédiatement ce bâtiment massif à la 
façade couverte de marbre qui donne directement sur l'avenue du duc Erland. 
Plusieurs statues ornent la façade un dizaine de mêtres au dessus du sol, 
celles-ci représentent les grands prêtres les plus illustres de l'histoire 
du temple et quelques grands paladins. 

La légende indique que le temple fut érigé aux débuts de la cités il y a 
environ 3400 ans, mais tout bon érudit vous dira que c'est fort improbable. 
La cité était alors bien trop pauvre et un édifice de cette taille date très 
certainement du premier grand age d'or de la ville, il y a environ 1800 ans,
les couvertures en marbres et les scultures ont quand a elle probablement 
quelques siècles tout au plus.

Les postes d'importance dans le temple sont le grand prètre (Arthur de Seryn), 
le grand paladin (Klermor d'Herinard) et le grand justice

\subsection{Temple de la Mort}

Au bord du grand cimetière du sud-ouest. Construction assez recente. Plusieurs petits mausolés annexes.

\subsection{Temple d'Azazel}

Temple secret dissimulé chez les Hjelmaster.

\subsection{Temple de l'Agriculture}

Au marché aux produits agricole, donne sur la place.

\subsection{Temple de la Source}

Dédié au dieu de la nature sur l'ile de la source dont il est l'origine du nom.

\subsection{Temple de la Pourriture}

Dieu interdit et essenciellement inconnu. Il en existe un temple caché, oublié de tous... pour le moment.

\subsection{Temple du Commerce}

Plusieurs temples en ville. Un grand temple forme une annexe au palais 
du prévot des marchands. Le culte est riche et bénéficie des nombreuses
donnations des bourgeois de la ville. Les clerc du dieu consacre leurs
temples pour y empêcher l'espionnage magique 

\subsection{Temple de la Mer}

Grand temple sur un ilot isolé devant la ville. Nombreux mausolés dans les 
zones portuaires. Un temple important consacré à Kima se trouve dans la 
ville d'Herinar du fait de la dévotion de la famille pour cette déesse. 
Ce dernier est consacré à un autre aspect de Kima en charge des terres
sauvages que les d'Herinar se consacre à dompter.

Le culte de Kima est connu pour des rites d'initiations violents pour
leur clergé et nombre de personnes pense que la religion devrait être
interdite comme d'autres. Le support des marins et des comtes d'Herinar 
rend evidemment une telle mesure impossible sans créer des troubles
graves dans toute la région.

\subsection{Temple du Culte Démoniaque}

Enfoui sous la colline des nobles par un accès provenant d'un petit port du 
sud. Il est fréquenté par des pirates et des fous, le secret est bien gardé. 
On dit que quelques nobles protègent le lieux pour y pratiquer de temps à
autres des invocations et des pactes avec de puissants démons. Une pratique
aussi dangereuse qu'illégale mais qui peut permettre à certains de gagner
de puissants pouvoirs.

\subsection{Déesse de la lumière et de la nuit}


