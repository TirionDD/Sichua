\section{Temples de Sichua}

\subsection{Brève théogonie}
D'abord fut {\bf CHAOS}, célébré par les sans dieux, (non-)patron de la mer astrale. \\
Puis furent {\bf LUMIÈRE} et {\bf NUIT}, représentés par un dieu et une déesse, tous deux neutres, dont les religions sont oubliées. Ils formèrent les plans élémentaires et la magie.\\
Enfin vinrent les 9 dieux de l'archimonde: \\
- {\bf Dieu de la justice, ERIL (Céleste LB)} patron du plan de l'Elysée, grand temple en centre ville sur l'îlot. \\
- {\bf Dieu de la mort, OULTOR (Monstruosité-LN)} patron du plan de l'ombre, grand temple au cimetière du sud-ouest. \\
- {\bf Déesse des enfers, AZAZEL (Diable-LM)} patronne du plan des enfers, il est interdit mais les Hjelmaster ont un temple au sous-sol de leur demeurre. \\
- {\bf Déesse de l'agriculture, EWEA (Fée-NB)} patronne du plan féérique, grand temple sur la place du marché aux légumes, mausolées nombreux dans les faubourgs ouest. \\
- {\bf Dieu de la nature, SYLVANUS (Humain-N)} patron du plan materiel, temple à la source sur l'îlot. \\
- {\bf Déesse de la pourriture et de la décomposition, KEMILLE (Gobelinoïde-NM)} patronne du plan de la guerre, il est interdit mais quelques mausolées sont dans les égouts. \\
- {\bf Déesse des marchands, RILLA (Gnome-CB)} patronne du plan de la mécanique et de la magie, temples nombreux sur la rive droite, principal au sein de la prévoté. \\
- {\bf Déesse de la mer et des tempètes, KIMA (Dragon-CN)} patronne du plan sauvage, grand temple sur un îlot devant la ville et des mausolées dans les ports. \\
- {\bf Dieu démon, THROKHRARTH (Démon-CM)} patron du plan des abysses, il est interdit mais il existe un cercle secret à sa botte. \\

\subsection{Temple de la Justice}

Dedié au dieu Eril, le temple de la justice trone en plein centre de la ville sur l'île de la Source.
On remarque immédiatement ce bâtiment massif à la façade couverte de marbre qui donne immédiatement
sur l'avenue du duc Erland. Plusieurs statues ornent la façade un dizaine de mêtres au dessus du sol,
celles-ci représentent les grands prêtres les plus illustres de l'histoire du temple. Celui-ci fut 
d'après la légende érigé aux débuts de la cités il y a environ 1400 ans, mais tout bon érudit vous
dira que c'est fort improbable. La cité était alors bien trop pauvre et un édifice aussi luxueu date très 
certainement du premier grand age d'or de la ville, il y a environ 1000 ans.

PNJ: Grand Prètre, Maître Paladin (Klermor d'Herinard), Le Grand Justice

\subsection{Temple de la Mort}

Au bord du grand cimetière du sud-ouest. Construction assez recente. Plusieurs petits mausolés annexes.

\subsection{Temple d'Azazel}

Temple secret dissimulé chez les Hjelmaster.

\subsection{Temple de l'Agriculture}

Au marché aux produits agricole, donne sur la place.

\subsection{Temple de la Source}

Dédié au dieu de la nature sur l'ile de la source dont il est l'origine du nom.

\subsection{Temple de la Pourriture}

Dieu interdit et essenciellement inconnu. Il en existe un temple caché, oublié de tous... pour le moment.

\subsection{Temple du Commerce}

Plusieurs temples en ville. Un grand temple forme une annexe au palais du prévot des marchands.

\subsection{Temple de la Mer}

Grand temple sur un ilot isolé devant la ville. Nombreux mausolés dans les zones portuaires

\subsection{Temple du Culte Démoniaque}

Enfoui sous la colline des nobles par un accès provenant d'un petit port du sud. Il est fréquenté par des pirates et des fous. Personne de sensé ne s'en approche sciemment.
