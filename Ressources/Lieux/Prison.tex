\section{La Prison}

Sur l'ilot de la source, donnant sur le bras sud du Tervyn, se trouve un large batiment carré à l'allure sombre
tranchant avec les batiments officiels somptuairs. Ce batiment formé
de quatre ailes autour d'une large cour interieure et courronné d'un donjon
massif est la prison de Sichua.

\subsection{Aile Nord}

L'aile nord accueil en son centre la porte d'entrée de la prison et forme la 
façade donnant sur la grande rue. Elle accueille les appartements des gardes 
et les bureaux administratifs. C'est ici que sont géré l'accueil des nouveaux 
prisonniers et la libération des pentionnaires ayant fini de servir leur 
peine. Un certain nombre de garde y ont aussi un petit appartement, ceci fait 
parti des privilèges du corps des gardes de la prison (utiliser les 
statistiques des vétérants page \pageref{Veteran} pour les gardes de la 
prison). Quoiqu'il ne soit pas très enviable d'habiter une prison, le simple 
fait d'avoir un logement de fonction en plein centre de la ville renforce 
l'attrait des postes de garde. Cet emploi est particulièrement prisé dans 
les couches les plus populaire de la ville et le statut qu'il offre renforce 
la fidélité et l'honêteté des gardes.

\begin{figure}[btp]
\center
\fbox{%
  \begin{minipage}[c]{\linewidth}
    \label{GermardHertac}
    {\bfseries\LARGE\scshape Germar d'Hertac} \\
    Humain de taille M, neutre \\
    \noindent\rule{\textwidth}{1pt} \\
    {\bfseries Classe d'armure} 19 (clibanion et bouclier)\\
    {\bfseries Points de vie} 85 (13d8 + 27) \\
    {\bfseries Vitesse} 9 m \\
    \noindent\rule{\textwidth}{1pt} \vskip 2pt
    \setlength{\tabcolsep}{4pt}
      {\footnotesize 
    \begin{tabular}{cccccc}
      \bf FOR & \bf DEX & \bf CON & \bf INT & \bf SAG & \bf CHA \\
       19 (+4) & 10 (0) & 16 (+3) & 10  (0) & 12 (+1) & 15 (+2) \\
    \end{tabular} }
    \setlength{\tabcolsep}{6pt}
    \noindent\rule{\textwidth}{1pt} \\
    {\bfseries Jets de sauvegarde} For +7, Con +6 \\
    {\bfseries Compétences} Athlétisme +7, Perception +4, Persuasion +5 \\
    {\bfseries Sens} Perception passive 14 \\
    {\bfseries Langues} commun \\
    {\bfseries Facteur de puissance} 4 (1100 XP)
\vspace{-10pt}
    \subsection*{Actions}
    {\bfseries Attaques multiples.} Germar réalise trois attaques, deux avec son épée longue et une
               avec son bouclier.\\
    {\bfseries Épée longue.} Attaque d'arme de corps à corps : +7 au toucher, allonge 1,50 m, 
               une cible. Dégâts : 8 (1d8 + 4) dégâts tranchants. \\
    {\bfseries Coup de bouclier.} Attaque d'arme de corps à corps : +7 au toucher, allonge 1,50 m, 
               une créature. Dégâts : 6 (1d4 + 4) dégâts contondants. Si la cible est une créature de 
               taille M ou plus petite, elle doit réussir un jet de sauvegarde de Force DD 14 pour ne 
               pas tomber à terre.\\
    \noindent\rule{\textwidth}{1pt} \\
Quatrième fils d'une petite famille noble, Garmar a toujours su qu'il n'aurait 
rien d'autre que son nom en héritage. Il a donc rapidement décidé de se lancer 
dans le metier des armes. Après quelques
campagnes avec le club des chasseurs, il a pris conscience du danger que pouvait 
comporter ces expéditions. Les excursions militaires du duc étant encore plus
risqués, Garmar utilisat ses relations pour obtenir un poste de garde à la prison.
Son statut de noble et son intelligence sociale lui ont rapidement permis de 
monter dans la hierarchie de la prison. Il est à présent le capitaine de la
garde et déjeune avec le directeur Herinos tous les jours pour lui faire son
rapport sur la securité de la prison.

Pour Germar la question se pose à présent de trouver un moyen de s'établir
avec une jeune femme de meilleure fortune que lui.
C'est avec cet objectif en tête qu'il se comporte avec la plus grande 
diligence auprès des puissants, nobles ou bourgeois, de la prison.
 \end{minipage}
}%
\end{figure}


\subsection{Aile Est}

L'aile est accueille les prisonniers de basse suretée, que ce soit par leur
origine, leur fortune ou le simple fait que leurs crimes soient mineurs. Ces
prisonniers disposent d'un traitement de faveur et d'une relative liberté 
dans la prison, les moins
riches travaillent pour la prison à divers postes: cuisines, nettoyage
de la cour, entretien des écuries ou simplement comme laquais pour 
des nobles résidents de l'aile nord. Les dortoires y sont bondés mais 
sont installés dans de grandes salles bien aérée. Pour certains des
pentionnaires des couches les plus populaires, cette aile de la prison
est une aubaine et on dit que certains prisonniers agés sont ici bien
au-delà de leur peine initiale, mais ne veulent simplement plus partir.
Non pas que la vie dans le quartier basse suretée soient très comfortable,
mais elle assure un toit et un niveau de violence bien plus limité que certains
faubourgs de la ville. La direction ferme les yeux sur ces prisonniers car
ils forment une main d'oeuvre au coût très limité et une source d'information
fiable sur les activités des autres pensionnaires.

\subsection{Aile Ouest}

L'aile ouest comporte diverses salles utilitaires, en particulier les cuisines 
qui fournissent des repas différents pour l'aile nord, pour les gardes et pour 
les prisonniers communs. Les écuries se 
trouvent aussi dans cette aile, on y trouve quelques chevaux de trait pour
les carosses de prisonniers utilisés par la garde pour les transfers vers les 
différentes cours de justice de la ville. On y trouve aussi quelques salles d'audience, utilisé par
les juges qui considèrent trop dangereux de transporter un prisonnier hors de
la prison jusqu'au tribunal. Ces salles peuvent aussi
servir aux interogatoires que les autorités veulent garder secrets. Enfin cette aile
accueil le logement du bourreau de la prison. Celui-ci dispose d'une porte 
vers l'exterieur et ne se montre jamais dans la prison sans sa cagoule 
noire. Sa seule vue fait frissonner toutes la prison jusque dans ses 
entrailles.

\begin{figure}[b!]
\center
\fbox{%
  \begin{minipage}[c]{.95\linewidth}
    \label{Ystar}
    {\bfseries\LARGE\scshape Ystar} \\
    Nain de taille M, loyal mauvais \\
    \noindent\rule{\textwidth}{1pt} \\
    {\bfseries Classe d'armure} 11 \\
    {\bfseries Points de vie} 17 \\
    {\bfseries Vitesse} 9 m \\
    \noindent\rule{\textwidth}{1pt} \vskip 2pt
    \setlength{\tabcolsep}{4pt}
      {\footnotesize 
    \begin{tabular}{cccccc}
      \bf FOR & \bf DEX & \bf CON & \bf INT & \bf SAG & \bf CHA \\
       8 (-1) & 12 (+1) & 10 (+0) & 16 (+3) &  14 (+2) &  8 (-1) \\
    \end{tabular} }
    \setlength{\tabcolsep}{6pt}
    \noindent\rule{\textwidth}{1pt} \\
    {\bfseries Sens} Perception passive 12 \\
    {\bfseries Langues} commun, comptable \\
    {\bfseries Facteur de puissance} STAT d'un sorcier (0 XP)\\
    hache buveuse d'ame \\
    \noindent\rule{\textwidth}{1pt} \\
Ystar est un nain de la communauté de la colline du phare,
c'est aussi le bourreau de la prison de Sichua. Depuis sa jeunesse
Ystar est fasciné par la mort et les arts occultes. Lors de ses recherches
dans les bibliothèques naines, Ystar a retrouvé des documents datant
de la construction de la prison. Il a alors décidé de s'y rendre pour en
comprendre le pouvoir, ses études l'ont lié à la prison d'une manière
unique et les runes inscrites sur sa hache lui permettent de communiquer
avec les ésprits prisonniers. Ystar tire de la prison un pouvoir personel
hors du commun, néanmois c'est un jeu dangereux et Ystar est loin d'être 
qualifié pour pratiqué une telle magie. Si Herinos venait à apprendre à quoi
joue son bourreau, Ystar serait sans aucun doute immédiatement remplacé et 
très certainement la première victime de son successeur!
 \end{minipage}
}%
\end{figure}


\subsection{Aile Sud}

Au fond de la prison se trouve l'aile des nobles. Celle-ci donne 
directement sur la rivière. Le directeur de la prison y habite avec sa 
famille, dans des appartements luxueux, d'autres appartements y sont 
réservés à la noblesse et la bourgeoisie. Pour des raisons de paix 
sociale dans la ville, il arrive qu'un juge se voit forcé de condamner une personne
de haut rangs, celle-ci va alors vivre dans ce petit havre de paix en bord
de rivière. On voit 
de temps à autres des figures cagoulés se glisser dans 
des barques depuis la prison. Ce sont généralement ces prisonniers, car il est 
toléré par la direction qu'ils puissent s'absenter quelques jours de temps 
à autre, on leur demande simplement de ne pas trop se montrer en ville 
tant que leur peine n'est pas officiellement terminée.

\begin{figure*}[tb!]
\center
\fbox{%
  \begin{minipage}[c]{.45\linewidth}
    \label{ErinosdHarmar}
    {\bfseries\LARGE\scshape Herinos d'Harmar} \\
    Humain de taille M, loyal neutre \\
    \noindent\rule{\textwidth}{1pt} \\
    {\bfseries Classe d'armure} 12 (15 avec armure de mage) \\
    {\bfseries Points de vie} 40 (9d8) \\
    {\bfseries Vitesse} 9 m \\
    \noindent\rule{\textwidth}{1pt} \vskip 2pt
    \setlength{\tabcolsep}{4pt}
      {\footnotesize 
    \begin{tabular}{cccccc}
      \bf FOR & \bf DEX & \bf CON & \bf INT & \bf SAG & \bf CHA \\
       9 (-1) & 14 (+2) & 10 (+0) & 17 (+3) & 12 (+1) &  15 (+2) \\
    \end{tabular} }
    \setlength{\tabcolsep}{6pt}
    \noindent\rule{\textwidth}{1pt} \\
    {\bfseries Jets de sauvegarde} Int +6, Sag +4\\
    {\bfseries Compétences} Arcanes +6, Histoire +6, Persuasion +5\\
    {\bfseries Sens} Perception passive 12 \\
    {\bfseries Langues} commun, draconique, elfique, infernal \\
    {\bfseries Facteur de puissance} 6 (2300 XP)\\
   \noindent\rule{\textwidth}{1pt} \\
    {\bfseries Sorts.} Un mage est un lanceur de sorts de niveau 9. Sa caractéristique pour lancer 
               des sorts est l'Intelligence (sauvegarde contre ses sorts DD 14, +6 au toucher avec des 
               attaques de sort). Un mage a les sorts de magicien suivant préparés :\\
Sorts mineurs (à volonté) : trait de feu, lumière, manipulation à distance, prestidigitation\\
Niveau 1 (4 emplacements) : armure de mage, bouclier, détection de la magie, rayon de maladie\\
Niveau 2 (3 emplacements) : foulée brumeuse, cécité/surdité\\
Niveau 3 (3 emplacements) : malédiction, contresort, vol
 \end{minipage}
  \hspace{4pt}
 \begin{minipage}[c]{.45\linewidth}
Niveau 4 (3 emplacements) : invisibilité suprême, flétrissement\\
Niveau 5 (1 emplacement) : domination
 \vspace{-10pt}
    \subsection*{Actions}
    {\bfseries Dague.} Attaque d'arme de corps à corps ou à distance : +5 au toucher, allonge 1,50 m ou portée 6/18 m,
         une cible. Dégâts : 4 (1d4 + 2) dégâts perforants. \\
   \noindent\rule{\textwidth}{1pt} \\
Herinos est un noble à la formation de magicien. Les membres de sa famille
dirigent la prison depuis de nombreuses générations et en connaissent tous les 
secrets. Herinos à remplis son devoir familliale depuis son plus jeune age et
suivi une formation de nécromancien à la tour phare pendant sa jeunesse pour 
prendre en charge la prison après son père. Il entretient les sortilèges de 
protection de la prison et dirige tous les aspects de la prison. Son poste est 
très réspécté en ville et rare sont ceux qui se confronte à sa famille. 

Herinos est un homme droit et peut enclun à profiter de manière abusive de
sa position, mais il connait tout de même ses intêrets et ne laissera personne
mettre en doute l'efficacité de sa prison ou son autorité sur cette institution.
 \end{minipage}
}%
\end{figure*}

\subsection{Donjon}

Au centre de la cour un énorme donjon de granit d'une trentaine de
mètres est entouré d'une fosse
d'une dizaine de mètres de fond. Devant la porte, un 
cercle de dalles de granit d'une dizaine de mètres de diamètre sert de lieu 
d'executions. Dans la tour,
le rez de chaussé est occupé par la salle des gardes. Les étages sont 
remplis de cellules aux portes de fer. Celle-ci sont scellée lors de 
l'emprisonnement et seulement déscellée pour une libération ou une 
execution. Dans ces cellules sont gardés les prisonniers de moyennes 
sécurités. Au dernier étage, se trouvent les cellules des 
condamnés à mort qui descendent le grand escalier en collimasson passant
devant chaque cellule pour leure dernière sortie. Une petite trappe dans 
la porte de chaque cellule permet d'apporter 
les repas et de sortir les bacs de commodités chaque jours. Ce travail est effectué par 
des prisonniers de basse sécurité de confiance. Généralement d'anciens 
mendiants qui ne voudraient être libéré sous aucun prétexte. Au sous-sol quelques cellules sont 
isolées par de nombreuses portes, elles accueillent les prisonniers 
au secret. Ce sont des gardes qui s'occupent d'ammener la nourriture a ces 
prisonniers.

Dans le dernier sous sol du donjon se trouve une étrange machine. Une cage de fer tenu 
par un cable, celle-ci peut descendre dans un tunel vertical fort profond
jusqu'au lieu de détention des prisonnier de haute sécurité. Pour une 
raison ou une autre, le juge ne pouvait pas pronnoncer la peine de mort
ou alors souhaitait il une peine pire que la mort? Au fond de ce tunnel une grande
salle sphérique. Les murs sont fait d'admantium pur, épais de plusieurs 
centimètres. Un garde descend chaque jour les rations dans la cage pour nourrire les 
survivants de cette salle. Parfois un prisonnier 
essaye de remonter avec l'ascenceur. Lorsque cela arrive, les gardes redescendent la cage et 
laissent les prisonniers sans nourriture quelques jours. L'ascenceur n'est utilisé
que pour apporter de nouveaux prisonniers, jamais pour en sortir un. En cas de
problème ou si ils soupçonnent une tentative de fuite, les gardes ont la possibilités 
de déverser des feux gréjoix dans le tunnel et de bruler tous les prisonniers 
haute-sécurité vifs.

\subsection{La Prison des Esprits}

Le secret de la sécurité du donjon est sa protection magique. La prison
agit comme un aimant pour les ésprits de ceux qui y meurent, fournissant
l'énergie magique nécessaire à la prison. Ces âmes damnées assoifées de vengence
contre les vivants sont prêtes à tout 
pour empecher la moindre évasion. L'énergie des ésprits alimente la magie de 
la prison et bloque toutes 
tentatives de teleportation que ce soit depuis ou vers la prison. La magie est d'une
puissance telle qu'elle reconstruit aussi murs et barreaux de fenètres
continuellement. Enfin, si quelqu'un venait à attérir sur le toit du donjon, il 
serait immédiattement attaqué par des spectres (page \pageref{Spectre}).
Ce système de protection magique a une conséquence particulerement effrayante,
la prison de Sichua permet de garder ses prisonniers même après la mort.
Les ames de
ceux qui meurent dans la prison ou dans le cercle d'exécution sont en effet piégées et
ces personnes ne peuvent rejoindre ni les enfers ni le paradis. Enfin elles ne
peuvent pas non plus être rescussités... Cette caratéristique de la prison est peu connue
en ville, car les autorités préfèrent rester discrètes à ce sujet. Il n'y a
en effet pas de moyen pour la prison de différencier les
gardes des prisonniers, ainsi les gardes mourrant dans l'enceinte de la prison
subissent le même sort que les prisonniers, or une telle nouvelle rendrait le 
recrutement de nouveaux gardes significativement plus difficile...


{\bf Possible add-on:} deux trois prisonniers connus.


