\section{Boutique Merveilleuse d'Hella RocheNoire}

\begin{figure*}[tb!]
\center
\fbox{%
  \begin{minipage}[c]{.45\linewidth}
    \label{ErinosdHarmar}
    {\bfseries\LARGE\scshape Hella RocheNoire} \\
    Gnome des roches de taille S, loyal neutre \\
    \noindent\rule{\textwidth}{1pt} \\
    {\bfseries Classe d'armure} 14 (17 avec armure de mage) \\
    {\bfseries Points de vie} 72 (13d6+26) \\
    {\bfseries Vitesse} 9 m \\
    \noindent\rule{\textwidth}{1pt} \vskip 2pt
    \setlength{\tabcolsep}{4pt}
      {\footnotesize 
    \begin{tabular}{cccccc}
      \bf FOR & \bf DEX & \bf CON & \bf INT & \bf SAG & \bf CHA \\
       7 (-2) & 17 (+3) & 14 (+2) & 20 (+5) & 18 (+4) &  15 (+2) \\
    \end{tabular} }
    \setlength{\tabcolsep}{6pt}
    \noindent\rule{\textwidth}{1pt} \\
    {\bfseries Jets de sauvegarde} Int +10, Sag +9\\
    {\bfseries Compétences} Arcanes +15, Histoire +15, Intuition +9, Investigation +10 (+ avantage), Perception +9, Persuasion +7\\
    {\bfseries Sens} Perception passive 19 \\
    {\bfseries Langues} commun, draconique, sylvain, infernal, abyssal \\
    {\bfseries Facteur de puissance} 10 (??? XP)\\
   \noindent\rule{\textwidth}{1pt} \\
    {\bfseries Sorts.} Hella est une lanceuse de sorts de niveau 13. Sa caractéristique pour lancer 
               des sorts est l'Intelligence (sauvegarde contre ses sorts DD 18, +13 au toucher avec des 
               attaques de sort). Un mage a les sorts de magicien suivant préparés :\\
Sorts mineurs (à volonté) : façonnage de la terre, contact glacial, lumière, manipulation à distance, réparation\\
Niveau 1 (4 emplacements) : armure de mage, bouclier, détection de la magie, orbe chromatique\\
Niveau 2 (3 emplacements) : foulée brumeuse, localisation d'objet, image miroir\\
Niveau 3 (3 emplacements) : don des langues, contresort, lueurs hypnotiques \\
Niveau 4 (3 emplacements) : façonnage de la pierre, invisibilité suprême, terrain hallucinatoire\\
Niveau 5 (2 emplacement) : transmutation de la roche, main de Bigby \\
Niveau 6 (1 emplacement) : défense magique, chaîne d'éclair \\
Niveau 7 (1 emplacement) : épée de Mordenkainen
 \end{minipage}
  \hspace{4pt}
 \begin{minipage}[c]{.45\linewidth}
 \vspace{-10pt}
    \subsection*{Actions}
    {\bfseries Dague.} Attaque d'arme de corps à corps ou à distance : +8 au toucher, allonge 1,50 m ou portée 6/18 m,
         une cible. Dégâts : 5 (1d4 + 3) dégâts perforants. 
    \subsection*{Réactions}
Hella peut créer une copie illusoire d'elle-même en un instant, comme un réflexe instinctif face à un danger. Quand une créature fait un jet d'attaque contre elle, elle peut utiliser sa réaction pour interposer ce double illusoire. L'attaque manque automatiquement et l'illusion se dissipe. Une fois que Hella a utilisé cette capacité, elle ne peut plus l'utiliser à nouveau jusqu'à ce qu'elle termine un repos court ou long. \\
   \noindent\rule{\textwidth}{1pt} \\
Hella est une des anciennes du clan RocheNoire originaire de collines à 
proximité de la grande forêt du nord. Après une jeunesse aventureuse, elle 
fit sa fortne en revendant les objets magiques qu'elle avait accumulée durant 
ses expeditions. Armée de son sens du commerce, elle pris son temps pour vendre
ses objets un à un au meilleur prix. Assez rapidement, elle se mit à vendre les
objets magiques d'autres groupes et son succès a été telle que finalement 
aujourd'hui à Sichua personne ne songerait faire une vente d'objet magique 
sans passer par son expertise.

Hella porte plusieurs objets magique puissants une baguette de mage de guerre +3 (page ??), des 
yeux grossissants (??) et une pierre ionique de protection ?? (page ??). Elle porte aussi un collier 
qui lui indique si un intrus est dans la salle des coffres de sa maison, le 
collier ne s'active pas si l'intrus est un gnome du clan RocheNoire.

\end{minipage}
}%
\end{figure*}

En début de falaise dans les quartiers bourgeois vos trouverez cette
maison très bourgeoise mais pas très haute (2.20 de hauteur de plafond).
Vous serez certainement arrivée ici sur recommendation, la boutique 
d'Hella RocheNoire est la destination incontournable de ceux qui 
recherchent des objets magiques. Elle dispose d'un quasi-monopole sur
les objets magiques en ville, leur commerce est rare et dangereux car
on a vite fait de se faire arnaquer avec un objet dont le sort est 
seulement temporaire ou pire maudit. Hella RocheNoire sert donc 
d'intermédiaire dans toutes les principales transactions d'objets
magiques et est la seule commerçante de la ville à disposer d'un stock
d'objets magiques conséquent.

L'entrée de la maison ouvre sur plusieurs salles d'attentes. Après y avoir
été guidé par un servant en livré, on y découvre un mobilier luxueux et on 
y sert du thé et des biscuits aux visiteurs en attendant d'être introduit
auprès de la maîtresse de maison.

Le bureau de Hella Rochenoire est une salle très étendu, son fils Helim 
et sa fille Hydale ont des bureaux près de l'entrée alors que leur mère 
occupe le centre de la pièce. Les murs sont couvert de livres et d'anciens 
manuscrit et parchemins, un oeil averti remarquera que ce sont principalement 
des journaux d'aventurier, principalement des mages ou bardes. Lorsque Hella 
reçoit des clients, ses enfants restent là pour travailler mais se tiennent 
à disposition de leur mère pour jeter un oeil sur un objet ou aller chercher 
un article précieux dans les coffres sous la demeurre.

\begin{table*} [tb]
    \setlength{\tabcolsep}{4pt}
    \center
\begin{tabular}{cccc}
  \bf d100 & \bf Nom de l'objet & \bf Prix & \bf Page \\
   \rowcolor{LightCyan}
   3 & Flèches +2 (x40)            & 120 PO chaque & \pageref{} \\
   4 & Flèchettes +2 (x20)         & 150 PO chaque & \pageref{} \\
   \rowcolor{LightCyan}            
   8 & Dague vicieuse              &        800 PO & \pageref{} \\
  13 & Bouteille fumante           &      1 000 PO & \pageref{} \\
   \rowcolor{LightCyan}
  15 & Bracelet d'archer           &      2 500 PO & \pageref{} \\
  20 & Armure de cuir clouté +1    &      3 000 PO & \pageref{} \\
   \rowcolor{LightCyan}
  21 & Bottes elfiques             &      4 500 PO & \pageref{} \\
  29 & Cape de protection          &      5 500 PO & \pageref{} \\
   \rowcolor{LightCyan}
  31 & Épée courte +2              &      7 000 PO & \pageref{} \\
  34 & Pierre ionique de régénération &   8 000 PO & \pageref{} \\
   \rowcolor{LightCyan}
  37 & Scimeterre de vitesse       &     10 000 PO & \pageref{} \\
  42 & Baguette de missile magique &     10 000 PO & \pageref{} \\
   \rowcolor{LightCyan}
  45 & Necklace of Fireball (6)    &     12 000 PO & \pageref{} \\
  48 & Armure de plate des nains   &     15 000 PO & \pageref{} \\
   \rowcolor{LightCyan}
  54 & Casque de telépathie        &     17 000 PO & \pageref{} \\
  55 & Baguette du garde du pacte +2 &   20 000 PO & \pageref{} \\
   \rowcolor{LightCyan}
  58 & Baton de feu                &     25 000 PO & \pageref{} \\
  63 & Arc court +3                &     30 000 PO & \pageref{} \\
   \rowcolor{LightCyan}
  68 & Bouclier sentinel           &     30 000 PO & \pageref{} \\
  75 & Pierre Ionique d'absorbtion &     40 000 PO & \pageref{} \\
   \rowcolor{LightCyan}
  80 & Baguette de boule de feu    &     50 000 PO & \pageref{} \\
  89 & Gemme de vision             &     50 000 PO & \pageref{} \\
   \rowcolor{LightCyan}
  90 & Armure Demi plaque +3       &     60 000 PO & \pageref{} \\
  91 & Anneau de l'archimage       &     60 000 PO & \pageref{} \\
   \rowcolor{LightCyan}
  92 & Baton des forêts            &     63 000 PO & \pageref{} \\
  97 & Robe des étoiles            &     75 000 PO & \pageref{} \\
\end{tabular}
    \setlength{\tabcolsep}{6pt}
\end{table*}

 
A droite du bureau, la chambre de Hella et sa salle d'étude personnelle
il y a une porte entre cette salle et le bureau, a gauche les chambres
de ses enfants Helim et Hydale ainsi que la cuisine où se trouve les deux
servants de la maison Renald et Krimmelle.

Dans l'étude d'Hella se trouve les escaliers pour descendre au sous-sol.
Seul un Rochenoire peut en ouvrir la porte, mais un jet d'investigation DD 18
peut permettre de détecter l'entrée. Il n'y a néanmoins aucun moyen 
d'ouvrir la porte de manière mécanique et toute tentative de forcer la 
porte, qui a une CA de 14 et 50 point de vie, voit l'auteur de l'attaque 
ciblé par un sortilège de petrification que son attaque ai réussit ou non 
(le sort est rélancé à chaque attaque). Il doit 
alors faire un jet de sauvegarde de constitution DD 18. En cas d'échec de 5 points ou 
plus, la créature est immédiatement pétrifiée. En cas de simple échec, 
elle commence à se changer en pierre et est entravée. Elle doit retenter 
le jet de sauvegarde à la fin de son prochain tour. En cas de succès, 
l'effet se termine. En cas d'échec, la créature est pétrifiée, jusqu'à 
ce qu'elle soit délivrée par un sort de restauration suprême ou par une 
magie similaire. 

Au sous sol se trouve une grande salle de la taille du bureau d'Hella, il 
va sans dire que le lieux est protégé contre la magie et la téléportation
par un puissant champ d'anti-magie. Une centaine de coffres de toutes tailles 
s'y trouvent, tous sceller dans la roche au sol ou aux murs. Chaque coffre
nécessite un jet d'outil de voleurs DC 18 pour être ouvert, un jet 
d'investigation DC 18 pour déceler le piège magique qui s'y trouve et un
jet d'arcane DC 18 pour désactiver ce piège. Si un joueur tente d'ouvrir
un coffre lancer un d6 pour décider le type de piège qui risque de se 
déclencher et un dé 100 pour trouver le contenu. Les tables qui suivent 
permettent d'interpréter les résultats, si le dé 100 n'indique aucun
objet, cela veut tout simplement dire que le coffre est vide.

\begin{table} [h]
    \setlength{\tabcolsep}{4pt}
    \center
\begin{tabular}{cc}
  \bf d6 & \bf Type de piège \\
   \rowcolor{LightCyan}
  1 & Projectiles magiques (lancé au niveau 7) \\
  2 & Flétrissement (DD 18)\\
   \rowcolor{LightCyan}
  3 & Chaine d'éclairs (DD 18)\\
  4 & Boule de feu  (DD 18)\\
   \rowcolor{LightCyan}
  5 & Cone de glace (DD 18)\\
  6 & Pétrification (DD 18)\\
\end{tabular}
    \setlength{\tabcolsep}{6pt}
\end{table}

\subsection{Idées d'Aventures}

Hella achète toutes sortes de livres rares sur la magie et les artefacts.
Lorsqu'elle pense avoir localisé un objet elle envoie une équipe sur ses 
traces, elle peut ainsi embaucher une équipe d'aventurier pour ça. Elle 
emploie aussi parfois des gardes armés si elle dispose d'un artefact 
particulierement précieux et que sa boutique risque d'être attaquée ou pour 
escorter un acheteur transportant une large somme d'argent. 
Il est aussi possible que Yeron son ex-mari et compagnon d'aventure entre 
en contact avec les joueurs pour aller voler la boutique d'Hella. Selon lui,
Hella ne lui aurait jamais donné sa part, selon Hella elle a juste gardé une 
partie de ce qui revenait à Yeron pour mettre un toit sur la tête de leurs 
enfants.

