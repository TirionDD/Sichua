\section{Boutique Merveilleuse d'Hella RocheNoire}

Boutique d'objets magiques tenu par une femme gnome de pierre
(Il n'y a que des gnomes de pierre dans la demeurre.

En début de falaise dans les quartiers bourgeois vos trouverez cette
maison très bourgeoise mais pas très haute (2.20 de hauteur de plafond).
Vous serez certainement arrivée ici sur recommendation, la boutique 
d'Hella RocheNoire est la destination incontournable de ceux qui 
recherchent des objets magiques. Elle dispose d'un quasi-monopole sur
les objets magiques en ville, leur commerce est rare et dangereux car
on a vite fait de se faire arnaquer avec un objet dont le sort est 
seulement temporaire ou pire maudit. Hella RocheNoire sert donc 
d'intermédiaire dans toutes les principales transactions d'objets
magiques et est la seule commerçante de la ville à disposer d'un stock
d'objets magiques conséquent.

L'entrée de la maison ouvre sur plusieurs salles d'attentes. Après y avoir
été guidé par un servant en livré, on y découvre un mobilier luxueux et on 
y sert du thé et des biscuits aux visiteurs en attendant d'être introduit
auprès de la maîtresse de maison.

Le bureau de Hella Rochenoire est une salle très étendu, son fils Helim 
et sa fille Hydale ont des bureaux près de l'entrée alors que leur mère 
occupe le centre de la pièce. Les murs sont couvert de livres et d'anciens 
manuscrit et parchemins, un oeil averti remarquera que ce sont principalement 
des journaux d'aventurier, principalement des mages ou bardes. Lorsque Hella 
reçoit des clients, ses enfants restent là pour travailler mais se tiennent 
à disposition de leur mère pour jeter un oeil sur un objet ou aller chercher 
un article précieux dans les coffres sous la demeurre.

A droite du bureau, la chambre de Hella et sa salle d'étude personnelle
il y a une porte entre cette salle et le bureau, a gauche les chambres
de ses enfants Helim et Hydale ainsi que la cuisine où se trouve les deux
servants de la maison Renald et Krimmelle.

Dans l'étude d'Hella se trouve les escaliers pour descendre au sous-sol.
Seul un Rochenoire peut en ouvrir la porte, mais un jet d'investigation
peut permettre de détecter l'entrée. Il n'y a néanmoins aucun moyen 
d'ouvrir la porte de manière mécanique et toute tentative de forcer la 
porte, qui a une CA de 14 et 50 point de vie, voit l'auteur de l'attaque 
ciblé par un sortilège de petrification que son attaque ai réussit ou non. Il doit 
alors faire un jet de sauvegarde de constitution DD 18. En cas d'échec de 5 points ou 
plus, la créature est immédiatement pétrifiée. En cas de simple échec, 
elle commence à se changer en pierre et est entravée. Elle doit retenter 
le jet de sauvegarde à la fin de son prochain tour. En cas de succès, 
l'effet se termine. En cas d'échec, la créature est pétrifiée, jusqu'à 
ce qu'elle soit délivrée par un sort de restauration suprême ou par une 
magie similaire. L'entrée du passage 

Au sous sol se trouve une grande salle de la taille du bureau d'Hella, il 
va sans dire que le lieux est protégé contre tous types de téléportation
et toute magie par un puissant champ d'anti-magie. Une centaines de coffres 
de toutes tailles s'y trouvent, tous sceller dans la roche au sol.
La plupart son vide, seuls les RocheNoire savent où sont range les objets. 

Liste d'objet en vente à générer. Faire une table? ou utiliser la table
du DMG. Liste d'objet secondaire, elle sait que des gens en ville veulent 
les vendre.

Hella achete toutes sortes de livres rares sur la magie et les artefacts.
Lorsqu'elle pense avoir localisé un objet elle envoie une équipe sur ses 
traces, elle peut embaucher une équipe pour ça. Elle emploie aussi des
gardes armés si elle dispose d'un artefact particulierement précieux.

Hella connait la pluspart des objets magiques puissant disponible en ville.
Elle sert souvent d'intermédiaire pour des ventes ou échanges entre familles
riches ou aventuriers. Elle assure alors à la fois la sécurité et 
l'expertise des objets échangés. 

