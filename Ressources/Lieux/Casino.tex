\section{Aux Dés d'Argent}

C'est dans un petit immeuble à la façade sale mais finement décorée qu'est 
installé le cercle de jeu le plus réputé de la ville. Son aspect extérieur 
laisse une impression équivoque, en effet la bâtisse transpire la richesse et 
la finesse de ses anciens propriétaires, mais son état actuel semble largement 
négligé. Alors que la plupart des habitants du quartier entretiennent leur 
immeuble comme une façade de leur réputation, le cercle de jeu garde ses 
richesses dissimulées. L'ambiance intérieure tranche particulièrement avec 
l'aspect extérieur par sa décoration abondante et clinquante. Dans les salles 
de jeux, d'épais rideaux de velours rouge bloquent la lumière du jour, 
celle-ci est agréablement remplacée par un éclairage doux provenant de nombreux 
chandeliers en argent aux styles disparates.

\begin{figure}[b!]
\center
\fbox{%
  \begin{minipage}[c]{.95\linewidth}
    \label{AlastarGremet}
    {\bfseries\LARGE\scshape Alastar Gremet} \\
    Humain de taille M, loyal mauvais \\
    \noindent\rule{\textwidth}{1pt} \\
    {\bfseries Classe d'armure} 11 \\
    {\bfseries Points de vie} 17 \\
    {\bfseries Vitesse} 9 m \\
    \noindent\rule{\textwidth}{1pt} \vskip 2pt
    \setlength{\tabcolsep}{4pt}
      {\footnotesize 
    \begin{tabular}{cccccc}
      \bf FOR & \bf DEX & \bf CON & \bf INT & \bf SAG & \bf CHA \\
       8 (-1) & 12 (+1) & 10 (+0) & 16 (+3) &  14 (+2) &  8 (-1) \\
    \end{tabular} }
    \setlength{\tabcolsep}{6pt}
    \noindent\rule{\textwidth}{1pt} \\
    {\bfseries Sens} Perception passive 12 \\
    {\bfseries Langues} commun, comptable \\
    {\bfseries Facteur de puissance} 0 (0 XP)\\
    \noindent\rule{\textwidth}{1pt} \\
Homme sans personnalité, son esprit semble avoir été taillé pour manier les 
chiffres plutôt que les mots. Il est sans coeur et n'a aucun scrupule à 
étrangler un client avec une dette excessive qui le mènera à la banqueroute et 
la vente de ses biens. Mais n'attendez pas une pièce de cuivre de crédit si 
vous n'avez rien pour assurer cet emprunt. Alastar est aussi le comptable du 
cercle et le seul à vraiment comprendre l'état financier de l'entreprise qui 
détient de nombreux emprunts, hypothèques et même les dépôts de certains
joueurs chanceux ne souhaitant pas rentrer chez eux de nuit avec une grosse 
somme.
 \end{minipage}
}%
\end{figure}

\begin{figure*}[tb!]
\center
\fbox{%
  \begin{minipage}[c]{.45\linewidth}
    \label{MeribaldDalloris}
    {\bfseries\LARGE\scshape Meribald Dalloris} \\
    Humain de taille M, neutre \\
    \noindent\rule{\textwidth}{1pt} \\
    {\bfseries Classe d'armure} 14 (armure de cuir) \\
    {\bfseries Points de vie} 58 (13d8) \\
    {\bfseries Vitesse} 9 m \\
    \noindent\rule{\textwidth}{1pt} \vskip 2pt
    \setlength{\tabcolsep}{4pt}
      {\footnotesize 
    \begin{tabular}{cccccc}
      \bf FOR & \bf DEX & \bf CON & \bf INT & \bf SAG & \bf CHA \\
       8 (-1) & 16 (+3) & 10 (+0) & 17 (+3) &  12 (+1) & 14 (+2) \\
    \end{tabular} }
    \setlength{\tabcolsep}{6pt}
    \noindent\rule{\textwidth}{1pt} \\
    {\bfseries Jets de sauvegardes} Dex +7, Int +7\\
    {\bfseries Compétences} Discrétion +7, Escamotage +11, Intimidation +6, Intuition +9, 
                            Représentation +10, Tromperie +10\\
    {\bfseries Sens} Perception passive 14\\
    {\bfseries Langues} commun, elfique, jargon des voleurs\\
    {\bfseries Facteur de puissance} 8 (3900 XP)\\
    \noindent\rule{\textwidth}{1pt} \\
    {\bfseries Attaque sournoise (1/tour).} Meribald inflige 21 (6d6) dégâts supplémentaires 
               quand il touche une cible lors d'une attaque avec une arme et qu'il a 
               l'avantage au jet d'attaque, ou lorsque la cible est 1,50 mètre ou moins d'un 
               allié de Meribald qui n'est pas incapable d'agir et si Meribald n'a pas un 
               désavantage à son jet d'attaque.
\vspace{-10pt}
    \subsection*{Actions}
    {\bfseries Attaques multiples.} Meribald réalise deux attaques avec sa rapière.\\
    {\bfseries Rapière.} Corps à corps : +7 (1d8+3 perforant ; finesse)
  \end{minipage}
  \hspace{14pt}
  \begin{minipage}[c]{.45\linewidth}
Meribald provient des couches les plus populaires de la ville. Depuis son 
plus jeune âge il observe tous ses riches personnages et s'étonne que des 
imbéciles puissent être riche. Il a toujours pensé qu'il pourrait rétablir 
quelque peu la balance en exploitant leurs faiblesses pour tirer des sommes 
importantes de ces gens. Sa philosophie est simple, il respecte l'intelligence 
et le savoir-faire. Maintenant qu'il a accumulé une petite fortune, dont le 
joyaux est le bâtiment abritant le cercle de jeu, il met un point d'honneur à 
ne recruter que des jeunes des classes populaires dans lesquels il reconnaît 
un fort potentiel.

Meribald a commencé sa carrière comme un joueur de taverne, son charisme et sa 
prestance lui ont rapidement valu de monter en gamme et de passer des jeux de 
tavernes à une pièce de cuivre le point aux salons de la ville haute à dix 
pièces d'or par point. Évidemment sa réussite ne tient en rien au hasard et 
c'est un maître de la triche, mais aussi un excellent acteur qui a toujours su 
équilibrer ses jours de gains avec des pertes raisonnables pour ne jamais être 
sérieusement soupçonné par ses partenaires. Sa méthode de triche favorite est 
l'usage de dés pipés qu'il introduit subtilement à la table lorsque les mises 
commencent à atteindre des mises inconfortablement élevées.

A présent à la tête d'un établissement reconnu, d'une petite milice et d'un 
réseau de renseignement efficace. Tout laisse à penser que Meribald va se 
lancer en politique... 

 \end{minipage}
}%
\end{figure*}

\begin{figure*}[tb!]
\center
\fbox{%
  \begin{minipage}[c]{.45\linewidth}
    \label{GrimKramdac}
    {\bfseries\LARGE\scshape Grim Kramdac alias Brise-Os} \\
    Demi-orque de taille M, neutre \\
    \noindent\rule{\textwidth}{1pt} \\
    {\bfseries Classe d'armure} 16 (cuirasse) \\
    {\bfseries Points de vie} 112 (15d8 + 45) \\
    {\bfseries Vitesse} 9 m \\
    \noindent\rule{\textwidth}{1pt} \vskip 2pt
    \setlength{\tabcolsep}{4pt}
      {\footnotesize 
    \begin{tabular}{cccccc}
      \bf FOR & \bf DEX & \bf CON & \bf INT & \bf SAG & \bf CHA \\
       18 (+4) & 15 (+2) & 16 (+3) & 10 (+0) &  12 (+1) & 17 (+3) \\
    \end{tabular} }
    \setlength{\tabcolsep}{6pt}
    \noindent\rule{\textwidth}{1pt} \\
    {\bfseries Jets de sauvegarde} For +7, Dex +5, Con +6 \\
    {\bfseries Compétences} Athlétisme +7, Intimidation +6
    {\bfseries Sens} Vision dans le noir à 18 m, perception passive 11 \\
    {\bfseries Langues} commun, orque \\
    {\bfseries Facteur de puissance} 5 (1800 XP)\\
    \noindent\rule{\textwidth}{1pt} \\
    {\bfseries Bravoure.} Kramdac a l'avantage aux jets de sauvegarde pour ne pas être effrayé. \\
    {\bfseries Brute.} Une arme de corps à corps inflige un dé supplémentaire de ses dégâts si 
                       Kramdac touche avec celle-ci. \\
    {\bfseries Endurance tenace.} Lorsque Kramdac est réduit à 0 point de vie, mais pas tué sur 
               le coup, il peut remonter à 1 point de vie. Il doit terminer un repos long pour 
               pouvoir utiliser cette capacité de nouveau.
\vspace{-10pt}
    \subsection*{Actions}
    {\bfseries Attaques multiples.} Kramdac réalise trois attaques de corps à corps ou deux 
               attaques à distance.\\
    {\bfseries Épée longue.} Attaque d'arme de corps à corps : +7 au toucher, allonge 1,50 m, 
               une cible. Dégâts : 15 (2d10 + 4) dégâts tranchants.
  \end{minipage}
  \hspace{14pt}
  \begin{minipage}[c]{.45\linewidth}
    {\bfseries Arbalète lourde.} Attaque d'arme à distance : +5 au toucher, portée 30/120 m, 
               une cible. Dégâts : 7 (1d10+2) dégâts perforants.
\vspace{-10pt}
    \subsection*{Réactions}
    {\bfseries Parade.} Kramdac ajoute 3 à sa CA contre une attaque de corps à corps 
               qui le toucherait. Pour ce faire, il doit voir l'attaquant et avoir en main 
               une arme de corps à corps. \\
    \noindent\rule{\textwidth}{1pt} \\
Grim est un vétéran de l'armée qui a mené de nombreuses expéditions contre les 
tribus d'orques et de gobelins environnantes. Sa qualité principale est qu'il 
ne pose pas de question et se contente d'appliquer les ordres. Il a une seconde 
qualité, sa largeur d'épaule impressionnante qui a un effet calmant sur les 
fauteurs de troubles. Les débiteurs du cercle savent néanmoins que c'est 
lorsqu'il frappe qu'il est le plus impressionnant, on ne s'en relève 
généralement pas...

Grim pourrait passer pour une sombre brute, mais il est en réalité plus 
intelligent qu'il n'y parait et ce n'est pas par hasard que Meribald l'a 
recruté. Il sait jongler entre les menaces physiques et d'autres ordres avec 
talent. D'un petit lieutenant d'une tour de reconnaissance lointaine, il est 
devenu une figure reconnue et respectée des hauts quartiers, il est largement 
reconnaissant de cette ascension sociale envers Meribald. Il lui reconnaît 
aussi une intelligence supérieure et lui est fidèle par intérêt, il sait que 
ceux qui doublent Meribald finissent toujours dans le fossé. Il sait aussi que 
quel que soit le gain matériel qu'il pourrait obtenir en trahissant son mentor, 
il le payerait au long terme. Inutile donc d'essayer de le corrompre.

 \end{minipage}
}%
\end{figure*}



Aux rez-de-chaussée se trouve la réception avec un imposant escalier menant 
aux premier et second étages. Les armes étant interdites dans l'établissement, 
c'est ici que les visiteurs devront les laisser dans un vestiaire à cet effet. 
De part et d'autre de la salle de réception se trouve les deux principales 
salles de jeu ouvertes aux bourses les plus modestes. Les clients plus fortunés 
se rendent directement au premier étage où ils pourront s'installer 
confortablement pour jouer tout en consommant un vin ou une liqueur de qualité 
supérieure. Alors que l'odeur d'herbe à pipe bas de gamme peut être 
incommodante au rez-de-chaussée, le mélange d'encens et d'épices luxueuses du 
premier étage est très agréable. Cet étage est celui où la bonne société veut 
être vue jouant des sommes importantes, mais pas excessives, en appréciant des 
produits exotiques et raffinés. Les joueurs qui se rendent au second étage sont 
ceux qui préfèrent rester discret, soit qu'ils jouent des sommes astronomiques 
soit qu'ils agrémentent leurs parties de drogues ou d'une compagnie moins 
socialement acceptable...

Au niveau organisationnel, une cuisine est installée sous l'escalier principal, 
elle entretient un buffet raffiné disponible au premier étage et sert sur 
commande dans toute la demeure. Pour la sécurité de l'établissement des gardes 
circulent dans toutes les salles et transportent les fonds recueillis aux 
tables vers le second étage dans le bureau d'Alastar Gremet, le banquier et 
comptable de l'établissement. Le second étage accueil aussi les bureaux des 
autres dirigeants du cercle : Grim Kardak en charge du recouvrement des dettes 
et bien sûr Meribald Dalloris le propriétaire du cercle. C'est dans son bureau 
que se trouve le coffre-fort du cercle. Celui-ci prend la forme d'une porte 
ouvrant sur un plan de poche, la porte est néanmoins ordinaire et n'ouvre que 
sur le mur si on ne visualise pas la clef du coffre, bien sûr seul Meribald 
sait quelle est cette pensée clé : le sceptre du Roi-Dieu. Finalement, sous 
les toits un troisième étage se trouve un certain nombre de chambres où dorment 
une partie des employés.

\begin{figure}[p]
\center
\fbox{%
  \begin{minipage}[c]{.95\linewidth}
    \label{MainLegere}
    {\bfseries\LARGE\scshape Elune et Solane MainLégère} \\
    Halfelines de taille P, chaotiques bonnes \\
    \noindent\rule{\textwidth}{1pt} \\
    {\bfseries Classe d'armure} 12 \\
    {\bfseries Points de vie} 27 (6d8) \\
    {\bfseries Vitesse} 7.5 m \\
    \noindent\rule{\textwidth}{1pt} \vskip 2pt
    \setlength{\tabcolsep}{4pt}
      {\footnotesize 
    \begin{tabular}{cccccc}
      \bf FOR & \bf DEX & \bf CON & \bf INT & \bf SAG & \bf CHA \\
       8 (-1) & 15 (+2) & 10 (+0) & 12 (+1) &  17 (+3) & 13 (+1) \\
    \end{tabular} }
    \setlength{\tabcolsep}{6pt}
    \noindent\rule{\textwidth}{1pt} \\
    {\bfseries Compétences} Discrétion +4, Escamotage +4, Intuition +5, Investigation +5, Perception +8 \\
    {\bfseries Sens} Perception passive 18 \\
    {\bfseries Langues} commun, langage des signes \\
    {\bfseries Facteur de puissance} 1 (200 XP)
\vspace{-10pt}
    \subsection*{Actions}
    {\bfseries Attaques multiples.} Les s\oe{}urs réalisent chacune deux attaques de corps à corps.\\
    {\bfseries Épée courte. }Attaque d'arme de corps à corps : +4 au toucher, allonge 1,50 m, une cible. 
               Dégâts : 5 (1d6 + 2) dégâts perforant.\\
    {\bfseries Arbalète de poing.} Attaque d'arme à distance : +4 au toucher, portée 9/36 m, une cible. 
               Dégâts : 5 (1d6 + 2) dégâts perforants.\\
    \noindent\rule{\textwidth}{1pt} \\
Elune et Solane MainLégère ont appris la triche au côté de Meribald dans les 
bas quartiers. Celui-ci a vite reconnu leur talent et s'est empressé de les 
faire travailler de l'autre coté des miroirs sans teint des plafonds de ses 
salles de jeux où seul des halfelins sans trop d'embonpoint peuvent se glisser. 
Expertes en triche, elles connaissent tous les coups tordus et signalent tout 
tricheur à Grim.  Ceux-ci sont menés rapidement au sous-sol, la première fois 
pour des remontrances musclées la seconde pour se faire passer à tabac 
allègrement. Il arrive aussi que Meribald mette à l'amande ces joueurs et leur 
fassent payer une pénalité financière, manière de prendre sa part sur toute 
triche se déroulant sous son toit.

L'architecture des pièces est telle que le son se transmet particulièrement 
bien vers les conduits ou se glissent les deux halfelines et celle-ci en 
profitent pour prendre des notes sur les confidences des clients. Toutes ses 
informations sont consignées par Meribald dans des dossiers stockés dans son 
coffre fort. On peut raisonnablement penser que ce dossier est en fait la chose 
la plus précieuse de ce coffre... Néanmoins, Elune et Solane qui s'amusent 
beaucoup de ce travail ne se préoccupent pas vraiment des grands desseins 
de leur employeur. 
 \end{minipage}
}%
\end{figure}

\begin{figure}[p]
\center
\fbox{%
  \begin{minipage}[c]{.95\linewidth}
    \label{MainLegere}
    {\bfseries\LARGE\scshape Eldrik de Mestarin} \\
    Humain de taille S, Loyal neutre \\
    \noindent\rule{\textwidth}{1pt} \\
    {\bfseries Classe d'armure} 13 \\
    {\bfseries Points de vie} 9 (2d8) \\
    {\bfseries Vitesse} 9 m \\
    \noindent\rule{\textwidth}{1pt} \vskip 2pt
    \setlength{\tabcolsep}{4pt}
      {\footnotesize 
    \begin{tabular}{cccccc}
      \bf FOR & \bf DEX & \bf CON & \bf INT & \bf SAG & \bf CHA \\
       11 (0) & 16 (+3) & 10 (0) & 12 (+1) &  14 (+2) &  16 (+3) \\
    \end{tabular} }
    \setlength{\tabcolsep}{6pt}
    \noindent\rule{\textwidth}{1pt} \\
    {\bfseries Compétences} Escamotage +8, Tromperie +5\\
    {\bfseries Sens} Perception passive 12 \\
    {\bfseries Langues} commun, elfique \\
    {\bfseries Facteur de puissance} 1/8 (25 XP)
\vspace{-10pt}
\subsection*{Actions}
    {\bfseries Rapière.} Attaque d'arme de corps à corps : +5 au toucher, allonge 1,50 m, une cible. 
                         Dégâts : 5 (1d8 + 3) dégâts perforants.
\vspace{-10pt}
\subsection*{Réactions}
    {\bfseries Parade.} Eldrik ajoute 2 à sa CA contre une attaque de corps à corps qui le toucherait. 
                        Pour ce faire, il doit voir l'attaquant et avoir en main une arme de corps à corps.\\
    \noindent\rule{\textwidth}{1pt} \\
Eldrik de Mestarin est un petit nobliau de province venu à Laelith pour briller 
en société et gagner sa vie plus richement que dans son petit domaine. 
Malheureusement, tout ne s'est pas passé comme prévu et le jeune homme a 
rapidement laissé les filles de petite vertu et le cercle de jeu le dépouiller 
de ses richesses. C'est lorsque tous ses biens furent hypothéqués qu'il se mit 
à tricher pour essayer de s'en sortir. Le fait est que s'il avait peu de 
jugeote pour l'argent, il est un plutôt bon tricheur, au point qu'Elune et 
Solane ont mis plusieurs jours à le repérer. Mais Meribald s'est immédiatement 
renseigné auprès d'Alastar, cela faisait apparemment plusieurs jour qu'Eldrik 
gagnait fréquemment...

Meribald fit une proposition qu'on ne peut refuser à Eldrik, ses halfelines 
fermeraient les yeux sur sa triche en échange de la moitié de ses gains. Une 
affaire pour Eldric à première vue, néanmoins lorsqu'on sait que la seconde 
moitié des gains revient aussi à Meribald en payement des intérêts de dettes 
passées... On comprend que la situation d'Eldrik n'est pas enviable. Mais c'est 
cela ou Meribald fera jouer ses hypothèques sur le domaine d'Eldrik. Pour 
Meribald, par contre c'est une manière de faire payer un supplément discret à 
ses joueurs sans attirer l'attention de qui que ce soit. 
 \end{minipage}
}%
\end{figure}

Une partie plus méconnue de la bâtisse est son sous-sol, il est accessible par 
un petit escalier discret. Celui-ci débouche dans la vaste salle de repos des 
employés, où on trouve aussi leurs vestiaires. Deux portes se trouvent dans 
cette pièce, si celle de gauche mène dans un lieu de stockage sans grand 
intérêt, celle de droite mène dans une salle plus surprenante. Là, plusieurs 
estrades sont disposés en cercle autour d'une fosse, c'est ici qu'ont lieux les 
combats de chien hebdomadaire. Officiellement ceux-ci sont interdit en ville, 
mais les habitués savent bien que le jour de fermeture du cercle n'est qu'une 
excuse pour les organiser discrètement.  

La maison se concentre le reste du temps sur des jeux plus raisonnables et ce 
sont les jeux de cartes et de dés qui dominent les salons de jeux. Ceux-ci 
opposent souvent les joueurs directement sans impliquer le cercle qui assure 
alors ses revenus grâce aux services qu'il offre: restauration, boissons, mais 
surtout emprunts. L'activité la plus rentable est en effet l'usure ! Les 
joueurs se trouvent souvent à cours de finances, mais refusent de quitter leur 
table, ils se font alors amener des fonds. Le taux d'intérêt varie de 1 à 5% 
par jour, selon la solvabilité du client, le taux étant décrété par Alastar. 
Cette activité bancaire s'est largement développée dans les années passées au 
point qu'un certain nombre de clients ne viennent que pour faire un emprunt et 
ne jouent même pas.
 
Évidemment ce type d'établissement nécessite une sécurité de qualité. Celle-ci 
démarre dans la rue devant le cercle où des gardes patrouillent régulièrement, 
car il n'est pas question qu'un client se fasse dévaliser devant la porte. 
Toutes les pièces où se trouvent des joueurs ou des valeurs sont aussi gardées. 
Enfin la salle de réception et les escaliers sont surveillés en permanence pour 
éviter qu'un joueur ne se rende dans un étage où il n'est pas le bienvenue. La 
sécurité est essentiellement assurée par des vétérans de la garde. Tout ceci 
n'est néanmoins que la partie émergée de l'iceberg, car durant la journée, 
l'activité principale de ces hommes est le recouvrement de fonds ! Ces gros 
bras se rendent alors chez les mauvais payeurs pour les convaincre qu'il est 
dans leur intérêt de rembourser rapidement... Cette partie du travail fait 
aussi intervenir une autre équipe de sécurité, beaucoup plus discrète celle-ci. 
En effet, deux halfelines à l'oeil aiguisé surveillent qu'il n'y a pas de 
tricheurs aux tables durant les parties. Leur rôle dans le recouvrement de 
fonds est généralement lié à leurs oreilles qui traînent et si vous parlez à 
l'un de vos amis de cette maîtresse que vous entretenez à l'insu de votre 
femme, sachez qu'Aux Dés d'Argent les murs ont des oreilles et que cette 
information a été immédiatement consignée. Ainsi, le temps venu un seul employé 
de la maison avec un simple message saura vous faire payer bien plus vite 
qu'une poignée de gros bras. Cette méthode douce est d'autant plus utile 
lorsque le mauvais payeur est un personnage important qu'on ne peut pas 
simplement molester dans une ruelle.

\subsection*{Le jeu du dé d'or:}
Jeu de dés traditionnel, c'est le plus joué des tavernes crapuleuses aux salons 
les plus huppés. Le jeu se joue avec trois dés, deux argentés et un doré. Les 
joueurs (de 4 à 6 en général) placent chacun une mise au centre de la table au 
départ puis lancent les dés à tour de rôle. Le but du jeu est d'obtenir un 
score plus important sur le dé doré que sur les deux dés argentés combinés, le 
premier à réussir remporte toutes les mises au centre de la table et la partie 
se termine. Il y a néanmoins un prix pour jouer, car à chaque lancer de dés si 
la somme des dés d'argent est plus faible que celle obtenue par le joueur 
précédent, vous devez ajouter des mises au centre de la table. Initialement, il 
ne faut placer qu'une mise supplémentaire au centre, mais ce prix pour jouer 
augmente d'une mise à chaque fois que cela arrive. Il se peut donc que des 
joueurs ne puissent plus payer les mises et soient obligés de quitter la 
partie. Il existe un cas spécial, lorsque la somme des dés d'argent est 
exactement celle du dé doré, chaque joueur donnent une mise au malheureux qui a 
loupé la cagnotte de si peu et ce lancer ne nécessite jamais de contribuer au 
pot même s'il est plus faible que le précédent. Notez qu'il n'est pas 
raisonnable de se lancer dans une partie sans au moins une cinquantaine de 
mises initiales disponibles.

