\section{Guilde des Alchimistes}

\subsection{Déscription}

Les maisons des différents alchimistes de la ville forment un ensemble compact 
et assez chaotique particulièrement original dans ce quartier huppé. Les 
longues cheminées d'où s'échappent des fumées multicolores révèlent 
immédiatement l'usage de ces bâtiments. Au rez-de-chaussée, les échoppes 
donnent pour la plupart sur la promenade des brouillards ou à défaut dans de 
petites venelles latérales. Au centre du complexe, un édifice plus cossu 
abrite le siège de la guilde en elle-même. Un vaste salon fort luxueux permet 
d'y recevoir les clients et fournisseurs les plus fortunés pour y négocier 
d'importants contrats autour d'un thé ou d'un vin aux origines lointaines. 
Au-dessus des échoppes se trouvent les bureaux et les habitations des 
alchimistes et de leurs familles, les laboratoires eux se trouvent généralement 
en retrait de la promenade au-dessus de hangars où sont installées de petites 
verreries artisanales.

Le travail d'alchimiste nécessite de se fournir en produits rares et exotiques
en provenance de toutes les parties du monde connu. Ce sont les dirigeants de
la guilde, dont les bureaux sont localisés au-dessus du grand salon, qui 
mènent les négociations avec les grands marchands de la ville pour acheter ces 
produits rares et ensuite les répartir entre les alchimistes. Cette dernière 
tâche, d'une importance capitale, est la raison pour laquelle les sièges 
héréditaires du conseil de la guilde se négocient à plusieurs centaines de
millier de pièces d'or. À l'autre bout de l'échelle sociale de la guilde on trouve 
les enfants coursiers qui forment un maillon essentiel de la guilde, les 
potions les plus vendues par la guilde étant les potions de guérison ou de 
soins, la plupart des clients ne sont pas en état de venir chercher leurs 
préparations et se les font donc livrer. Ces jeunes coursiers proviennent 
généralement des quartiers pauvres et doivent payer une redevance pour être 
autorisés à transporter les potions, en contre partie, ils touchent des primes
importantes à chaque ivraisons et élisent chaque année 
un représentant au conseil de la guilde.

La guilde ne se limite pas à la surface, les sous sols accueillent un certain
nombre de laboratoires et de lieux de stockage pour les matières les plus 
sensibles. En particulier, la grande salle des coffres sous le salon d'honneur 
de la guilde est hautement sécurisée et il y patrouille jour et nuit deux 
golems de fer. Cette grande salle voûtée, toujours parcourue par un courant
d'air dispersant les émanations toxiques ou corrosives, abrite le long de ses
murs les coffres de la guilde et des alchimistes les plus fortunés. On dit que 
cette salle est protégée contre tous types d'intrusion magique. Les trésors qui 
y sont stockés sont l'objet de nombreux bruits et fantasmes en ville, mais la 
réalité les surpasses tant on peut y trouver de potions légendaires dont la 
puissance n'est égalée que par leur prix. Les alchimistes y stockent aussi les 
matériaux les plus rares utilisés pour la fabrication de ces mêmes potions. Si 
l'on en crois les rumeurs cela inclurait du sang de dragons anciens, des plumes 
de solaire ou de la poudre d'os de liche... Un tel lieu pourrait légitimement 
attirer les convoitises et certains bruits dans les bas fonds indiquent que 
la guilde des alchimistes en a négocié l'immunité avec les guildes de voleurs. 
L'accord stipulerait que les guildes de voleurs punissent de mort toutes 
tentatives ou même projet de cambriolage du lieu, en échange de quoi les 
alchimistes fourniraient régulièrement des poisons rares à ces guildes. Un 
échange de bons procédés moralement discutable dont les instances dirigeantes 
de la guilde ont toujours nié l'existence.

\subsection{Quelques membres du conseil}

\label{AquilaPrimus}
{\bf Aquila Primus} est l'actuel Grand Alchimiste et maître de l'ordre. Sa famille
détient les recettes les plus complexes et les secrets pour extraire l'essence
des produits les plus rares. Les potions de sa fabrication sont d'une puissance
incomparable et on vient de très loin pour lui acheter ses produits. La rareté
de sa production lui laisse tout le temps nécessaire pour s'occuper de la 
gestion de la guilde et rencontrer ses partenaires commerciaux.

\label{TiberiusHondar}
{\bf Tiberius Hondar} est l'un des producteurs de potions de soins les plus réputés. 
En particulier ses onguents de régénérations qui ont permis à tant de riches 
clients de retrouver un doigt, un nez, voir parfois une jambe entière. Ce 
miracle a néanmoins un coût : plusieurs milliers de pièces d'or par semaine de 
traitement.

\label{KrassiusEldark}
{\bf Krassius Eldark} est un alchimiste atypique, car la spécialité de sa famille 
depuis de nombreuses générations est la fabrication de golems. Lorsque vous 
verrez l'un de ces petits golems de bois faire le service dans un hôtel 
particulier, sachez qu'il en est l'unique fabriquant et que ces petits bijoux 
valent une quinzaine de milliers de pièces d'or. La construction de golems de 
guerre, tel ceux de la salle aux trésors de la guilde, est néanmoins une toute 
autre affaire. Ceux-ci ne sont disponibles que sur commande plusieurs mois à 
l'avance et selon la disponibilité de matériaux de construction rares. Enfin, 
il vous faudra aussi débourser plusieurs centaines de milliers de pièces 
d'or... Vous comprenez à présent pourquoi ils ne courent pas les rues.

\begin{table} [h]
    \setlength{\tabcolsep}{4pt}
    \center
\begin{tabular}{lcc}
  \bf Nom & \bf Prix & \bf Page\\
   \rowcolor{LightCyan}
  Armure animée (FP 1) & 8 000 po & \pageref{ArmureAnimee}\\
  Golem de bois (FP 2) & 15 000 po & \pageref{GolemBois}\\
   \rowcolor{LightCyan}
  Golem de chair* (FP 5) & 50 000 po & \pageref{GolemChair}\\
  Gardien animé (FP 7) & 80 000 po & \pageref{GardienAnime}\\
   \rowcolor{LightCyan}
  Golem d'argile (FP 9) & 125 000 po & \pageref{GolemArgile}\\
  Golem de pierre (FP 10) & 150 000 po & \pageref{GolemPierre}\\
   \rowcolor{LightCyan}
  Golem de bronze (FP 13) & 250 000 po & \pageref{GolemBronze}\\
  Golem de fer (FP 16) & 400 000 po & \pageref{GolemFer}\\
\end{tabular}
\caption{\small * proposé uniquement aux clients de confiance...}
    \setlength{\tabcolsep}{6pt}
\end{table}

\label{KaesoHartar}
{\bf Kaeso Hartar} est le spécialiste incontesté des poisons et potions aux
objectifs les plus discutables. Son rôle dans la fourniture des guildes
d'assassins en fait un membre incontournable de la guilde. Il est néanmoins
très discret et son laboratoire est probablement la pièce la plus profonde
du complexe de la guilde des alchimistes. Sa specialité serait un poison à base 
de sang de Diantrefosse qui, selon la rumeure, enverrait sa victime directement 
en enfer pour une éternité de supplices...


\begin{table}[t!]
    \setlength{\tabcolsep}{4pt}
\begin{tabular}{lc}
  \bf Nom & \bf Prix \\
   \rowcolor{LightCyan}
  Potion de soins & 50 po \\
  Potion de poison** & 100 po \\
   \rowcolor{LightCyan}
  Potion de soins améliorés & 150 po \\
  Onguent de régénération & 150 po \\
   \rowcolor{LightCyan}
  Potion d'amitié animale & 200 po \\
  Potion d'escalade & 200 po \\
   \rowcolor{LightCyan}
  Potion d'héroïsme & 200 po \\
  Potion d'invisibilité & 200 po \\
   \rowcolor{LightCyan}
  Potion de lecture des pensées & 200 po \\
  Potion de respiration aquatique & 200 po \\
   \rowcolor{LightCyan}
  Potion de diminution & 250 po \\
  Potion de croissance & 250 po \\
   \rowcolor{LightCyan}
  Potion d'étât gazeux & 300 po \\
  Potion de résistance & 300 po \\
   \rowcolor{LightCyan}
  Potion de vitesse & 400 po \\
  Potion de force de géant des collines & 400 po \\
   \rowcolor{LightCyan}
  Huile d'insaisissabilité & 450 po \\
  Potion de soins supérieurs & 450 po \\
   \rowcolor{LightCyan}
  Philtre d'amour & 500 po \\
  Potion de vol & 500 po \\
   \rowcolor{LightCyan}
  Potion de clairvoyance & 1000 po \\
  Potion de force de géant du froid & 1000 po \\
   \rowcolor{LightCyan}
  Potion de vitalité & 1000 po \\
  Potion de soins suprêmes & 1350 po \\
   \rowcolor{LightCyan}
  Potion de force de géant du feu & 1600 po \\
  Huile éthérée* & 2000 po \\
   \rowcolor{LightCyan}
  Potion de souffle draconique* & 2500 po \\
  Huile d’affûtage* & 3000 po \\
   \rowcolor{LightCyan}
  Potion de force de géant des nuages* & 3000 po \\
  Potion d'immunité* & 3000 po \\
   \rowcolor{LightCyan}
  Potion d'invulnérabilité* & 4000 po \\
  Potion de force de géant des tempêtes* & 5000 po \\
   \rowcolor{LightCyan}
  Poison des enfers** & 6000 po \\
  Potion de soins complets* & 7000 po \\
   \rowcolor{LightCyan}
  Potion de longévité* & 9000 po \\
\end{tabular}
\caption{Description des potions page \pageref{potions} \newline \small
* potions disponible uniquement chez Aquila Primus.\newline
** potion disponible uniquement chez Kaeso Hartar}
    \setlength{\tabcolsep}{6pt}
\end{table}

\subsection{Accroche d'aventure}

Vos aventuriers peuvent se rendre à la guilde pour différentes raisons. La plus
commune est pour l'achat de potions. Dans ce cas, la taille et l'opulence de la 
boutique à laquelle ils s'adresseront dépend de la valeur des potions qu'ils 
achètent. Les membres du conseil de la guilde reçoivent généralement leurs clients 
réguliers dans le salon d'honneur. Il est aussi possible de revendre un certain 
nombre de plantes ou d'organes de créatures à la guilde. Néanmoins, un jet 
d'arcane DD 15 est nécessaire pour savoir quelles sont les parties intéressantes 
d'un monstre (s'il y en a) et un jet de nature DD 15 permet de les recueillir 
sans les altérer. Les alchimistes sont ouverts au troc et sont toujours à la 
recherche de matières premières, en cas de pénuries il peut même arriver qu'ils 
passent des commandes directement aux aventuriers de passage. Il est aussi 
possible que des aventuriers soient envoyés pour enquêter sur certaines 
activités de la guilde. En particulier, il existe des rumeurs indiquant que 
Krassius utiliserait des âmes humaines pour la construction de ses golems. 
Évidemment, toute enquête sur un empoisonnement mènera à Kaeso qui peut en tant 
qu'expert aider à trouver la nature d'un poison contre menu payement. Il 
ne donnera néanmoins jamais d'indication sur ses clients et ne garde d'ailleurs 
aucune note à leur sujet. On pourra seulement trouver des indications sur ses 
ventes récentes dans ses livres de comptes, mais il ne partagera jamais ces 
informations volontairement. 


