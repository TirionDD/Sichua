\noindent {\bfseries\large\scshape Épée d'Eril}\\
{\small \it Épée longue, rare (nécessite un lien par
un paladin)}\\
\label{EpeeEril}

Vous avez un bonus aux jets d'attaque et de dégâts effectués avec cette 
épée longue. Le bonus est initiallement de +1, il passe à +2 lorsque deux
capacités secrètes sont débloquées et à +3 lorsque les cinq sont débloquées.
Dans cette dernière configuration, les célestes et 
les fiélons que vous rencontrez vous reconnaissent comme un Élu d'Eril.

\paragraph{Capacités Secrètes:} ~ \\

Si vous réussissez un jet de sauvegarde contre un charme par une marge
de 5 ou plus lorsque vous portez l'épée d'Eril, celle-ci vous octroie
un bonus de +1 à la CA et aux jets de sauvegarde.

Si vous ne tombez pas inconscient après une attaque vous ayant infligée
50 point de dégats d'un même type lorsque vous portez l'épée d'Eril,
celle ci vous octroie à présent la résistance à ce types de dégâts. Cette 
propriété ne s'active qu'une seule fois et seulement si le personnage ne
dispose pas déjà d'une résistance à ce type de dégât. 

Si vous tuez un fiélon avec l'épée d'Eril, celle-ci vous octroie
la résistance au poison et l'avantage aux jets de sauvegarde pour leur
résister.

Si vous tuez un mort-vivant avec l'épée d'Eril, celle-ci vous octroie
a présent la résistance aux dégâts nécrotiques.

Si vous tuez une créature avec un coup critique porté avec l'épée d'Eril,
vous gagnez un bonus aux dégâts spécial pour les coups critiques. Lorsque 
vous portez un coup critique
vous pouvez doubler votre bonus de caractéristique pour le calcul des dégâts.

Si vous achevez une créature légendaire avec l'épée d'Eril, vous gagnez
une action légendaire. Vous pouvez utiliser votre réaction pour benéficier
d'une action action supplémentaire à la fin du tour de n'importe quel autre 
joueur ou monstre. Une fois cette capacité 
utilisée, vous devez terminer un repos court ou long pour pouvoir l'utiliser 
de nouveau. \\
