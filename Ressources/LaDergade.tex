\subsection{La Dergade}

\subsubsection*{L'organisation}

Réseau de contrebandier fournisseur de produits de contrebande en tout genre
en provenance des tréfonds et de certains plans peu fréquentables. Ils ont 
un monopole dans le duché de Sichua sur la pluspart des drogues disponibles. 
Ils disposent ainsi de champignons, spores, algues, mousses, venins et autre 
produits exotiques à l'origine douteuse possédant divers types de vertus 
euphorisante, halucinatoire, anesthésiante ou même calmante, mais très 
souvent aussi fortement addictive. On les soupçonne d'activité encore plus 
sombre telle que l'esclavagisme ou encore le commerce de produits alchimiques 
utilisé pour des invocations démonistes et autres rituels noirs. Moins connue
du grand publique est leur activité légale servant de couverture discrète aux
autres activités. En effet, un certain nombre de potions et sortilèges 
nécessitent des produits qui ne peuvent être trouvé que dans les tréfonds,
leur commerce est l'activité officielle des caravanes de la Dergade 
lorsqu'elles traversent les territoires.

Les accès aux tréfonds sont rares et le plus souvent secret, la Dergade 
utilise un passage à l'ouest des marais de la Lune connu d'eux seuls. Les
hommes de la Dergade connaissant le lieux sont tous sous l'effet d'un sort de
Mission leur interdisant de révéler l'emplacement du passage sous peine de
subir 5d10 dégâts psychiques. L'entrée vers les tréfonds se trouve dans les
ruines d'un ancien fort drow dont l'entrée discrète se trouve dans une grotte
au flan d'une colline perdue dans les steppes. La Dergade entretient une petite
troupe dans ce petit fort particulièrement discret. Néanmoins, ces collines
sont aussi sur le territoire d'un royaume d'orques particulièrement belliqueux.
Par chance, ou par design dirons les mieux renseignés, le roi des orques et
les grands prètres le conseillant sont accros aux spores de myconnides qu'ils
utilisent pour se ``connecter'' à leur divinité. Cet état de fait permet à la
Dergade de traverser les territoires orques en echange d'un petit droit de
passage...

Le comptoir de Kerlarflune dans les trefond est la destination des caravanes 
de la Dergade, il leur faut environ cinq jours pour le rejoindre depuis la 
surface. Ce comptoir tenu par des elfes noirs sert de point de rencontre 
entre les différentes communautés des trefonds, principalement pour le 
commerce et parfois pour des rencontres diplomatiques. Kerlarflune est tenu
par le Clan Deboer, un clan indépendant des grandes cités elfes noirs des 
tréfonds. Leur relative faiblesse par rapport aux grandes puissances les 
entourant fait d'eux un groupe insignifiant mais fort utile que toutes les
races des trefonds civilisées respectent, ou tout du moins tolèrent. Les 
membres de la Dergade sont suffisamment inelligent pour ne pas s'aventurer 
plus profondément dans les tréfonds.

\subsubsection*{Le Quartier Général}

Le QG de la Dergade est un hangar dans le quartier commerçant de Mevyr dans
le compté d'Herinar. Le hangar peut sembler modeste avec son exterieur 
discret et ses étales couvertent de produits que seul les érudits peuvent
reconnaitre, néanmoins il existe une arrière boutique... Celle ci n'est
par contre pas aisé à trouver, car quiconque ouvrant la porte arrière arrive
dans un bureau sombre et crasseux ou se trouve Karduc, l'homme lézard guerrier
officiellement propriétaire d lieu mais réellement seulement en charge de sa 
sécurité. La veritable arrière boutique est accéssible par la même porte,
du bureau vers le hangar, mais seulement si l'on prononce le mot de passe
"Tigre et Serpents" sans quoi la porte fonctionne de manière normale.  

L'arrière boutique accessible par cette porte secrète donne accès à un manoir
somptueux de Mordenkainen permanent. À l'interieur se trouve les reserves
de produits illégaux de la Dergade et surtout les appartements des deux
véritables dirigeants de l'organisation: Rénard et Serla, un Rakshasa et une
méduse. Ceux-ci fondent un couple fusionnel engagé dans le crime autant pour
s'amuser et l'adrénaline que pour l'argent. En cas de soucis, ils seront pret
à se sacrifier l'un pour l'autre, mais ne se préoccuperont pas le moins du 
monde pour un de leur subordonné. Si l'un deux est tué d'une manière ou 
d'une autre, le second mettra sa vie et la Dergade au seul service d'une 
future vengence. 


\subsubsection*{Idée d'Aventure}

Si les aventuriers obtiennent une réputation de mercenaires discrets, la
Dergade pourrait bien leur confier une mission. Aucune nouvelle n'a été
reçue de la dernière caravane envoyée à Kerlarflune, Rénard et Serla 
décident donc d'envoyer les aventuriers avec la caravane suivante pour 
assurer la sécurité du convois. Cette aventure peut aussi être initié 
par les joueurs eux même, si ils recherchent un accès aux tréfonds.

Le groupe se trouve attaqué par les orques sur le chemin qui semblent 
determinés contre toute intrusion dans leur 
territoire. Ils trouveront ensuite à l'entree des trefonds les corps des 
gardiens de la Dergade, ceux-ci sont eparpillés dans plusieurs pièces et 
portent tous des blessures profondes infligé par une créature certainement 
massive et puissante. Ce sont en fait des ombres des roches qui ont fuit
et que le groupe peut chasser en surface, maintenant ou plus tard. 

Une fois arrivé à Kerfarflune, le groupe trouve un comptoire en partie détruit 
et dont il rèste que quelques groupes cloîtré dans des grottes. La caravane
peut y vendre sa marchandise (essenciellement de la nourriture de la surface
(miel, épices, fromages, vins etc.) à prix d'or, les survivants étant pret 
à payer très chère pour quelques fournitures. Par contre, ceux-ci ne dispose 
pas de grand chose à part de l'or, en particulier très peu de produits 
stupefiant. Il retrouve sur place un ou deux membres de la precédente 
caravane qui leur explique que des monstres défèrlent regulièrement depuis
pusieurs semaines.

Si les aventuriers explorent plus en avant, ils découvriront à l'origine 
de ces créatures, le nid d'un groupe de dévoreurs mentaux. Ceux-ci semblent 
avoir été décimés et leur esclaves se sont echappé dans les tréfonds créant 
tout ces troubles. Une enquête approfondie indiquera que le nid à eté détruit
par un ver pourpre, un expert ou un survivant permettra aux aventuriers de
comprendra que l'une des larves de dévoreur s'est échappé de leur couveuse.
Elle est finalement revenu après avoir grandi dans des proportions démesuré
et a détruit le nid de sa naissance. Il est bien sûr possible de se lancer 
à la poursuite de la bête, aux risques et périls des aventuriers. Néanmoins 
la découverte precédente indique clairement que les attaques devraient 
rapidement se rarefier, les dévoreurs n'ayant eu qu'un nombre limité 
d'esclaves à l'origine. 

