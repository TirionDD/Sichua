\begin{figure*}[hb!]
\fbox{%
  \begin{minipage}[c]{.45\linewidth}
    \label{Eclaireur}
    {\bfseries\LARGE\scshape Éclaireur}\\
    {\small (System Reference Document trad. Aidedd.org)}\\
    Humanoïde (toute race) de taille M, tout alignement \\
    \noindent\rule{\textwidth}{1pt} \\
    {\bfseries Classe d'armure} 13 (armure de cuir) \\
    {\bfseries Points de vie} 16 (3d8 + 3) \\
    {\bfseries Vitesse} 9 m \\
    \noindent\rule{\textwidth}{1pt} \vskip 2pt
    \setlength{\tabcolsep}{4pt}
      {\footnotesize 
    \begin{tabular}{cccccc}
      \bf FOR & \bf DEX & \bf CON & \bf INT & \bf SAG & \bf CHA \\
      11 (0) & 14 (+2) & 12 (+1) & 11 (0) & 13 (+1) & 11 (0) \\
    \end{tabular} }
    \setlength{\tabcolsep}{6pt}
    \noindent\rule{\textwidth}{1pt} \\
    {\bfseries Compétences} Discrétion +6, Nature +4, Perception +5, Survie +5 \\
    {\bfseries Sens} Perception passive 15 \\
    {\bfseries Langues} une langue au choix (généralement le commun) \\
    {\bfseries Facteur de puissance} 1/2 (100 XP) \\
    \noindent\rule{\textwidth}{1pt} \\
    {\bfseries Ouïe et vue aiguisées.} Un éclaireur a l'avantage aux jets de Sagesse (Perception) qui reposent sur l'ouïe ou la vue. 
  \end{minipage}
  \hspace{4pt}
  \begin{minipage}[c]{.45\linewidth}
\vspace{-10pt}
    \subsection*{Actions}
    {\bfseries Attaques multiples.} Un éclaireur réalise deux attaques au corps à corps ou deux attaques à distance. \\
    {\bfseries Épée courte.} Attaque au corps à corps avec une arme : +4 au toucher, allonge 1,50 m, une cible. Dégâts : 5 (1d6 + 2) dégâts perforants. \\
    {\bfseries Arc long.} Attaque à distance avec une arme : +4 au toucher, portée 45/180 m, une cible. Dégâts : 6 (1d8 + 2) dégâts perforants. \\
    \noindent\rule{\textwidth}{1pt} \\
Les éclaireurs sont des chasseurs et pisteurs entraînés qui proposent leurs services contre rémunération. La majorité chasse le gibier sauvage mais quelques-uns travaillent comme chasseurs de primes, servent de guides, ou effectuent des reconnaissances militaires.
  \end{minipage}
}%
\end{figure*}

