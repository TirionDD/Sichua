\begin{figure*}[hb!]
\fbox{%
  \begin{minipage}[c]{.45\linewidth}
    \label{Espion}
    {\bfseries\LARGE\scshape Éspion}\\
    {\small (System Reference Document trad. Aidedd.org)}\\
    Humanoïde (toute race) de taille M, tout alignement \\
    \noindent\rule{\textwidth}{1pt} \\
    {\bfseries Classe d'armure} 12 \\
    {\bfseries Points de vie} 27 (6d8) \\
    {\bfseries Vitesse} 9 m \\
    \noindent\rule{\textwidth}{1pt} \vskip 2pt
    \setlength{\tabcolsep}{4pt}
      {\footnotesize 
    \begin{tabular}{cccccc}
      \bf FOR & \bf DEX & \bf CON & \bf INT & \bf SAG & \bf CHA \\
      10 (0) & 15 (+2) & 10 (0) & 12 (+1) & 14 (+2) & 16 (+3) \\
    \end{tabular} }
    \setlength{\tabcolsep}{6pt}
    \noindent\rule{\textwidth}{1pt} \\
    {\bfseries Compétences} Discrétion +4, Escamotage +4, Investigation +5, Perception +6, Perspicacité +4, Persuasion +5, Tromperie +5 \\
    {\bfseries Sens} Perception passive 16 \\
    {\bfseries Langues} deux langues au choix \\
    {\bfseries Facteur de puissance} 1 (200 XP) \\
    \noindent\rule{\textwidth}{1pt} \\
    {\bfseries Attaque sournoise (1/tour).} Un espion inflige 7 (2d6) dégâts supplémentaires à une créature qu'il touche avec une arme s'il a l'avantage au jet d'attaque, ou si la cible est à 1,50 mètre ou moins d'un allié de l'espion qui n'est pas incapable d'agir et que l'espion n'a pas un désavantage au jet d'attaque.
  \end{minipage}
  \hspace{4pt}
  \begin{minipage}[c]{.45\linewidth}
    {\bfseries Action fourbe.} À chacun de ses tours, un espion peut utiliser une action bonus pour se désengager, se cacher ou courir. 
\vspace{-10pt}
    \subsection*{Actions}
    {\bfseries Attaques multiples.} Un espion réalise deux attaques au corps à corps. \\
    {\bfseries Épée courte.} Attaque au corps à corps avec une arme : +4 au toucher, allonge 1,50 m, une cible. Dégâts : 5 (1d6 + 2) dégâts perforant \\
    {\bfseries Arbalète de poing.} Attaque à distance avec une arme : +4 au toucher, portée 9/36 m, une cible. Dégâts : 5 (1d6 + 2) dégâts perforants. \\
    \noindent\rule{\textwidth}{1pt} \\
Les dirigeants, les nobles, les marchands, les maîtres de guilde et autres riches notables utilisent les espions pour avoir le dessus dans un monde à la politique sans foi ni loi. Un espion est entraîné à récolter secrètement de l'information. Les espions loyaux préféreraient mourir que divulguer des renseignements qui pourraient les compromettre, eux ou leurs employeurs.
  \end{minipage}
}%
\end{figure*}

