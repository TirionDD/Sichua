\begin{figure*}[hb!]
\fbox{%
  \begin{minipage}[c]{.45\linewidth}
    \label{DiableCornu}
    {\bfseries\LARGE\scshape Diable cornu}\\
    {\small (System Reference Document trad. Aidedd.org)}\\
    Fiélon (diable) de taille G, Loyal Mauvais \\
    \noindent\rule{\textwidth}{1pt} \\
    {\bfseries Classe d'armure} 18 (armure naturelle) \\
    {\bfseries Points de vie} 178 (17d10 + 85) \\
    {\bfseries Vitesse} 6 m, vol 18 m \\
    \noindent\rule{\textwidth}{1pt} \vskip 2pt
    \setlength{\tabcolsep}{4pt}
      {\footnotesize 
    \begin{tabular}{cccccc}
      \bf FOR & \bf DEX & \bf CON & \bf INT & \bf SAG & \bf CHA \\
      22 (+6) & 17 (+3) & 21 (+5) & 12 (+1) & 16 (+3) & 17 (+3) \\
    \end{tabular} }
    \setlength{\tabcolsep}{6pt}
    \noindent\rule{\textwidth}{1pt} \\
    {\bfseries Jets de sauvegarde} For +10, Dex +7, Sag +7, Cha +7 \\
    {\bfseries Résistances aux dégâts} froid ; contondant, perforant et tranchant d'attaques non magiques non réalisées avec des armes en argent \\
    {\bfseries Immunités aux dégâts} feu, poison \\
    {\bfseries Immunités aux conditions} empoisonné \\
    {\bfseries Sens} vision dans le noir à 36 m, Perception passive 13 \\
    {\bfseries Langues} infernal, télépathie à 36 m \\
    {\bfseries Facteur de puissance} 11 (7200 XP) \\
    \noindent\rule{\textwidth}{1pt} \\
    {\bfseries Vue de diable.} Des ténèbres magiques ne gênent pas la vision dans le noir du diable. \\
    {\bfseries Résistance à la magie.} Le diable a l'avantage aux jets de sauvegarde effectués contre les sorts ou tout autre effet magique. 
  \end{minipage}
  \hspace{4pt}
  \begin{minipage}[c]{.45\linewidth}
    \subsection*{Actions}
    {\bfseries Attaques multiples.} Le diable effectue trois attaques au corps à corps : deux attaques avec sa fourche et une attaque avec sa queue. Il peut utiliser son Jet de flammes à la place n'importe quelle attaque au corps à corps. \\
    {\bfseries Fourche.} Attaque au corps à corps avec une arme : +10 au toucher, allonge 3 m, une cible. Dégâts : 15 (2d8 + 6) dégâts perforants. \\
    {\bfseries Queue.} Attaque au corps à corps avec une arme : +10 au toucher, allonge 3 m, une cible. Dégâts : 10 (1d8 + 6) dégâts perforants. Si la cible est une créature autre qu'un mort-vivant ou une créature artificielle, elle doit réussir un jet de sauvegarde de Constitution DD 17 sous peine de perdre 10 (3d6) points de vie au début de chacun de ses tours suivants à cause de cette plaie infernale. Chaque fois que le diable frappe la créature blessée avec cette attaque, les dégâts infligés par la plaie augmentent de 10 (3d6). N'importe quelle créature peut utiliser une action pour endiguer l'hémorragie, à condition qu'elle réussisse un jet de Sagesse (Médecine) DD 12. La plaie se referme également si la cible reçoit des soins magiques. \\
    {\bfseries Jet de flammes.} Attaque à distance avec un sort : +7 au toucher, portée 45 m, une cible. Dégâts : 14 (4d6) dégâts de feu. Si la cible est un objet inflammable qui n'est pas porté ni tenu, elle prend feu.
  \end{minipage}
}%
\end{figure*}

