\begin{figure*}[hb!]
\fbox{%
  \begin{minipage}[c]{.45\linewidth}
    \label{DiableEnfer}
    {\bfseries\LARGE\scshape Chien d'enfer}\\
    {\small (System Reference Document trad. Aidedd.org)}\\
    Fiélon (diable) de taille M, Loyal Mauvais \\
    \noindent\rule{\textwidth}{1pt} \\
    {\bfseries Classe d'armure} 15 (armure naturelle) \\
    {\bfseries Points de vie} 45 (7d8 + 14) \\
    {\bfseries Vitesse} 15 m \\
    \noindent\rule{\textwidth}{1pt} \vskip 2pt
    \setlength{\tabcolsep}{4pt}
      {\footnotesize 
    \begin{tabular}{cccccc}
      \bf FOR & \bf DEX & \bf CON & \bf INT & \bf SAG & \bf CHA \\
      17 (+3) & 12 (+1) & 14 (+2) & 6 (-2) & 13 (+1) & 6 (-2) \\
    \end{tabular} }
    \setlength{\tabcolsep}{6pt}
    \noindent\rule{\textwidth}{1pt} \\
    {\bfseries Compétences} Perception +5 \\
    {\bfseries Immunités aux dégâts} feu \\
    {\bfseries Sens} vision dans le noir à 18 m, Perception Passive 15 \\
    {\bfseries Langues} comprend l'infernal mais ne peut pas parler \\
    {\bfseries Facteur de puissance} 3 (700 XP) \\
    \noindent\rule{\textwidth}{1pt} \\
    {\bfseries Odorat et ouïe aiguisés.} Le chien a l'avantage aux jets de Sagesse (Perception) faisant appel à l'odorat ou à l'ouïe. \\
    {\bfseries Tactique de meute.} Un chien a l'avantage aux jets d'attaque contre une créature si au moins l'un de ses alliés est à 1,50 mètre ou moins de la créature et n'est pas incapable d'agir.
  \end{minipage}
  \hspace{4pt}
  \begin{minipage}[c]{.45\linewidth}
    \subsection*{Actions}
    {\bfseries Morsure.} Attaque au corps à corps avec une arme : +5 au toucher, allonge 1,50 m, une cible. Dégâts : 7 (1d8 + 3) dégâts perforants plus 7 (2d6) de feu. \\
    {\bfseries Souffle de feu (Recharge 5-6).} Le chien exhale des flammes dans un cône de 4,50 mètres. Toutes les créatures dans la zone doivent réaliser un jet de sauvegarde de Dextérité DD 12, subissant 21 (6d6) dégâts de feu si le jet de sauvegarde est raté, ou la moitié des dégâts en cas de réussite. \\
    \noindent\rule{\textwidth}{1pt} \\
Fiélons au souffle enflammé prenant la forme de puissants chiens, les chiens d'enfer sont communément au service de créatures maléfiques qui les utilisent comme compagnons ou chiens de garde.
  \end{minipage}
}%
\end{figure*}

