\begin{figure*}[hb!]
\fbox{%
  \begin{minipage}[c]{.45\linewidth}
    \label{Mage}
    {\bfseries\LARGE\scshape Mage}\\
    {\small (System Reference Document trad. Aidedd.org)}\\
    Humanoïde (toute race) de taille M, tout alignement \\
    \noindent\rule{\textwidth}{1pt} \\
    {\bfseries Classe d'armure} 12 (15 avec armure de mage) \\
    {\bfseries Points de vie} 40 (9d8) \\
    {\bfseries Vitesse} 9 m \\
    \noindent\rule{\textwidth}{1pt} \vskip 2pt
    \setlength{\tabcolsep}{4pt}
      {\footnotesize 
    \begin{tabular}{cccccc}
      \bf FOR & \bf DEX & \bf CON & \bf INT & \bf SAG & \bf CHA \\
       9 (-1) & 14 (+2) & 11 (0) & 17 (+3) & 12 (+1) & 11 (0) \\
    \end{tabular} }
    \setlength{\tabcolsep}{6pt}
    \noindent\rule{\textwidth}{1pt} \\
    {\bfseries Jets de sauvegarde} Int +7, Sag +5 \\
    {\bfseries Compétences} Arcanes +7, Histoire +7 \\
    {\bfseries Sens} Perception passive 11 \\
    {\bfseries Langues} quatre langues au choix \\
    {\bfseries Facteur de puissance} 6 (2300 XP) \\
    \noindent\rule{\textwidth}{1pt} \\
    {\bfseries Sorts.} Un mage est un lanceur de sorts de niveau 9. Sa caractéristique pour lancer des sorts est l'Intelligence (sauvegarde contre ses sorts DD 15, +7 au toucher pour les attaques avec un sort). Un mage a les sorts de magicien suivant préparés : \\
Sorts mineurs (à volonté) : trait de feu, lumière, manipulation à distance, prestidigitation 
  \end{minipage}
  \hspace{4pt}
  \begin{minipage}[c]{.45\linewidth}
Niveau 1 (4 emplacements) : armure de mage, bouclier, détection de la magie, projectile magique \\
Niveau 2 (3 emplacements) : foulée brumeuse, suggestion \\
Niveau 3 (3 emplacements) : boule de feu, contresort, vol \\
Niveau 4 (3 emplacements) : invisibilité suprême, tempête de grêle \\
Niveau 5 (1 emplacement) : cône de froid
\vspace{-10pt}
    \subsection*{Actions}
    {\bfseries Dague.} Attaque au corps à corps ou à distance avec une arme : +6 au toucher, allonge 1,50 m ou portée 6/18 m, une cible. Dégâts : 4 (1d4 + 2) dégâts perforants. \\
    \noindent\rule{\textwidth}{1pt} \\
Les mages passent leur vie à étudier et expérimenter la magie. Les mages bons conseillent des nobles ou d'autres dirigeants, alors que les mages mauvais résident dans des lieux isolés pour y pratiquer leurs expériences impies sans ingérence.
  \end{minipage}
}%
\end{figure*}


