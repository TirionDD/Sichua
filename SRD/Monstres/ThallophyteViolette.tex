\begin{figure*}[hb!]
\fbox{%
  \begin{minipage}[c]{.45\linewidth}
    \label{ThallophyteViolette}
    {\bfseries\LARGE\scshape Thallophyte Violette}\\
    {\small (System Reference Document trad. Aidedd.org)}\\
    Plante de taille M, sans alignement \\
    \noindent\rule{\textwidth}{1pt} \\
    {\bfseries Classe d'armure} 5 \\
    {\bfseries Points de vie} 18 (4d8) \\
    {\bfseries Vitesse} 1,50 m \\
    \noindent\rule{\textwidth}{1pt} \vskip 2pt
    \setlength{\tabcolsep}{4pt}
      {\footnotesize 
    \begin{tabular}{cccccc}
      \bf FOR & \bf DEX & \bf CON & \bf INT & \bf SAG & \bf CHA \\
      3 (-4) & 1 (-5) & 10 (+0) &  1 (-5) & 3 (-4) &  1 (-5) \\
    \end{tabular} }
    \setlength{\tabcolsep}{6pt}
    \noindent\rule{\textwidth}{1pt} \\
    {\bfseries Immunités aux conditions} aveuglé, assourdi, effrayé \\
    {\bfseries Sens} vision aveugle à 9 m (aveugle au-delà de ce rayon), Perception passive 6 
  \end{minipage}
  \hspace{4pt}
  \begin{minipage}[c]{.45\linewidth}
    {\bfseries Langues} -\\
    {\bfseries Facteur de puissance} 1/4 (50 XP) \\
    \noindent\rule{\textwidth}{1pt} \\
    {\bfseries Fausse apparence.} Tant que la thallophyte violette reste inactive, on ne peut pas la 
               distinguer d'un champignon ordinaire.
\vspace{-10pt}
    \subsection*{Actions}
    {\bfseries Attaques multiples.} La thallophyte effectue 1d4 attaques de Contact pourrissant. \\
    {\bfseries Contact pourrissant.} Attaque au corps à corps avec une arme : +2 au toucher, allonge 3 m, 
               une créature. Dégâts : 4 (1d8) dégâts nécrotiques.
  \end{minipage}
}%
\end{figure*}
