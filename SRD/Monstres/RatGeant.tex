\begin{figure*}[hb!]
\fbox{%
  \begin{minipage}[c]{.45\linewidth}
    \label{RatGeant}
    {\bfseries\LARGE\scshape Rat géant}\\
    {\small (System Reference Document trad. Aidedd.org)}\\
    Bête de taille P, sans alignement \\
    \noindent\rule{\textwidth}{1pt} \\
    {\bfseries Classe d'armure} 12 \\
    {\bfseries Points de vie} 7 (2d6) \\
    {\bfseries Vitesse} 9 m \\
    \noindent\rule{\textwidth}{1pt} \vskip 2pt
    \setlength{\tabcolsep}{4pt}
      {\footnotesize 
    \begin{tabular}{cccccc}
      \bf FOR & \bf DEX & \bf CON & \bf INT & \bf SAG & \bf CHA \\
      7 (-2) & 15 (+2) & 11 (+0) &  2 (-4) & 10 (+0) &  4 (-3) \\
    \end{tabular} }
    \setlength{\tabcolsep}{6pt}
    \noindent\rule{\textwidth}{1pt} \\
    {\bfseries Sens} vision dans le noir à 18 m, Perception passive 10 \\
    {\bfseries Langues} -\\
    {\bfseries Facteur de puissance} 1/8 (25 XP)
  \end{minipage}
  \hspace{4pt}
  \begin{minipage}[c]{.45\linewidth}
\vspace{-10pt}
    {\bfseries Odorat aiguisé.} Un rat a l'avantage aux jets de Sagesse (Perception) faisant appel à l'odorat. \\
    {\bfseries Tactique de meute.} Un rat a l'avantage aux jets d'attaque contre une créature si au moins l'un de ses alliés est à 1,50 mètre ou moins de la créature et n'est pas incapable d'agir. 
    \subsection*{Actions}
    {\bfseries Morsure.} Attaque au corps à corps avec une arme : +4 au toucher, allonge 1,50 m, une cible. Dégâts : 4 (1d4 + 2) dégâts perforants.

  \end{minipage}
}%
\end{figure*}

