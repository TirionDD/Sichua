\begin{figure*}[hb!]
\fbox{%
  \begin{minipage}[c]{.45\linewidth}
    \label{Quasit}
    {\bfseries\LARGE\scshape Quasit}\\
    {\small (System Reference Document trad. Aidedd.org)}\\
    Fiélon (démon, métamorphe) de taille TP, chaotique mauvais \\
    \noindent\rule{\textwidth}{1pt} \\
    {\bfseries Classe d'armure} 13 \\
    {\bfseries Points de vie} 7 (3d4) \\
    {\bfseries Vitesse} 12 m \\
    \noindent\rule{\textwidth}{1pt} \vskip 2pt
    \setlength{\tabcolsep}{4pt}
      {\footnotesize 
    \begin{tabular}{cccccc}
      \bf FOR & \bf DEX & \bf CON & \bf INT & \bf SAG & \bf CHA \\
       5 (-3) & 17 (+3) & 10 (+0) &  7 (-2) & 10 (+0) & 10 (+0) \\
    \end{tabular} }
    \setlength{\tabcolsep}{6pt}
    \noindent\rule{\textwidth}{1pt} \\
    {\bfseries Compétences} Discrétion +5 \\
    {\bfseries Résistances aux dégâts} froid, feu, foudre ; contondant, perforant et tranchant d'attaques non magiques \\
    {\bfseries Immunités aux dégâts} poison \\
    {\bfseries Immunités aux conditions} empoisonné \\
    {\bfseries Sens} vision dans le noir à 36 m, Perception passive 10 \\
    {\bfseries Langues} abyssal, commun \\
    {\bfseries Facteur de puissance 1} (200 PX) \\
    \noindent\rule{\textwidth}{1pt} \\
    {\bfseries Métamorphe.} Le quasit peut utiliser son action pour prendre la forme d'une bête qui ressemble à une chauve-souris (vitesse 3 m, vol 12 m), un mille-pattes (12 m, escalade 12 m), ou un crapaud (12 m, nage 12 m), ou reprendre sa véritable forme. Ses statistiques sont les mêmes quelle que soit sa forme, à l'exception de sa vitesse (comme indiqué précédemment). L'équipement qu'il porte ou transporte n'est pas transformé avec lui. Il retrouve sa véritable forme s'il meurt.
  \end{minipage}
  \hspace{4pt}
  \begin{minipage}[c]{.45\linewidth}
    {\bfseries Résistance à la magie.} Le quasit a l'avantage aux jets de sauvegarde contre les sorts ou tout autre effet magique. \\
\vspace{-10pt}
    \subsection*{Actions}
    {\bfseries Griffes (Morsure sous forme de bête).} Attaque au corps à corps avec une arme : +4 au toucher, allonge 1,50 m, une cible. Dégâts : 5 (1d4 + 3) dégâts perforants, et la cible doit réussir un jet de sauvegarde de Constitution DD 10 sous peine de subir 5 (2d4) dégâts de poison et être empoisonné pendant 1 minute. La cible peut retenter son jet de sauvegarde à la fin de chacun de ses tours, mettant un terme à l'effet qui l'affecte en cas de réussite.
    {\bfseries Frayeur (1/jour).} Une créature que le quasit choisit, se trouvant dans un rayon de 6 mètres, doit réussir un jet de sauvegarde de Constitution DD 10 sous peine d'être effrayée pendant 1 minute. La cible peut retenter son jet de sauvegarde à la fin de chacun de ses tours, avec un désavantage si le quasit est dans sa ligne de mire, mettant un terme à l'effet qui l'affecte en cas de réussite.
    {\bfseries Invisibilité.} Le quasit devient invisible par magie jusqu'à ce qu'il attaque ou utilise sa Frayeur, ou jusqu'à ce que sa concentration s'arrête (comme s'il se concentrait sur un sort). L'équipement qu'il porte ou transporte est invisible en même temps que lui.
  \end{minipage}
}%
\end{figure*}

