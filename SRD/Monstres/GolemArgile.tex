%\makebox[.4\textwidth][c]{%
\begin{figure*}[hbp]
\fbox{%
  \begin{minipage}[c]{.45\linewidth}
    \label{GolemArgile}
    {\bfseries\LARGE\scshape Golem d'argile}\\
    {\small (System Reference Document trad. Aidedd.org)}\\
    Créature artificielle de taille G, sans alignement \\
    \noindent\rule{\textwidth}{1pt} \\
    {\bfseries Classe d'armure} 14 (armure naturelle) \\
    {\bfseries Points de vie} 133 (14d10 + 56) \\
    {\bfseries Vitesse} 6 m \\
    \noindent\rule{\textwidth}{1pt} \vskip 2pt
    \setlength{\tabcolsep}{4pt}
      {\footnotesize 
    \begin{tabular}{cccccc}
      \bf FOR & \bf DEX & \bf CON & \bf INT & \bf SAG & \bf CHA \\
      20 (+5) &  9 (-1) & 18 (+4) &  3 (-4) &  8 (-1) &  1 (-5) \\
    \end{tabular} }
    \setlength{\tabcolsep}{6pt}
    \noindent\rule{\textwidth}{1pt} \\
    {\bfseries Immunités aux dégâts} acide, poison, psychique ; contondant, perforant et tranchant d'attaques non magiques qui ne sont pas en adamantium \\
    {\bfseries Immunités aux conditions} charmé, empoisonné, épuisement, paralysé, pétrifié, effrayé \\
    {\bfseries Sens} vision dans le noir à 18 m, Perception passive 9\\
    {\bfseries Langues} comprend les langages de son créateur mais ne peut pas parler\\
    {\bfseries Facteur de puissance} 9 (5000 XP)\\
    \noindent\rule{\textwidth}{1pt} \\
    {\bfseries Absorption de l'acide.} Lorsque le golem subit des dégâts d'acide, il ne subit aucun dommage 
               et récupère un nombre de points de vie égal aux dégâts d'acide causés. \\
    {\bfseries Folie Dévastatrice.} Chaque fois que le golem débute son tour avec 60 points de vie ou 
               moins, lancez un d6. Sur un résultat de 6, le golem entre dans un état de folie 
               dévastatrice. Lors de chacun de ses tours de jeu, le golem attaque la créature la plus 
               proche qu'il peut voir. Si aucune créature n'est assez proche de lui pour qu'il puisse se 
               déplacer et l'attaquer, il attaquera un objet, de préférence plus petit que lui.
  \end{minipage}
  \hspace{4pt}
  \begin{minipage}[c]{.45\linewidth}
               Une fois 
               que le golem est entré en folie dévastatrice, il demeure dans cet état jusqu'à ce qu'il soit 
               détruit ou jusqu'à ce qu'il récupère tous ses points de vie. \\
    {\bfseries Forme Immuable.} Le golem est immunisé aux sorts et effets qui altèreraient son apparence. \\
    {\bfseries Armes Magiques.} Les attaques d'arme du golem sont magiques. \\
    {\bfseries Résistance à la magie.} Le golem a l'avantage aux jets de sauvegarde contre les sorts et les 
               effets magiques.
\vspace{-10pt}
    \subsection*{Actions}
    {\bfseries Attaques multiples.} Le golem réalise deux attaques de coup.\\
    {\bfseries Coup.} Attaque d'arme de corps à corps : +8 au toucher, allonge 1,50 m, une cible. 
               Dégâts : 16 (2d10 + 5) dégâts contondants. Si la cible est une créature, celle-ci doit 
               réussir un jet de sauvegarde de Constitution DD 15 pour ne pas subir une diminution de ses 
               points de vie maximums égale aux dégâts subis. La cible meurt si cet effet réduit ses points 
               de vie à 0. Cette diminution perdure jusqu'à ce que la cible bénéficie d'un sort de 
               restauration suprême ou d'une magie similaire. \\
    {\bfseries Hâte (Recharge 5-6).} Jusqu'à la fin de son prochain tour de jeu, le golem bénéficie par 
               magie d'un bonus de +2 à sa CA, de l'avantage à ses jets de sauvegarde de Dextérité, et peut 
               utiliser une attaque de coup en action bonus.
  \end{minipage}
}%
\end{figure*}
%}%

