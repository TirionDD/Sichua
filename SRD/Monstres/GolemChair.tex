%\makebox[.4\textwidth][c]{%
\begin{figure*}[hbp]
\fbox{%
  \begin{minipage}[c]{.45\linewidth}
    \label{GolemChair}
    {\bfseries\LARGE\scshape Golem de chair}\\
    {\small (System Reference Document trad. Aidedd.org)}\\
    Créature artificielle de taille M, neutre \\
    \noindent\rule{\textwidth}{1pt} \\
    {\bfseries Classe d'armure} 9 \\
    {\bfseries Points de vie} 93 (11d8 + 44) \\
    {\bfseries Vitesse} 9 m \\
    \noindent\rule{\textwidth}{1pt} \vskip 2pt
    \setlength{\tabcolsep}{4pt}
      {\footnotesize 
    \begin{tabular}{cccccc}
      \bf FOR & \bf DEX & \bf CON & \bf INT & \bf SAG & \bf CHA \\
      19 (+4) &  9 (-1) & 18 (+4) &  6 (-2) & 10 (+0) &  5 (-3) \\
    \end{tabular} }
    \setlength{\tabcolsep}{6pt}
    \noindent\rule{\textwidth}{1pt} \\
    {\bfseries Immunités aux dégâts} foudre, poison ; contondant, perforant et tranchant d'attaques non magiques qui ne sont pas en adamantium \\
    {\bfseries Immunités aux conditions} charmé, empoisonné, épuisement, paralysé, pétrifié, effrayé \\
    {\bfseries Sens} vision dans le noir à 18 m, Perception passive 10\\
    {\bfseries Langues} comprend les langages de son créateur mais ne peut pas parler\\
    {\bfseries Facteur de puissance} 5 (1800 XP)\\
    \noindent\rule{\textwidth}{1pt} \\
    {\bfseries Folie Dévastatrice.} Chaque fois que le golem débute son tour avec 40 points de vie ou 
               moins, lancez un d6. Sur un résultat de 6, le golem entre dans un état de folie 
               dévastatrice : lors de chacun de ses tours de jeu, il attaque la créature la plus proche 
               qu'il peut voir. Si aucune créature n'est assez proche de lui pour qu'il puisse se déplacer 
               et l'attaquer, il attaquera un objet, de préférence plus petit que lui. Une fois que le 
               golem est entré en folie dévastatrice, il demeure dans cet état jusqu'à ce qu'il soit 
               détruit ou jusqu'à ce qu'il récupère tous ses points de vie. Le créateur du golem, s'il se 
               trouve à 18 mètres de lui, peut tenter de le calmer en lui parlant fermement et de manière 
               persuasive.   \end{minipage}
  \hspace{4pt}
  \begin{minipage}[c]{.45\linewidth}
               Le golem doit pouvoir entendre son créateur, qui doit utiliser une action pour 
               réaliser un jet de Charisme (Persuasion) DD~15.
               Si le jet est réussi, la folie dévastatrice 
               du golem cesse. S'il subit des dégâts alors que ses points de vie sont à 40 ou moins, le 
               golem peut encore entrer en état de folie dévastatrice. \\
    {\bfseries Absorption de la foudre.} Lorsque le golem subit des dégâts de foudre, il ne subit aucun 
               dommage et récupère un nombre de points de vie égal aux dégâts de foudre causés. \\
    {\bfseries Aversion au Feu.} Si le golem subit des dégâts de feu, il subit un désavantage à ses jets 
               d'attaques et de caractéristiques jusqu'à la fin de son tour de jeu. \\
    {\bfseries Forme Immuable.} Le golem est immunisé aux sorts et effets qui altèreraient son apparence. \\
    {\bfseries Armes Magiques.} Les attaques d'arme du golem sont magiques. \\
    {\bfseries Résistance à la magie.} Le golem a l'avantage aux jets de sauvegarde contre les sorts et les 
               effets magiques.
\vspace{-10pt}
    \subsection*{Actions}
    {\bfseries Attaques multiples.} Le golem réalise deux attaques de coup.\\
    {\bfseries Coup.} Attaque d'arme de corps à corps : +7 au toucher, allonge 1,50 m, une cible. 
               Dégâts : 13 (2d8 + 4) dégâts contondants. \\
    \noindent\rule{\textwidth}{1pt} \\
    Un golem de chair est le macabre assortiment de plusieurs parties de cadavres humanoïdes, cousues et 
    vissées les unes aux autres, dans le but de créer une brute vigoureuse possédant une force prodigieuse. 
    De puissants enchantements le protègent, repoussant les sortilèges et les armes courantes.
  \end{minipage}
}%
\end{figure*}
%}%

