\begin{figure*}[hb!]
\fbox{%
  \begin{minipage}[c]{.45\linewidth}
    \label{Archimage}
    {\bfseries\LARGE\scshape Archimage}\\
    {\small (System Reference Document trad. Aidedd.org)}\\
    Humanoïde (toute race) de taille M, tout alignement \\
    \noindent\rule{\textwidth}{1pt} \\
    {\bfseries Classe d'armure} 12 (15 avec armure de mage) \\
    {\bfseries Points de vie} 99 (18d8 + 18) \\
    {\bfseries Vitesse} 9 m \\
    \noindent\rule{\textwidth}{1pt} \vskip 2pt
    \setlength{\tabcolsep}{4pt}
      {\footnotesize 
    \begin{tabular}{cccccc}
      \bf FOR & \bf DEX & \bf CON & \bf INT & \bf SAG & \bf CHA \\
       10 (0) & 14 (+2) & 12 (+1) & 20 (+5) & 15 (+2) & 16 (+3) \\
    \end{tabular} }
    \setlength{\tabcolsep}{6pt}
    \noindent\rule{\textwidth}{1pt} \\
    {\bfseries Jets de sauvegarde} Int +9, Sag +6 \\
    {\bfseries Compétences} Arcanes +13, Histoire +13 \\
    {\bfseries Résistances aux dégâts} dégâts des sorts ; contondants, perforants et tranchants non magiques (sort peau de pierre) \\
    {\bfseries Sens} Perception passive 12 \\
    {\bfseries Langues} six langues au choix \\
    {\bfseries Facteur de puissance} 12 (8400 XP) \\
    \noindent\rule{\textwidth}{1pt} \\
    {\bfseries Résistance à la magie.} L'archimage a l'avantage aux jets de sauvegarde contre les sorts et autres effets magiques. \\
    \noindent\rule{\textwidth}{1pt} \\
    {\bfseries Sorts.} L'archimage est un lanceur de sorts de niveau 18. L'Intelligence est la caractéristique utilisée pour lancer des sorts (DD de sauvegarde d'un sort 17, +9 de modificateur à l'attaque d'un sort). L'archimage peut lancer à volonté les sorts déguisement et invisibilité. L'archimage a les sorts de magiciens suivants préparés : \\
Sorts mineurs (à volonté) : trait de feu, lumière, manipulation à distance, prestidigitation, décharge électrique \\
Niveau 1 (4 emplacements) : détection de la magie, identification, armure de mage$^\star$, projectile magique 
  \end{minipage}
  \hspace{4pt}
  \begin{minipage}[c]{.45\linewidth}
Niveau 2 (3 emplacements) : détection de pensées, image miroir, foulée brumeuse \\
Niveau 3 (3 emplacements) : contresort, vol, éclair \\
Niveau 4 (3 emplacements) : bannissement, bouclier de feu, peau de pierre$^\star$ \\
Niveau 5 (3 emplacements) : cône de froid, scrutation, mur de force \\
Niveau 6 (1 emplacement) : globe d'invulnérabilité \\
Niveau 7 (1 emplacement) : téléportation \\
Niveau 8 (1 emplacement) : esprit impénétrable$^\star$ \\
Niveau 9 (1 emplacement) : arrêt du temps \\
$^\star$ L'archimage jette ces sorts avant le combat. \\
\vspace{-10pt}
    \subsection*{Actions}
    {\bfseries Dague.} Attaque au corps à corps ou à distance avec une arme : +8 au toucher, allonge 1,50 m ou portée 6/18 m, une cible. Dégâts : 4 (1d4 + 2) dégâts perforants. \\
    \noindent\rule{\textwidth}{1pt} \\
Les archimages sont de puissants (et généralement assez âgés) jeteurs de sorts qui consacrent leur vie à l'étude des arcanes. Les plus bienveillants d'entre eux conseillent les rois et les reines, tandis que les malveillants règnent en tyrans et cherchent à devenir des liches. Ceux qui ne tendent ni vers le bien, ni vers le mal, s'enferment dans des tours isolées afin de pratiquer leur magie sans risque d'être dérangés. \\
Un archimage a généralement un ou plusieurs apprentis, et sa demeure possède de nombreux gardiens et protections magiques destinés à décourager les intrus.
  \end{minipage}
}%
\end{figure*}

