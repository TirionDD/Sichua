\begin{figure*}[hb!]
\fbox{%
  \begin{minipage}[c]{.45\linewidth}
    \label{Succubus}
    {\bfseries\LARGE\scshape Succubus}\\
    {\small (System Reference Document trad. Aidedd.org)}\\
    Fiélon (métamorphe) de taille M, neutre mauvais \\
    \noindent\rule{\textwidth}{1pt} \\
    {\bfseries Classe d'armure} 15 (armure naturelle) \\
    {\bfseries Points de vie} 66 (12d8 + 12) \\
    {\bfseries Vitesse} 9 m, vol 18 m \\
    \noindent\rule{\textwidth}{1pt} \vskip 2pt
    \setlength{\tabcolsep}{4pt}
      {\footnotesize 
    \begin{tabular}{cccccc}
      \bf FOR & \bf DEX & \bf CON & \bf INT & \bf SAG & \bf CHA \\
       8 (-1) & 17 (+3) & 15 (+2) & 15 (+2) & 12 (+1) & 20 (+5) \\
    \end{tabular} }
    \setlength{\tabcolsep}{6pt}
    \noindent\rule{\textwidth}{1pt} \\
    {\bfseries Compétences} Discrétion +7, Perception +5, Perspicacité +5, Persuasion +9, Tromperie +9 \\
    {\bfseries Résistances aux dégâts} froid, feu, foudre, poison ; contondant, perforant et tranchant d'attaques non magiques \\
    {\bfseries Sens} vision dans le noir à 18 m, Perception passive 15 \\
    {\bfseries Langues} abyssal, commun, infernal, télépathie à 18 m \\
    {\bfseries Facteur de puissance 4} (1100 XP) \\
    \noindent\rule{\textwidth}{1pt} \\
    {\bfseries Lien télépathique.} Le fiélon ignore la restriction de portée de sa télépathie quand il communique avec une créature qu'il a charmée. Les deux individus n'ont même pas besoin d'être sur le même plan d'existence. \\
    {\bfseries Métamorphe.} Le fiélon peut utiliser son action pour se métamorphoser en un humanoïde de taille P ou M, ou pour retrouver son apparence originale. Sans ses ailes, le fiélon perd sa capacité de vol. À part sa taille et sa vitesse de mouvement, ses statistiques sont les mêmes pour chaque forme. L'équipement qu'il porte ou transporte n'est pas transformé avec lui. Il retrouve sa forme originale s'il meurt.
  \end{minipage}
  \hspace{4pt}
  \begin{minipage}[c]{.45\linewidth}
\vspace{-10pt}
    \subsection*{Actions}
    {\bfseries Griffes (forme fiélone uniquement).} Attaque au corps à corps avec une arme : +5 au toucher, allonge 1,50 m, une cible. Dégâts : 6 (1d6 + 3) dégâts tranchants. \\
    {\bfseries Charme.} Un humanoïde que le fiélon peut voir et qui se situe à 9 mètres de lui doit réussir un jet de sauvegarde de Sagesse DD 15 ou être charmé par magie pendant 1 journée. La cible charmée obéit aux commandes verbales ou télépathiques du fiélon. Si la cible subit des dégâts ou reçoit un ordre suicidaire, celle-ci peut retenter le jet de sauvegarde, mettant fin à l'effet en cas de succès. Si le jet de sauvegarde de la créature est réussi ou que l'effet qu'elle subit se termine, celle-ci devient immunisée au Charme du fiélon pendant les prochaines 24 heures. Le fiélon ne peut avoir qu'une cible charmée à la fois. S'il charme une autre cible, l'effet se termine. \\
    {\bfseries Baiser corrupteur.} Le fiélon embrasse une créature qu'il a charmée ou une créature volontaire. La cible doit réaliser un jet de sauvegarde de Constitution DD 15 contre cette magie, subissant 32 (5d10 + 5) dégâts psychiques en cas d'échec, ou la moitié des dégâts en cas de succès. Les points de vie maximums de la cible sont diminués d'un montant égal aux dégâts subis. Cette diminution perdure jusqu'à ce que la cible termine un repos long. La cible meurt si cet effet réduit ses points de vie maximums à 0. \\
    {\bfseries Forme éthérée.} Le fiélon pénètre dans le plan Éthéré à partir du plan Matériel, ou vice versa.
  \end{minipage}
}%
\end{figure*}

ACTIONS
