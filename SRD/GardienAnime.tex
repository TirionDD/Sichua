%\makebox[.4\textwidth][c]{%
\begin{figure*}[hbp]
\fbox{%
  \begin{minipage}[c]{.45\linewidth}
    \label{GardienAnime}
    {\bfseries\LARGE\scshape Gardien animé}\\
    {\small (System Reference Document trad. Aidedd.org)}\\
    Créature artificielle de taille G, sans alignement \\
    \noindent\rule{\textwidth}{1pt} \\
    {\bfseries Classe d'armure} 17 (armure naturelle) \\
    {\bfseries Points de vie} 142 (15d10 + 60) \\
    {\bfseries Vitesse} 9 m \\
    \noindent\rule{\textwidth}{1pt} \vskip 2pt
    \setlength{\tabcolsep}{4pt}
      {\footnotesize 
    \begin{tabular}{cccccc}
      \bf FOR & \bf DEX & \bf CON & \bf INT & \bf SAG & \bf CHA \\
      18 (+4) &  8 (-1) & 18 (+4) &  7 (-2) &  10 (+0) &  3 (-4) \\
    \end{tabular} }
    \setlength{\tabcolsep}{6pt}
    \noindent\rule{\textwidth}{1pt} \\
    {\bfseries Immunités aux dégâts} poison \\
    {\bfseries Immunités aux conditions} charmé, épuisement, effrayé, paralysé, empoisonné \\
    {\bfseries Sens} vision aveugle à 3 m, vision dans le noir à 18 m, Perception passive 10\\
    {\bfseries Langues} comprend les ordres donnés dans toutes les langues mais ne peut pas parler\\
    {\bfseries Facteur de puissance} 7 (2900 XP)\\
    \noindent\rule{\textwidth}{1pt} \\
    {\bfseries Lien.} Le gardien animé est connecté magiquement à une amulette. Tant que le gardien et son 
               amulette sont dans le même plan d'existence, le porteur de l'amulette peut appeler 
               télépathiquement le gardien animé à le rejoindre, et le gardien sait automatiquement à 
               quelle distance et dans quelle direction se trouve l'amulette. Si le gardien se trouve à 
               18 mètres ou moins du porteur de l'amulette, la moitié des dégâts que le porteur subit 
               (arrondis au supérieur) est transférée au gardien.\\
    {\bfseries Régénération.} Le gardien animé récupère 10 points de vie au début de son tour de jeu s'il 
               lui reste au moins 1 point de vie.
  \end{minipage}
  \hspace{4pt}
  \begin{minipage}[c]{.45\linewidth}
    {\bfseries Réservoir de sort.} Un lanceur de sorts qui porte l'amulette du gardien animé peut faire en 
               sorte que le garde animé emmagasine un sort de niveau 4 ou inférieur. Pour ce faire, le 
               porteur de l'amulette doit lancer le sort sur le gardien. Le sort n'a pas d'effet mais il 
               est stocké dans le gardien. Lorsque le porteur de l'amulette le lui ordonne, ou lorsqu'une 
               situation définie à l'avance par le lanceur de sorts survient, le gardien animé lance le 
               sort emmagasiné avec tous les paramètres établis par le lanceur de sorts originel, et sans 
               avoir besoin de composantes. Lorsque le sort est lancé ou lorsqu'un nouveau sort est 
               emmagasiné, tout sort précédemment stocké est perdu.
    \subsection*{Actions}
    {\bfseries Attaques multiples.} Le golem réalise deux attaques de poing.\\
    {\bfseries Poing.} Attaque d'arme de corps à corps : +7 au toucher, allonge 1,50 m, une cible. 
               Dégâts : 11 (2d6 + 4) dégâts contondants.
    \subsection*{Réactions}
    {\bfseries Bouclier.} Lorsqu'une créature effectue une attaque contre le porteur de l'amulette, le 
               gardien animé confère au porteur de l'amulette un bonus de +2 à la CA s'ils se trouvent à 
               1,50 mètre ou moins l'un de l'autre.
  \end{minipage}
}%
\end{figure*}
%}%

